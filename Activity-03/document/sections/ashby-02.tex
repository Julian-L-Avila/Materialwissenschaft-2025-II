\section{Actividad 02 -- Diagramas de Ashby}

\begin{problem}[Física de los Hornos]
	Explicar el principio físico de un horno de fundición (ferrería),
	reverbero, horno Bessemer, horno de arco, inducción, inducción al
	vacío y horno de tratamiento térmico para la metalurgia.
\end{problem}

\subsection{Alto Horno}

El alto horno es un reactor químico diseñado para la reducción
carbotérmica de minerales de hierro. Su principio operativo se
fundamenta en una gestión rigurosa del flujo de materia y energía
en contracorriente. Una carga sólida (mineral, coque, fundente)
desciende por gravedad, mientras que gases calientes ascienden desde la
base.

La generación de energía térmica es iniciada por la combustión de coque
con aire precalentado ($T > \qty{1200}{\kelvin}$), inyectado por toberas:
\ce{C + O2 -> CO2}. Esta reacción eleva la
temperatura por encima de \qty{2000}{\kelvin}. El \ce{CO2} reacciona
inmediatamente con carbono incandescente en la reacción endotérmica de
Boudouard, \ce{C + CO2 <=> 2CO},
generando el principal agente reductor, el monóxido de carbono.

Desde una perspectiva termodinámica, el proceso es viable porque la
energía libre de Gibbs ($ \Delta G $) para la oxidación del carbono
se vuelve más negativa que la de los óxidos de hierro a altas
temperaturas. La reducción del hierro ocurre secuencialmente a medida que la
carga desciende a zonas de mayor temperatura:
$ \ce{3Fe2O3 -> 2Fe3O4 -> FeO -> Fe} $.

El transporte de calor es un proceso complejo que involucra convección
forzada, donde el número de Péclet es elevado, y radiación térmica
desde las zonas incandescentes hacia la carga porosa.

\subsection{Horno de Reverbero}

A diferencia del alto horno, este diseño aísla la carga del
combustible para evitar su contaminación. El principio físico
fundamental es la maximización de la transferencia de calor por
radiación. Una llama y gases calientes fluyen a lo largo de un techo
refractario abovedado, que se calienta hasta la incandescencia.

Este techo actúa como un cuerpo radiante secundario, emitiendo energía
hacia la carga metálica situada en la solera. El flujo de calor
radiante ($q$) se rige por la ley de Stefan-Boltzmann,
$ q = \sigma \epsilon F_{1-2} (T_1^4 - T_2^4) $, donde $T_1$ es la
temperatura del techo, $T_2$ la de la carga, $\epsilon$ la emisividad y
$F_{1-2}$ es el factor de forma (o de visión), un término geométrico
que cuantifica la fracción de energía que intercepta la carga. La
geometría del horno es, por tanto, un parámetro de diseño crítico.

\subsection{Convertidor Bessemer}

La física del convertidor Bessemer es la de un reactor químico
autógeno, cuya energía proviene de reacciones de oxidación exotérmicas
\emph{in situ}. No requiere fuente de calor externa. Se inyecta aire u
oxígeno a presión a través del arrabio fundido, induciendo la oxidación
selectiva de impurezas.

La termodinámica dicta la secuencia de oxidación: elementos con mayor
afinidad por el oxígeno, como el silicio
(\ce{Si + O2 -> SiO2}) y el manganeso
(\ce{2Mn + O2 -> 2MnO}), reaccionan primero,
formando una escoria. Finalmente, se oxida el carbono,
\ce{2C + O2 -> 2CO}, cuya violenta formación
gaseosa provoca una ebullición característica.

Desde la fluidodinámica, la inyección del gas genera una turbulencia
extrema y un área interfacial gas-líquido inmensa, lo que resulta en
una cinética de transferencia de masa extraordinariamente rápida,
completando el afino del acero en minutos.

\subsection{Horno de Arco Eléctrico}

Su principio físico es la generación de un arco voltaico de alta
potencia entre electrodos de grafito y la carga metálica. El arco es un
plasma, un gas ionizado estable cuyas temperaturas pueden exceder los
\qty{6000}{\kelvin}.

La transferencia de energía al metal es un proceso multifísico:
\begin{enumerate}
	\item \textbf{Radiación térmica:} Es el mecanismo dominante. La
		columna de plasma irradia energía intensamente según la ley de
		Stefan-Boltzmann.
	\item \textbf{Transferencia en los electrodos:} Bombardeo directo de
		electrones e iones en los puntos de anclaje del arco con el baño
		metálico.
	\item \textbf{Convección:} Transferencia de calor desde el plasma
		caliente al metal.
\end{enumerate}
La potencia eléctrica disipada, $ P = V_{\text{arc}} I_{\text{arc}} $, se
controla con precisión para optimizar la fusión de la chatarra.

\subsection{Calentamiento por Inducción}

Este horno opera bajo los principios del electromagnetismo. Una
corriente alterna ($ I(t) $) fluye por una bobina que envuelve un
crisol con la carga metálica. Por la ley de Ampère, esta corriente
genera un campo magnético variable en el tiempo, $ B(t) $.

La ley de Faraday, $ \nabla \wedge E = - \pdv{B}!{t} $,\footnote{Se emplea uso del
formalismo del álgebra de Clifford $\mathcal{C}\ell_{3} (\mathbb{R})$, donde $E$
es un vector y $B$ un bivector.} dicta que $B(t)$ induce
un campo eléctrico $E$. Este campo impulsa la circulación de corrientes
parásitas (Eddy currents), $J = \sigma_e E$.

El calentamiento se produce por efecto Joule, con una densidad de
potencia volumétrica dada por $ p = J \cdot E = J^2 \sigma^{-1}_e $. Un
fenómeno clave es el efecto piel (\emph{skin effect}), que confina a
$J$ a una capa superficial de profundidad
$ \delta = \sqrt{2 \rho_e (\omega \mu)^{-1}} $, donde $\rho_e$ es la
resistividad, $\omega$ la frecuencia angular y $\mu$ la permeabilidad
magnética.

Adicionalmente, la interacción del campo $B$ con las corrientes $J$
genera una fuerza de Lorentz ($ F = J \cdot B $), que induce un
flujo de agitación (stirring) en el metal líquido, garantizando una
homogeneización térmica y química superior.

\subsection{Inducción al Vacío (VIM)}

El horno VIM acopla la física de la inducción con los principios de la
termodinámica en baja presión. Al operar en vacío, la presión parcial
de gases ($p_i$) sobre el metal se reduce drásticamente.

Según la ley de Sieverts, la solubilidad de gases diatómicos como
\ce{N2}, \ce{O2} y \ce{H2} en el metal es proporcional a $\sqrt{p_i}$.
Una disminución de $p_i$ reduce el potencial químico del gas en la
atmósfera, impulsando su desorción del metal fundido para alcanzar un
nuevo equilibrio.

El vacío también facilita la eliminación de elementos volátiles con
alta presión de vapor y desplaza equilibrios de desoxidación como
\ce{[C] + [O] <=> CO_{(g)}}.
La extracción continua del producto gaseoso CO, conforme al principio
de Le Châtelier, lleva la reacción a completarse.

\subsection{Tratamiento Térmico: Física del Estado Sólido}

Estos hornos manipulan la microestructura del material en estado
sólido. El principio físico subyacente es la cinética de las
transformaciones de fase, controlada por ciclos térmicos precisos.

El calentamiento de una pieza se rige por la ecuación de difusión del
calor, $ \pdv{T}!{t} = \alpha \nabla^2 T $, donde $\alpha$ es la
difusividad térmica. El objetivo es llevar el material a una región de
su diagrama de fases donde una nueva estructura cristalina sea estable
(e.g., austenización del acero).

La etapa crítica es el enfriamiento. La velocidad, $ \mdif{t}{T} $,
determina la transformación de fase resultante.
\begin{itemize}
	\item \textbf{Enfriamiento lento:} Permite transformaciones por
		difusión (nucleación y crecimiento), formando microestructuras
		como ferrita, perlita o bainita.
	\item \textbf{Enfriamiento rápido (temple):} Suprime la difusión. La
		transformación ocurre de forma displaciva (sin difusión), mediante
		un mecanismo de cizallamiento de la red cristalina, dando lugar a
		fases metaestables como la martensita.
\end{itemize}

\begin{problem}[Diagrama de Ashby]
	Interprete el diagrama de Ashby presentado en la \cref{fig:ashby-3}.
	\begin{figure}[htbp!]
		\centering
		\includegraphics[width=0.5\linewidth]{./images/ashby.png}
		\caption{Diagrama de Ashby (Módulo de Young y Densidad).}
		\label{fig:ashby-3}
	\end{figure}
\end{problem}

\subsubsection{Eje Y: Módulo de Young, \texorpdfstring{$E$ (\si{\giga\pascal})}{E (GPa)}}

Esta propiedad cuantifica la rigidez de un material en el régimen
elástico. Se define como la constante de proporcionalidad entre el
esfuerzo uniaxial ($\sigma$) y la deformación unitaria ($\epsilon$)
según la Ley de Hooke, $\sigma = E \epsilon$.

El módulo de Young está determinado por la naturaleza de los enlaces interatómicos.
Es proporcional a la curvatura del potencial de energía interatómica $U(r)$ en
la distancia de equilibrio $r_0$:
$$ E \propto \left( \mdif{r}^2 U \right)_{r=r_0} $$
Un valor de $E$ elevado implica enlaces fuertes y una gran resistencia a
la deformación elástica. El rango logarítmico del eje abarca siete
órdenes de magnitud, evidenciando la vasta diversidad de materiales.

\subsubsection{Eje X: Densidad, \texorpdfstring{$\rho$
(\si{\mega\gram\per\cubic\meter})}{rho (Mg/m3)}}

La densidad es la masa por unidad de volumen. Microscópicamente, está
determinada por tres factores:
\begin{enumerate}
	\item La masa atómica ($A$) de los elementos constituyentes.
	\item El factor de empaquetamiento atómico (APF), una propiedad
		geométrica de la estructura cristalina o molecular.
	\item La porosidad presente en el material.
\end{enumerate}

\subsection{Interpretación de las Familias de Materiales}

Las regiones en el diagrama agrupan materiales con características
físicas y estructurales similares.

\subsubsection{Metales (Gris)}

Se ubican en la región superior derecha: alta rigidez y alta densidad.
Su enlace metálico, caracterizado por un gas de electrones
deslocalizados, proporciona una fuerte cohesión no direccional. Esto,
sumado a sus estructuras cristalinas compactas (e.g., FCC, BCC con APF
elevados), resulta en módulos de Young altos y densidades notables.

\subsubsection{Cerámicas Técnicas (Rojo)}

Ocupan la cúspide del diagrama, exhibiendo la máxima rigidez. Sus
enlaces, predominantemente iónicos (fuertes, no direccionales) y/o
covalentes (muy fuertes y direccionales), exigen una energía muy alta
para deformarse. Su densidad es moderada, menor que la de muchos
metales, pues suelen componerse de átomos más ligeros (\ce{O}, \ce{Si}, \ce{Al},
\ce{C}).

\subsubsection{Polímeros y Elastómeros (Azul)}

Se localizan en la región central e inferior, con baja rigidez y
densidad. Su comportamiento mecánico no está dominado por el
estiramiento de enlaces primarios, sino por interacciones secundarias
(fuerzas de van der Waals) y por la rotación de enlaces en las largas
cadenas moleculares. En los elastómeros, la rigidez es particularmente
baja y su origen es fundamentalmente entrópico: la fuerza restauradora
proviene de la tendencia del sistema a maximizar su entropía
conformacional (retorno al estado de ovillo estadístico).

\subsubsection{Compuestos (Amarillo)}

Estos materiales de ingeniería se posicionan estratégicamente para
optimizar la relación rigidez-peso. Combinan una matriz (usualmente
polimérica, de baja densidad) con un refuerzo de alta rigidez (e.g.,
fibras de carbono, vidrio). Su objetivo es maximizar el módulo
específico, $E \rho^{-1}$, un índice de rendimiento clave. Por ello, se
ubican en la deseable región superior izquierda del diagrama.

\subsubsection{Materiales Naturales (Marrón)}

Exhiben una gran diversidad, a menudo como compuestos jerárquicos
optimizados por la evolución. La madera, un compuesto celular de fibras
de celulosa en una matriz de lignina, logra una excelente rigidez
específica.

\subsubsection{Espumas (Verde)}

Ocupan la esquina inferior izquierda: densidad y rigidez extremadamente
bajas. Son estructuras celulares con una alta fracción de volumen de
gas. Su respuesta mecánica no está dictada por las propiedades del
material base, sino por la arquitectura de la espuma. La deformación se
produce por la flexión y el pandeo de las paredes celulares.

\begin{problem}[Material para una taza]
	Para fabricar una taza, ¿qué se debe tener en cuenta en la selección
	del material desde una perspectiva de la física de materiales?
\end{problem}

La selección de un material para una taza es un problema de diseño que
requiere satisfacer un conjunto de restricciones termofísicas,
mecánicas y químicas.

\subsubsection{Restricciones Termofísicas}

\paragraph{Resistencia al Choque Térmico.} El material debe soportar
gradientes de temperatura abruptos sin fracturarse. Al
verter un líquido caliente, la superficie interna se expande
repentinamente, generando un esfuerzo de tracción en la superficie
externa (más fría).

\paragraph{Aislamiento Térmico.} Para mantener la bebida caliente y el
exterior a una temperatura manejable, se requiere una baja
conductividad térmica, $k$. A nivel microscópico, esto implica una baja
eficiencia en la propagación de fonones (cuantos de vibración de la
red). Materiales con estructuras cristalinas complejas, desorden
atómico (vidrios) o grandes diferencias de masa atómica favorecen la
dispersión de fonones y, por tanto, poseen un bajo valor de $k$.

\subsubsection{Restricciones Mecánicas}

\paragraph{Rigidez y Resistencia a la Compresión.} El material debe ser
suficientemente rígido ($E$ elevado) y resistente a la compresión para
soportar su propio peso y el del líquido sin deformarse. Las cerámicas
y vidrios, con sus fuertes enlaces iónicos/covalentes, cumplen
excelentemente este requisito.

\paragraph{Fragilidad y Tenacidad a la Fractura.} La contrapartida de
los enlaces cerámicos es su fragilidad. La estructura atómica inhibe el
deslizamiento de dislocaciones, impidiendo la deformación plástica como
mecanismo de disipación de energía. Por tanto, la energía de un
impacto se invierte en la propagación de grietas. La propiedad que
cuantifica esta resistencia es la tenacidad a la fractura, $K_{\text{Ic}}$.
Las cerámicas tienen valores de $K_{\text{Ic}}$ muy bajos en comparación con
los metales, lo que las hace susceptibles a la fractura catastrófica
ante un defecto o impacto.

\paragraph{Dureza Superficial.} Se requiere una alta dureza para resistir
la abrasión y el rayado por el uso de utensilios (e.g., cucharas) y
durante la limpieza. La dureza está directamente relacionada con la
fuerza de los enlaces interatómicos, por lo que las cerámicas son
ideales en este aspecto.

\subsubsection{Restricciones Químicas y de Biocompatibilidad}

\paragraph{Inercia Química.} El material debe ser químicamente inerte. No
debe reaccionar, corroerse ni degradarse en contacto con bebidas que
pueden ser ácidas (café, pH $\approx 5$) o alcalinas. La elevada
energía de los enlaces en óxidos cerámicos (e.g., $\ce{SiO2}$,
$\ce{Al2O3}$) les confiere una estabilidad termodinámica excepcional en
estos entornos.

\paragraph{Biocompatibilidad.} El material debe ser apto para el contacto
con alimentos. Esto significa que no debe haber lixiviación (leaching)
de especies iónicas tóxicas hacia la bebida. En el caso de cerámicas
esmaltadas, es crucial que el vidriado se haya procesado a una
temperatura adecuada para garantizar la completa vitrificación y evitar
la liberación de metales pesados (e.g., \ce{Pb}, \ce{Cd}) que pudieran estar
presentes en los pigmentos.
