\section{Análisis de la Estructura Cúbica Simple (SC)}

La topología de la estructura cúbica simple (SC) se define por la ubicación de
puntos de red exclusivamente en los vértices del cubo unitario.

\subsection{Ocupación de la Celda Unitaria}

La periodicidad traslacional de la red implica que cada punto nodal en un
vértice es compartido simultáneamente por ocho celdas unitarias contiguas. En
consecuencia, la multiplicidad efectiva de átomos, $n$, contenida en el
volumen fundamental se reduce a la unidad:

\begin{equation}
  n = (8) \frac{1}{8} = 1 \quad \text{átomo por celda.}
\end{equation}

Esta singularidad define a la celda SC como una celda primitiva propiamente
dicha.

\subsection{Geometría de Contacto y Parámetro de Red}

Asumiendo el modelo de esferas duras, la restricción geométrica de máxima
proximidad dicta que el contacto atómico ocurre a lo largo de las aristas del
cubo. Dado un radio atómico $R$, la relación con el parámetro de red $a$ es
inmediata y lineal:

\begin{equation}
  a = 2R.
\end{equation}

Esta relación establece que el espacio intersticial es máximo en el centro de
la celda cúbica.

\subsection{Número de Coordinación: Argumento de Simetría}

El número de coordinación ($Z$) se define como el número de vecinos más
cercanos equidistantes a un átomo de referencia. En lugar de una enumeración
espacial trivial, consideramos la simetría del grupo puntual cúbico.

Si situamos un átomo en el origen $(0,0,0)$, los vecinos más próximos se
encuentran a una distancia $d=a$. Debido a la ortogonalidad de la base, estos
vecinos corresponden a las traslaciones elementales a lo largo de los ejes
cartesianos principales positivo y negativo. El conjunto de vectores de
posición para los vecinos es:

\begin{equation}
  \vec{r}_{nn} = \{ \pm a e_1, \pm a e_2, \pm a e_3\}.
\end{equation}

La cardinalidad de este conjunto determina unívocamente el número de
coordinación:
\begin{equation}
  Z = 6.
\end{equation}

\subsection{Factor de Empaquetamiento Atómico (APF)}

La eficiencia del llenado espacial se cuantifica mediante el Factor de
Empaquetamiento Atómico (APF). Sustituyendo la relación geométrica $a(R)$ y la
ocupación $n$ en la definición de densidad volumétrica adimensional, se obtiene
una constante trascendente independiente del radio atómico:

\begin{equation}
  \text{APF} = \frac{n V_{\text{átomo}}}{V_{\text{celda}}}
  = \frac{\frac{4}{3}\pi R^3}{(2R)^3}
  = \frac{\pi}{6} \approx 0.5236.
\end{equation}

Este valor, sustancialmente bajo, indica que el $47.64\%$ del volumen de la
estructura es espacio vacío, lo cual explica la inestabilidad termodinámica
intrínseca de esta configuración frente a empaquetamientos más densos (BCC o
FCC) en enlaces metálicos no direccionales.

\section{Análisis de la Estructura Cúbica Centrada en el Cuerpo (BCC)}

La topología de la estructura cúbica centrada en el cuerpo (BCC) se caracteriza
por la inserción de un punto de red adicional en el baricentro de la celda
cúbica.

\subsection{Ocupación de la Celda Unitaria}

La base asociada a esta red de Bravais contiene un motivo de dos átomos. La
celda convencional encierra el átomo central en su totalidad (contribución
unitaria) y comparte los ocho átomos de los vértices con las celdas adyacentes
(contribución de $1/8$ cada uno). El número de átomos por celda, $n$, es
invariante:

\begin{equation}
  n = 1 + 8\left(\frac{1}{8}\right) = 2 \quad \text{átomos.}
\end{equation}

Este resultado duplica la densidad másica efectiva respecto a la estructura SC
para un mismo volumen y especie atómica.

\subsection{Geometría de Contacto y Parámetro de Red}

A diferencia de la estructura simple, los átomos en los vértices no se tocan
entre sí a lo largo de la arista $e_i$. La restricción geométrica de
esferas duras impone que la tangencia ocurra a lo largo de la diagonal espacial
del cubo (dirección $\langle 111 \rangle$).

La longitud de esta diagonal es la norma del vector $d = a e_1 +
ae_2 + ae_3$, es decir, $\|d\| = a\sqrt{3}$. Esta distancia
debe acomodar dos radios atómicos completos (del átomo central) y dos radios
parciales (de los vértices), totalizando $4R$. La relación fundamental es:

\begin{equation}
  a\sqrt{3} = 4R \implies a = \frac{4R}{\sqrt{3}}.
\end{equation}

\subsection{Número de Coordinación: Argumento de Simetría}

Para determinar el número de coordinación $Z$, situamos el origen en el átomo
central de la celda, cuya posición relativa es $\frac{1}{2}(e_1 +
e_2 + e_3)$. Los vecinos más próximos son los átomos ubicados en
los vértices del cubo.

El conjunto de vectores de desplazamiento $\delta$ desde el centro hacia
los vecinos más cercanos se construye mediante las permutaciones de signo de
las componentes semidiagonales:

\begin{equation}
  \delta_{nn} = \left\{ \frac{a}{2}(\sigma_1 e_1 + \sigma_2
  e_2 + \sigma_3 e_3) \mid \sigma_i \in \{1, -1\} \right\}.
\end{equation}

Dado que existen 3 grados de libertad binarios ($\sigma_1, \sigma_2, \sigma_3$),
la cardinalidad del conjunto de vecinos equidistantes es $2^3$. Por lo tanto:

\begin{equation}
  Z = 8.
\end{equation}

\subsection{Factor de Empaquetamiento Atómico (APF)}

Evaluando la eficiencia volumétrica mediante la sustitución de $n=2$ y la
relación $R(a)$, obtenemos una expresión compacta para el APF:

\begin{equation}
  \text{APF} = \frac{(2) \frac{4}{3}\pi R^3}{a^3}
  = \frac{8\pi R^3}{3 \left( \frac{4R}{\sqrt{3}} \right)^3}
  = \frac{8\pi R^3}{3} \frac{3\sqrt{3}}{64 R^3}.
\end{equation}

Simplificando los términos algebraicos:

\begin{equation}
  \text{APF} = \frac{\pi\sqrt{3}}{8} \approx 0.6802.
\end{equation}

Este incremento en la densidad de empaquetamiento (del $52\%$ al $68\%$)
justifica parcialmente la predominancia de fases BCC sobre SC en la naturaleza,
aunque aún no alcanza el límite geométrico de máximo empaquetamiento (0.74).

\section{Análisis de la Estructura Cúbica Centrada en las Caras (FCC)}

La estructura cúbica centrada en las caras (FCC) representa el límite
geométrico superior de densidad para el empaquetamiento de esferas idénticas
(junto con la estructura hexagonal compacta).
La red se define mediante la colocación de nodos en los vértices y en el centro
geométrico de cada una de las seis caras del cubo.

\subsection{Ocupación de la Celda Unitaria}

La evaluación del número de átomos $n$ requiere sumar las contribuciones de dos
conjuntos de sitios cristalográficos distintos. Los 8 átomos de vértice
contribuyen $1/8$ cada uno, mientras que los 6 átomos centrados en las caras,
compartidos únicamente por dos celdas adyacentes, contribuyen $1/2$ cada uno.
La suma resultante es:

\begin{equation}
  n = 8\left(\frac{1}{8}\right) + 6\left(\frac{1}{2}\right) = 4 \quad \text{átomos.}
\end{equation}

Esta ocupación de 4 átomos por celda es el doble de la estructura BCC, lo que
sugiere una alta eficiencia en la ocupación del espacio.

\subsection{Geometría de Contacto y Parámetro de Red}

El contacto entre esferas rígidas en la red FCC no ocurre a lo largo de la
arista del cubo ($a$), sino a lo largo de la diagonal de las caras. Esta
dirección corresponde a la familia de vectores $\langle 110 \rangle$.
Considerando una cara en el plano $e_{12}$, el vector diagonal es $d = a e_1
+ a e_2$.

La norma de este vector es $\|d\| = a\sqrt{2}$. Sobre esta longitud
se distribuyen: un radio del átomo en el origen, el diámetro completo del átomo
central de la cara, y un radio del átomo en el vértice opuesto. La condición de
tangencia establece:

\begin{equation}
  a\sqrt{2} = 4R \implies a = 2\sqrt{2}R.
\end{equation}

Esta relación confirma que los átomos están más densamente agrupados que en las
estructuras SC y BCC.

\subsection{Número de Coordinación: Argumento de Simetría}

El número de coordinación $Z$ se determina identificando el conjunto de vecinos
más cercanos al átomo situado en el origen. Debido a la simetría cúbica, estos
vecinos son los átomos centrados en las caras adyacentes al origen.

Las posiciones de estos átomos se generan mediante las permutaciones de los
vectores base tomados de a dos. El conjunto de desplazamientos $\delta$ hacia
los vecinos más cercanos es:

\begin{equation}
  \delta_{nn} = \left\{ \frac{a}{2}(\sigma_i e_i + \sigma_j e_j) \mid i \neq j,
  \, \sigma \in \{1, -1\} \right\}.
\end{equation}

Aquí, los índices $i, j$ se eligen del conjunto $\{1, 2, 3\}$. Existen
$\binom{3}{2} = 3$ pares de planos coordenados posibles. En cada plano, existen
$2^2 = 4$ combinaciones de signos para los cuadrantes. El número total de
elementos es, por tanto:

\begin{equation}
  Z = 3 \times 4 = 12.
\end{equation}

Este es el máximo número de coordinación posible para esferas de igual tamaño.

\subsection{Factor de Empaquetamiento Atómico (APF)}

La eficiencia máxima del empaquetamiento se calcula sustituyendo la ocupación
$n=4$ y la restricción geométrica $a(R)$ en la definición de densidad
volumétrica.

\begin{equation}
  \text{APF} = \frac{(4) \frac{4}{3}\pi R^3}{a^3}
  = \frac{16\pi R^3}{3 (2\sqrt{2}R)^3}.
\end{equation}

Desarrollando el denominador, $(2\sqrt{2})^3 = 16\sqrt{2}$. La expresión se
reduce a:

\begin{equation}
  \text{APF} = \frac{16\pi R^3}{(3) 16\sqrt{2} R^3}
  = \frac{\pi}{3\sqrt{2}} \approx 0.7405.
\end{equation}

Este valor de $\approx 0.74$ representa la densidad máxima teórica para un
empaquetamiento periódico de esferas, significativamente superior a la
estructura BCC ($0.68$).

\section{Análisis de la Estructura Hexagonal Compacta (HCP)}

La estructura Hexagonal Compacta (HCP) representa, junto con la FCC, la
solución al problema de empaquetamiento óptimo de esferas.
Formalmente, no constituye una red de Bravais por sí misma, sino una red
hexagonal simple asociada a una base de dos átomos.

\subsection{Métrica del Espacio y Ocupación}

La celda unitaria convencional es un prisma hexagonal definido por los vectores
base $\{e_1, e_2, e_3\}$. La métrica del espacio no es euclidiana estándar; los
vectores basales del plano $e_{12}$ forman un ángulo de $120^{\circ}$,
mientras que $e_3$ es ortogonal al plano. El tensor métrico $g_{ij} = e_i \cdot
e_j$ presenta componentes no diagonales:

\begin{equation}
  g_{ij} = \begin{pmatrix}
    a^2 & -a^2/2 & 0 \\
    -a^2/2 & a^2 & 0 \\
    0 & 0 & c^2
  \end{pmatrix}.
\end{equation}

La ocupación $n$ se deriva de la topología del prisma: 12 átomos en los
vértices (contribución de $1/6$ cada uno), 2 átomos en los centros de las caras
basales (contribución de $1/2$), y 3 átomos interiores completamente contenidos
en la celda.

\begin{equation}
  n = 12\left(\frac{1}{6}\right) + 2\left(\frac{1}{2}\right) + 3 = 6 \quad
  \text{átomos.}
\end{equation}

\subsection{Geometría de Contacto y Relación Axial}

La condición de esferas duras impone restricciones tanto en el plano basal como
en la dirección de apilamiento. En el plano definido por $e_{12}$, los
átomos son tangentes a lo largo de la arista, implicando $a = 2R$.

La restricción crítica ocurre entre capas alternas (secuencia ABAB). Tres átomos
del plano basal forman un triángulo equilátero de lado $a$. El átomo de la capa
siguiente descansa en el hueco formado por estos tres, creando un tetraedro
regular de lado $a$. La altura de este tetraedro corresponde a $c/2$. Por
argumentos trigonométricos sobre el tetraedro regular, la relación ideal entre
la altura y la base es:

\begin{equation}
  \frac{c}{a} = \sqrt{\frac{8}{3}} \approx 1.633.
\end{equation}

Desviaciones de este valor ideal indican una distorsión de la esfericidad del
potencial interatómico (anisotropía de enlace).

\subsection{Número de Coordinación: Topología de Apilamiento}

El análisis de coordinación aprovecha la simetría de traslación entre capas.
Consideremos un átomo en el plano B (capa intermedia).

\begin{enumerate}
  \item Intra-capa: Dentro de su propio plano hexagonal, el átomo está
    rodeado por 6 vecinos tangentes a distancia $a$.
  \item Inter-capa (Sup/Inf): Debido a la simetría de apilamiento, el
    átomo descansa sobre un ``nido'' de 3 átomos de la capa A inferior y
    soporta a 3 átomos de la capa A superior.
\end{enumerate}

La geometría del tetraedro regular garantiza que la distancia a los vecinos de
las capas adyacentes es idéntica a la distancia intra-capa ($a$), siempre que
$c/a = \sqrt{8/3}$. Por consiguiente, la suma directa es:

\begin{equation}
  Z = 6_{\text{plano}} + 3_{\text{sup}} + 3_{\text{inf}} = 12.
\end{equation}

Esto confirma que la densidad local es idéntica a la estructura FCC.

\subsection{Factor de Empaquetamiento Atómico (APF)}

El volumen de la celda hexagonal $V_c$ se calcula mediante el producto triple de
los vectores base: $V_c = |e_1 \wedge e_2 \wedge e_3|$. Dado el ángulo de
$120^{\circ}$ (o $\pi/3$ en radianes para el área del rombo base,
multiplicado por 3 para el hexágono):

\begin{equation}
  V_c = \text{ÁreaBase} \cdot c = \left( 6 \frac{\sqrt{3}}{4}a^2
  \right) c = \frac{3\sqrt{3}}{2} a^2 c.
\end{equation}

Sustituyendo la relación ideal $c = a\sqrt{8/3}$ y la ocupación $n=6$:

\begin{equation}
  \text{APF} = \frac{(6) \frac{4}{3}\pi R^3}{\frac{3\sqrt{3}}{2} a^2
  \left( a\sqrt{\frac{8}{3}} \right)}
  = \frac{8\pi R^3}{\frac{3\sqrt{3}}{2} (2R)^3 \sqrt{\frac{8}{3}}}.
\end{equation}

Tras la simplificación algebraica, el factor converge al mismo límite irracional
que la estructura FCC:

\begin{equation}
  \text{APF} = \frac{\pi}{\sqrt{18}} = \frac{\pi}{3\sqrt{2}} \approx 0.7405.
\end{equation}

Esto demuestra que la eficiencia de empaquetamiento es invariante ante la
secuencia de apilamiento (ABC vs AB), dependiendo únicamente de la
compacidad de las capas individuales.
