\documentclass[
  11pt,
  a4paper,
  numbers=noenddot
]{scrartcl}

\usepackage[utf8]{inputenc}
\usepackage[T1]{fontenc}
\usepackage{lmodern}
\usepackage[spanish]{babel}
\usepackage{csquotes}
\usepackage[sorting=nyt]{biblatex}

\addbibresource{./references.bib}

\usepackage[
  a4paper,
  left=2.5cm,
  right=2.5cm,
  top=3cm,
  bottom=3cm,
]{geometry}
\usepackage{microtype}
\usepackage{xcolor}

\usepackage{amsmath, amssymb, amsfonts}
\usepackage{amsthm}
\usepackage{siunitx}
\usepackage{derivative}
\usepackage{braket}
\usepackage[version=4]{mhchem}

\usepackage{graphicx}
\usepackage{booktabs}
\usepackage{multirow}
\usepackage{listings}

\usepackage{enumitem}
\usepackage[title]{appendix}

\usepackage{hyperref}
\usepackage[spanish]{cleveref}

\hypersetup{
  colorlinks=true,
  linkcolor=black,
  citecolor=blue,
  urlcolor=blue,
  pdftitle={Actividad 05: Estructura Atómica y Enlaces},
  pdfauthor={Julian L. Avila-Martinez},
  pdfsubject={Material Science},
  pdfkeywords={Ashby},
  bookmarksopen=true,
  bookmarksnumbered=true
}
\urlstyle{same}

\sisetup{
  group-digits=true,
  group-separator={\,},
  separate-uncertainty
}

\AtBeginDocument{\decimalpoint}

\theoremstyle{plain}
\newtheorem{theorem}{Teorema}[section]
\newtheorem{proposition}[theorem]{Proposición}

\theoremstyle{definition}
\newtheorem{definition}{Definición}[section]
\newtheorem{problem}{Problema}[section]
\newtheorem{example}{Ejemplo}[section]

\theoremstyle{remark}
\newtheorem{remark}{Observación}[section]

\renewcommand{\arraystretch}{1.2}
\raggedbottom

\begin{document}

\title{Actividad 05: Estructuras Cristalinas}
\author{Julian L. Avila-Martinez \\ \small\texttt{jlavilam@udistrital.edu.co}}
\date{Universidad Distrital Francisco José de Caldas \\ \today}
\maketitle

\section{Análisis de la Estructura Cúbica Simple (SC)}

La topología de la estructura cúbica simple (SC) se define por la ubicación de
puntos de red exclusivamente en los vértices del cubo unitario.

\subsection{Ocupación de la Celda Unitaria}

La periodicidad traslacional de la red implica que cada punto nodal en un
vértice es compartido simultáneamente por ocho celdas unitarias contiguas. En
consecuencia, la multiplicidad efectiva de átomos, $n$, contenida en el
volumen fundamental se reduce a la unidad:

\begin{equation}
  n = (8) \frac{1}{8} = 1 \quad \text{átomo por celda.}
\end{equation}

Esta singularidad define a la celda SC como una celda primitiva propiamente
dicha.

\subsection{Geometría de Contacto y Parámetro de Red}

Asumiendo el modelo de esferas duras, la restricción geométrica de máxima
proximidad dicta que el contacto atómico ocurre a lo largo de las aristas del
cubo. Dado un radio atómico $R$, la relación con el parámetro de red $a$ es
inmediata y lineal:

\begin{equation}
  a = 2R.
\end{equation}

Esta relación establece que el espacio intersticial es máximo en el centro de
la celda cúbica.

\subsection{Número de Coordinación: Argumento de Simetría}

El número de coordinación ($Z$) se define como el número de vecinos más
cercanos equidistantes a un átomo de referencia. En lugar de una enumeración
espacial trivial, consideramos la simetría del grupo puntual cúbico.

Si situamos un átomo en el origen $(0,0,0)$, los vecinos más próximos se
encuentran a una distancia $d=a$. Debido a la ortogonalidad de la base, estos
vecinos corresponden a las traslaciones elementales a lo largo de los ejes
cartesianos principales positivo y negativo. El conjunto de vectores de
posición para los vecinos es:

\begin{equation}
  \vec{r}_{nn} = \{ \pm a e_1, \pm a e_2, \pm a e_3\}.
\end{equation}

La cardinalidad de este conjunto determina unívocamente el número de
coordinación:
\begin{equation}
  Z = 6.
\end{equation}

\subsection{Factor de Empaquetamiento Atómico (APF)}

La eficiencia del llenado espacial se cuantifica mediante el Factor de
Empaquetamiento Atómico (APF). Sustituyendo la relación geométrica $a(R)$ y la
ocupación $n$ en la definición de densidad volumétrica adimensional, se obtiene
una constante trascendente independiente del radio atómico:

\begin{equation}
  \text{APF} = \frac{n V_{\text{átomo}}}{V_{\text{celda}}}
  = \frac{\frac{4}{3}\pi R^3}{(2R)^3}
  = \frac{\pi}{6} \approx 0.5236.
\end{equation}

Este valor, sustancialmente bajo, indica que el $47.64\%$ del volumen de la
estructura es espacio vacío, lo cual explica la inestabilidad termodinámica
intrínseca de esta configuración frente a empaquetamientos más densos (BCC o
FCC) en enlaces metálicos no direccionales.

\section{Análisis de la Estructura Cúbica Centrada en el Cuerpo (BCC)}

La topología de la estructura cúbica centrada en el cuerpo (BCC) se caracteriza
por la inserción de un punto de red adicional en el baricentro de la celda
cúbica.

\subsection{Ocupación de la Celda Unitaria}

La base asociada a esta red de Bravais contiene un motivo de dos átomos. La
celda convencional encierra el átomo central en su totalidad (contribución
unitaria) y comparte los ocho átomos de los vértices con las celdas adyacentes
(contribución de $1/8$ cada uno). El número de átomos por celda, $n$, es
invariante:

\begin{equation}
  n = 1 + 8\left(\frac{1}{8}\right) = 2 \quad \text{átomos.}
\end{equation}

Este resultado duplica la densidad másica efectiva respecto a la estructura SC
para un mismo volumen y especie atómica.

\subsection{Geometría de Contacto y Parámetro de Red}

A diferencia de la estructura simple, los átomos en los vértices no se tocan
entre sí a lo largo de la arista $e_i$. La restricción geométrica de
esferas duras impone que la tangencia ocurra a lo largo de la diagonal espacial
del cubo (dirección $\langle 111 \rangle$).

La longitud de esta diagonal es la norma del vector $d = a e_1 +
ae_2 + ae_3$, es decir, $\|d\| = a\sqrt{3}$. Esta distancia
debe acomodar dos radios atómicos completos (del átomo central) y dos radios
parciales (de los vértices), totalizando $4R$. La relación fundamental es:

\begin{equation}
  a\sqrt{3} = 4R \implies a = \frac{4R}{\sqrt{3}}.
\end{equation}

\subsection{Número de Coordinación: Argumento de Simetría}

Para determinar el número de coordinación $Z$, situamos el origen en el átomo
central de la celda, cuya posición relativa es $\frac{1}{2}(e_1 +
e_2 + e_3)$. Los vecinos más próximos son los átomos ubicados en
los vértices del cubo.

El conjunto de vectores de desplazamiento $\delta$ desde el centro hacia
los vecinos más cercanos se construye mediante las permutaciones de signo de
las componentes semidiagonales:

\begin{equation}
  \delta_{nn} = \left\{ \frac{a}{2}(\sigma_1 e_1 + \sigma_2
  e_2 + \sigma_3 e_3) \mid \sigma_i \in \{1, -1\} \right\}.
\end{equation}

Dado que existen 3 grados de libertad binarios ($\sigma_1, \sigma_2, \sigma_3$),
la cardinalidad del conjunto de vecinos equidistantes es $2^3$. Por lo tanto:

\begin{equation}
  Z = 8.
\end{equation}

\subsection{Factor de Empaquetamiento Atómico (APF)}

Evaluando la eficiencia volumétrica mediante la sustitución de $n=2$ y la
relación $R(a)$, obtenemos una expresión compacta para el APF:

\begin{equation}
  \text{APF} = \frac{(2) \frac{4}{3}\pi R^3}{a^3}
  = \frac{8\pi R^3}{3 \left( \frac{4R}{\sqrt{3}} \right)^3}
  = \frac{8\pi R^3}{3} \frac{3\sqrt{3}}{64 R^3}.
\end{equation}

Simplificando los términos algebraicos:

\begin{equation}
  \text{APF} = \frac{\pi\sqrt{3}}{8} \approx 0.6802.
\end{equation}

Este incremento en la densidad de empaquetamiento (del $52\%$ al $68\%$)
justifica parcialmente la predominancia de fases BCC sobre SC en la naturaleza,
aunque aún no alcanza el límite geométrico de máximo empaquetamiento (0.74).

\section{Análisis de la Estructura Cúbica Centrada en las Caras (FCC)}

La estructura cúbica centrada en las caras (FCC) representa el límite
geométrico superior de densidad para el empaquetamiento de esferas idénticas
(junto con la estructura hexagonal compacta).
La red se define mediante la colocación de nodos en los vértices y en el centro
geométrico de cada una de las seis caras del cubo.

\subsection{Ocupación de la Celda Unitaria}

La evaluación del número de átomos $n$ requiere sumar las contribuciones de dos
conjuntos de sitios cristalográficos distintos. Los 8 átomos de vértice
contribuyen $1/8$ cada uno, mientras que los 6 átomos centrados en las caras,
compartidos únicamente por dos celdas adyacentes, contribuyen $1/2$ cada uno.
La suma resultante es:

\begin{equation}
  n = 8\left(\frac{1}{8}\right) + 6\left(\frac{1}{2}\right) = 4 \quad \text{átomos.}
\end{equation}

Esta ocupación de 4 átomos por celda es el doble de la estructura BCC, lo que
sugiere una alta eficiencia en la ocupación del espacio.

\subsection{Geometría de Contacto y Parámetro de Red}

El contacto entre esferas rígidas en la red FCC no ocurre a lo largo de la
arista del cubo ($a$), sino a lo largo de la diagonal de las caras. Esta
dirección corresponde a la familia de vectores $\langle 110 \rangle$.
Considerando una cara en el plano $e_{12}$, el vector diagonal es $d = a e_1
+ a e_2$.

La norma de este vector es $\|d\| = a\sqrt{2}$. Sobre esta longitud
se distribuyen: un radio del átomo en el origen, el diámetro completo del átomo
central de la cara, y un radio del átomo en el vértice opuesto. La condición de
tangencia establece:

\begin{equation}
  a\sqrt{2} = 4R \implies a = 2\sqrt{2}R.
\end{equation}

Esta relación confirma que los átomos están más densamente agrupados que en las
estructuras SC y BCC.

\subsection{Número de Coordinación: Argumento de Simetría}

El número de coordinación $Z$ se determina identificando el conjunto de vecinos
más cercanos al átomo situado en el origen. Debido a la simetría cúbica, estos
vecinos son los átomos centrados en las caras adyacentes al origen.

Las posiciones de estos átomos se generan mediante las permutaciones de los
vectores base tomados de a dos. El conjunto de desplazamientos $\delta$ hacia
los vecinos más cercanos es:

\begin{equation}
  \delta_{nn} = \left\{ \frac{a}{2}(\sigma_i e_i + \sigma_j e_j) \mid i \neq j,
  \, \sigma \in \{1, -1\} \right\}.
\end{equation}

Aquí, los índices $i, j$ se eligen del conjunto $\{1, 2, 3\}$. Existen
$\binom{3}{2} = 3$ pares de planos coordenados posibles. En cada plano, existen
$2^2 = 4$ combinaciones de signos para los cuadrantes. El número total de
elementos es, por tanto:

\begin{equation}
  Z = 3 \times 4 = 12.
\end{equation}

Este es el máximo número de coordinación posible para esferas de igual tamaño.

\subsection{Factor de Empaquetamiento Atómico (APF)}

La eficiencia máxima del empaquetamiento se calcula sustituyendo la ocupación
$n=4$ y la restricción geométrica $a(R)$ en la definición de densidad
volumétrica.

\begin{equation}
  \text{APF} = \frac{(4) \frac{4}{3}\pi R^3}{a^3}
  = \frac{16\pi R^3}{3 (2\sqrt{2}R)^3}.
\end{equation}

Desarrollando el denominador, $(2\sqrt{2})^3 = 16\sqrt{2}$. La expresión se
reduce a:

\begin{equation}
  \text{APF} = \frac{16\pi R^3}{(3) 16\sqrt{2} R^3}
  = \frac{\pi}{3\sqrt{2}} \approx 0.7405.
\end{equation}

Este valor de $\approx 0.74$ representa la densidad máxima teórica para un
empaquetamiento periódico de esferas, significativamente superior a la
estructura BCC ($0.68$).

\section{Análisis de la Estructura Hexagonal Compacta (HCP)}

La estructura Hexagonal Compacta (HCP) representa, junto con la FCC, la
solución al problema de empaquetamiento óptimo de esferas.
Formalmente, no constituye una red de Bravais por sí misma, sino una red
hexagonal simple asociada a una base de dos átomos.

\subsection{Métrica del Espacio y Ocupación}

La celda unitaria convencional es un prisma hexagonal definido por los vectores
base $\{e_1, e_2, e_3\}$. La métrica del espacio no es euclidiana estándar; los
vectores basales del plano $e_{12}$ forman un ángulo de $120^{\circ}$,
mientras que $e_3$ es ortogonal al plano. El tensor métrico $g_{ij} = e_i \cdot
e_j$ presenta componentes no diagonales:

\begin{equation}
  g_{ij} = \begin{pmatrix}
    a^2 & -a^2/2 & 0 \\
    -a^2/2 & a^2 & 0 \\
    0 & 0 & c^2
  \end{pmatrix}.
\end{equation}

La ocupación $n$ se deriva de la topología del prisma: 12 átomos en los
vértices (contribución de $1/6$ cada uno), 2 átomos en los centros de las caras
basales (contribución de $1/2$), y 3 átomos interiores completamente contenidos
en la celda.

\begin{equation}
  n = 12\left(\frac{1}{6}\right) + 2\left(\frac{1}{2}\right) + 3 = 6 \quad
  \text{átomos.}
\end{equation}

\subsection{Geometría de Contacto y Relación Axial}

La condición de esferas duras impone restricciones tanto en el plano basal como
en la dirección de apilamiento. En el plano definido por $e_{12}$, los
átomos son tangentes a lo largo de la arista, implicando $a = 2R$.

La restricción crítica ocurre entre capas alternas (secuencia ABAB). Tres átomos
del plano basal forman un triángulo equilátero de lado $a$. El átomo de la capa
siguiente descansa en el hueco formado por estos tres, creando un tetraedro
regular de lado $a$. La altura de este tetraedro corresponde a $c/2$. Por
argumentos trigonométricos sobre el tetraedro regular, la relación ideal entre
la altura y la base es:

\begin{equation}
  \frac{c}{a} = \sqrt{\frac{8}{3}} \approx 1.633.
\end{equation}

Desviaciones de este valor ideal indican una distorsión de la esfericidad del
potencial interatómico (anisotropía de enlace).

\subsection{Número de Coordinación: Topología de Apilamiento}

El análisis de coordinación aprovecha la simetría de traslación entre capas.
Consideremos un átomo en el plano B (capa intermedia).

\begin{enumerate}
  \item Intra-capa: Dentro de su propio plano hexagonal, el átomo está
    rodeado por 6 vecinos tangentes a distancia $a$.
  \item Inter-capa (Sup/Inf): Debido a la simetría de apilamiento, el
    átomo descansa sobre un ``nido'' de 3 átomos de la capa A inferior y
    soporta a 3 átomos de la capa A superior.
\end{enumerate}

La geometría del tetraedro regular garantiza que la distancia a los vecinos de
las capas adyacentes es idéntica a la distancia intra-capa ($a$), siempre que
$c/a = \sqrt{8/3}$. Por consiguiente, la suma directa es:

\begin{equation}
  Z = 6_{\text{plano}} + 3_{\text{sup}} + 3_{\text{inf}} = 12.
\end{equation}

Esto confirma que la densidad local es idéntica a la estructura FCC.

\subsection{Factor de Empaquetamiento Atómico (APF)}

El volumen de la celda hexagonal $V_c$ se calcula mediante el producto triple de
los vectores base: $V_c = |e_1 \wedge e_2 \wedge e_3|$. Dado el ángulo de
$120^{\circ}$ (o $\pi/3$ en radianes para el área del rombo base,
multiplicado por 3 para el hexágono):

\begin{equation}
  V_c = \text{ÁreaBase} \cdot c = \left( 6 \frac{\sqrt{3}}{4}a^2
  \right) c = \frac{3\sqrt{3}}{2} a^2 c.
\end{equation}

Sustituyendo la relación ideal $c = a\sqrt{8/3}$ y la ocupación $n=6$:

\begin{equation}
  \text{APF} = \frac{(6) \frac{4}{3}\pi R^3}{\frac{3\sqrt{3}}{2} a^2
  \left( a\sqrt{\frac{8}{3}} \right)}
  = \frac{8\pi R^3}{\frac{3\sqrt{3}}{2} (2R)^3 \sqrt{\frac{8}{3}}}.
\end{equation}

Tras la simplificación algebraica, el factor converge al mismo límite irracional
que la estructura FCC:

\begin{equation}
  \text{APF} = \frac{\pi}{\sqrt{18}} = \frac{\pi}{3\sqrt{2}} \approx 0.7405.
\end{equation}

Esto demuestra que la eficiencia de empaquetamiento es invariante ante la
secuencia de apilamiento (ABC vs AB), dependiendo únicamente de la
compacidad de las capas individuales.


\section{Determinación de Índices Cristalográficos Direccionales}

La definición formal de una dirección cristalográfica $[uvw]$ se construye a
partir del vector de desplazamiento relativo $\Delta r$ entre dos puntos
nodales de la red, $P_{\text{inicial}}$ y $P_{\text{final}}$. El procedimiento
algorítmico requiere la reducción de las componentes de este vector a su
conjunto de enteros coprimos más pequeño, preservando la colinealidad.

Sean las coordenadas generalizadas en la base de la red $\{e_i\}$. El vector
se define como:
\begin{equation}
  r = (x_f - x_i)e_1 + (y_f - y_i)e_2 + (z_f - z_i)e_3.
\end{equation}

Procedemos al análisis de los tres casos propuestos:

\subsection{Caso A: Eje Principal}
\textbf{Datos:} $P_i = (0,0,0)$, $P_f = (1,0,0)$.

El vector de desplazamiento es inmediato:
\begin{equation}
  r_A = (1-0)e_1 + (0-0)e_2 + (0-0)e_3 = e_1.
\end{equation}
Las componentes ya son enteras y mínimas. La dirección coincide con el eje
fundamental de la celda.
\begin{equation}
  \text{Índices:} \quad [100]
\end{equation}

\subsection{Caso B: Diagonal Principal (Cúbica)}
\textbf{Datos:} $P_i = (0,0,0)$, $P_f = (1,1,1)$.

El vector resultante atraviesa el cuerpo de la celda desde el origen:
\begin{equation}
  r_B = (1-0)e_1 + (1-0)e_2 + (1-0)e_3 = e_1 + e_2 + e_3.
\end{equation}
Al ser las tres componentes la unidad, no requieren normalización.
\begin{equation}
  \text{Índices:} \quad [111]
\end{equation}

\subsection{Caso C: Vector Generalizado}
\textbf{Datos:} $P_i = (1/2, 1, 0)$, $P_f = (0,0,1)$.

Calculamos las componentes fraccionarias del desplazamiento:
\begin{equation}
  r_C = (0 - 1/2)e_1 + (0 - 1)e_2 + (1 - 0)e_3 = -\frac{1}{2}e_1 - e_2 + e_3.
\end{equation}
Para satisfacer la definición de índices de Miller (números enteros), aplicamos
un factor de escala escalar $\lambda = 2$ para eliminar el denominador, una
operación que es invariante respecto a la dirección del vector:
\begin{equation}
  2 \cdot r_C = -e_1 - 2e_2 + 2e_3.
\end{equation}
Adoptando la convención cristalográfica donde los componentes negativos se
denotan con una barra superior ($\bar{u}$):
\begin{equation}
  \text{Índices:} \quad [\bar{1}\bar{2}2]
\end{equation}

\section{Determinación de Índices de Miller para Planos Cristalográficos}

La notación de Miller $(hkl)$ especifica una familia de planos mediante un
vector normal en el espacio recíproco. El procedimiento estándar invierte los
interceptos axiales y normaliza a enteros coprimos. Formalmente, si los
interceptos son $x_1 e_1, x_2 e_2, x_3 e_3$, los índices se derivan de la
relación de proporcionalidad $h:k:l = x_1^{-1} : x_2^{-1} : x_3^{-1}$.

\subsection{Plano A: El Plano Octaédrico}
\textbf{Interceptos:} $(1, 0, 0), (0, 1, 0), (0, 0, 1)$.

Los interceptos normalizados respecto a los parámetros de red son $\mathbf{x} =
(1, 1, 1)$. Calculamos los recíprocos:
\begin{equation}
  \left( 1^{-1}, 1^{-1}, 1^{-1} \right) = (1, 1, 1).
\end{equation}
Al ser enteros unitarios, no requieren reducción.
\begin{equation}
  (hkl) = (111).
\end{equation}
Este plano representa la máxima densidad planar en sistemas FCC.

\subsection{Plano B: Plano Prismático General}
\textbf{Interceptos:} $(1, 0, 0), (0, 2, 0)$.

La ausencia de un tercer intercepto explícito implica paralelismo con el eje
$e_3$. Matemáticamente, el intercepto tiende al infinito asintótico:
$\mathbf{x} = (1, 2, \infty)$.
Tomando los inversos:
\begin{equation}
  \left( \frac{1}{1}, \frac{1}{2}, \lim_{\xi \to \infty}\frac{1}{\xi} \right)
  = \left( 1, 0.5, 0 \right).
\end{equation}
Para recuperar la naturaleza diofántica de los índices, aplicamos el factor de
escala mínimo $\lambda = 2$:
\begin{equation}
  (hkl) = 2 (1, 0.5, 0) = (210).
\end{equation}

\subsection{Plano C: Formulación Algebraica (Bivector)}
\textbf{Definición:} El plano subyacente al bivector $B = e_3 \wedge e_1$.

En lugar de buscar interceptos geométricos, utilizamos el álgebra de Clifford
$\mathcal{C}\ell_{3}$. El bivector $B$ codifica intrínsecamente la orientación del
subespacio plano. El vector normal $\mathbf{n}$ (asociado a los índices de
Miller) es el dual de Hodge (o complemento ortogonal) del bivector en el
espacio tridimensional euclidiano.

Utilizando el pseudoescalar $I = e_1 \wedge e_2 \wedge e_3$, la relación de
dualidad establece:
\begin{equation}
  \mathbf{n} \propto -I B = -(e_1 \wedge e_2 \wedge e_3)(e_3 \wedge e_1).
\end{equation}
Dada la anticonmutatividad del producto exterior y la normalización $e_i^2=1$:
\begin{equation}
  \mathbf{n} \propto e_2.
\end{equation}
El vector normal $e_2$ corresponde a los índices direccionales $[010]$. Por
consiguiente, la familia de planos ortogonales a este vector se denota:
\begin{equation}
  (hkl) = (010).
\end{equation}


\section{Densidades Cristalográficas y Empaquetamiento}

En el análisis de estructuras cristalinas, la eficiencia del
empaquetamiento atómico y las propiedades direccionales se
cuantifican mediante tres métricas fundamentales de densidad,
derivadas directamente de la geometría de la celda unitaria.

\subsection{Densidad Teórica (Volumétrica)}

La densidad teórica ($\rho$) se define como la masa total de los
átomos contenidos en una celda unitaria, dividida por el volumen
de dicha celda ($V_C$). Representa la densidad ideal del material
en un estado cristalino perfecto, conectando la microestructura
atómica con la densidad macroscópica observable.

La fórmula general para la densidad teórica es:
$$ \rho = \frac{n A}{V_C N_A} $$
Donde $n$ es el número de átomos (o unidades fórmula) por celda
unitaria; $A$ es la masa atómica [\si{\gram\per\mol}]; $V_C$ es
el volumen de la celda unitaria [\si{\cubic\centi\metre}]; y $N_A$
es el Número de Avogadro ($\approx \num{6.022e23}$ \si{átomos\per\mol}).

\subsubsection{Ejemplo: Hierro \texorpdfstring{$\alpha$-\ce{Fe}}{α-FE} (Estructura BCC)}
Para la estructura cúbica centrada en el cuerpo (BCC) del
Hierro-$\alpha$, el número de átomos por celda $n$ es 2 (un átomo
central más la contribución de $1/8$ por cada uno de los 8 vértices).
La masa atómica $A$ es $\approx \num{55.845}$ \si{\gram\per\mol}.
En la estructura BCC, el contacto atómico ocurre a lo largo de la
diagonal del cubo, $a\sqrt{3}$, la cual equivale a $4R$, donde $R$
es el radio atómico. De esta relación, se despeja el parámetro
de red $a = 4R/\sqrt{3}$. El volumen $V_C = a^3$ resulta entonces
en $\left(4R/\sqrt{3}\right)^3 = 64R^3/(3\sqrt{3})$.
Sustituyendo $n$ y $V_C$ en la ecuación de densidad:
$$ \rho_{\text{BCC}} = \frac{2 A}{\left(\frac{64R^3}{3\sqrt{3}}\right) N_A}
= \frac{6\sqrt{3} A}{64 R^3 N_A} = \frac{3\sqrt{3} A}{32 R^3 N_A} $$

\subsection{Densidad Lineal (DL)}

La densidad lineal (DL) se define como el número de átomos cuyos
centros son interceptados por un vector de dirección
cristalográfica, dividido por la longitud de dicho vector.
Esta métrica es fundamental para el análisis de la deformación
plástica, ya que las dislocaciones tienden a moverse
preferencialmente a lo largo de direcciones de alto empaquetamiento.

$$ \text{DL} = \frac{\text{Número de átomos centrados en el vector}}
{\text{Longitud del vector de dirección}} $$

\subsubsection{Ejemplo: Estructura FCC - Dirección [110]}
Consideremos la dirección [110] en una estructura FCC. Este vector
conecta el origen $(0,0,0)$ con el vértice $(a,a,0)$, y su
longitud es $L = \sqrt{a^2 + a^2 + 0^2} = a\sqrt{2}$.
El número de átomos centrados en este vector es 2. Este conteo
proviene de la suma de contribuciones: $1/2$ átomo en el origen
$(0,0,0)$, un átomo completo en el centro de la cara
$(a/2, a/2, 0)$, y $1/2$ átomo en el vértice $(a,a,0)$.
La densidad lineal para la dirección [110] es, por tanto:
$$ \text{DL}_{[110]} = \frac{2}{a\sqrt{2}} $$
En la estructura FCC, el contacto atómico se da en la diagonal
de la cara, $a\sqrt{2} = 4R$. Al sustituir esta relación,
la densidad lineal se simplifica notablemente a:
$$ \text{DL}_{[110]} = \frac{2}{4R} = \frac{1}{2R} $$

\subsection{Densidad Planar (DP)}

La densidad planar (DP) se define como el número de átomos
centrados en un plano cristalográfico, por unidad de área de
dicho plano, restringido a los límites de la celda unitaria.
Los planos de alta densidad planar (planos compactos) constituyen
los sistemas de deslizamiento preferenciales para la deformación.

$$ \text{DP} = \frac{\text{Número de átomos centrados en el área planar}}
{\text{Área del plano dentro de la celda}} $$

\subsubsection{Ejemplo: Estructura FCC - Plano (111)}
Este es el plano de empaquetamiento compacto en la estructura FCC.
Intercepta los ejes en $(a,0,0)$, $(0,a,0)$ y $(0,0,a)$,
formando un triángulo equilátero dentro de la celda unitaria.
La longitud de cada lado del triángulo, $L$, corresponde a la
diagonal de una cara: $L = a\sqrt{2}$.
El área de esta sección planar ($A_p$) es la de un triángulo
equilátero de lado $L$:
$$ A_p = \frac{1}{2} L \left(L \sin\left(\frac{\tau}{6}\right) \right) $$
$$ A_p = \frac{1}{2} \left(a\sqrt{2}\right) \left(a\sqrt{2}
	\frac{\sqrt{3}}{2}\right)
= \frac{a^2 \sqrt{3}}{2} $$
El número de átomos $N_p$ centrados en esta área se determina
sumando las contribuciones de los átomos que intercepta.
El plano pasa por 3 átomos en los vértices del triángulo (cada
uno contribuyendo $1/6$ al área interna) y 3 átomos en los
centros de las caras adyacentes (cada uno contribuyendo $1/2$).
El número total de átomos es, por tanto:
$N_p = 3 (1/6) + 3 (1/2) = 2$ átomos.
La densidad planar para el plano (111) resulta ser:
$$ \text{DP}_{(111)} = \frac{N_p}{A_p} = \frac{2}{a^2 \sqrt{3} / 2}
= \frac{4}{a^2 \sqrt{3}} $$
Usando la relación de FCC, $a\sqrt{2} = 4R \implies a = 2\sqrt{2}R$:
$$ \text{DP}_{(111)} = \frac{4}{(2\sqrt{2}R)^2 \sqrt{3}}
= \frac{4}{8R^2 \sqrt{3}} = \frac{1}{2\sqrt{3}R^2} $$
Esta configuración representa el empaquetamiento 2D más denso.


\section{Polimorfismo y Alotropía}

El polimorfismo es la capacidad de un material sólido de existir
en más de una estructura cristalina, o red de Bravais,
manteniendo una composición química idéntica. Cada una de estas
estructuras, conocidas como polimorfos, corresponde a un mínimo
de energía libre de Gibbs diferente y es, por lo tanto,
termodinámicamente estable bajo un conjunto específico de
condiciones de presión ($P$) y temperatura ($T$).

La transición entre fases polimórficas es una transición de
fase de primer orden, generalmente implicando un cambio
discontinuo en el volumen, la simetría y la entalpía.

Cuando este fenómeno se observa en un elemento puro, como el
carbono o el hierro, se utiliza el término más específico de
alotropía. Las diferentes formas se denominan alótropos.
La alotropía es, por tanto, un subconjunto del polimorfismo.

\subsection{Ejemplo: Alotropía del Hierro (\texorpdfstring{\ce{Fe}}{Fe})}
Un ejemplo de importancia fundamental en la ciencia de materiales
es la alotropía del hierro puro. A presión atmosférica y por
debajo de \qty{912}{\celsius}, el hierro es estable en la fase
$\alpha$-\ce{Fe} (ferrita), la cual posee la estructura cúbica centrada
en el cuerpo (BCC) analizada en la sección anterior.

Al superar los \qty{912}{\celsius}, el hierro experimenta una
transición de fase alotrópica a la fase $\gamma$-\ce{Fe}, conocida
como austenita. La austenita posee una estructura cúbica centrada
en las caras (FCC). Esta transformación (BCC $\to$ FCC) es
esencial, ya que la estructura FCC, con un factor de
empaquetamiento atómico superior, permite una solubilidad de
carbono significativamente mayor. Este hecho es el principio
fundamental para la formación y el tratamiento térmico de
los aceros.

Si se continúa calentando, el hierro vuelve a una fase BCC
($\delta$-\ce{Fe}) a \qty{1394}{\celsius}, antes de alcanzar su punto
de fusión a \qty{1538}{\celsius}. Las propiedades mecánicas y
magnéticas de estas fases son drásticamente diferentes.


\section{Espaciado Interplanar y la Ley de Bragg}

El espaciado interplanar, denotado $d_{hkl}$, es un parámetro
geométrico fundamental de la red cristalina. Se define como la
distancia perpendicular que separa dos planos paralelos
adyacentes en una familia de planos, identificados por los
índices de Miller $(hkl)$.

El cálculo de este espaciado depende intrínsecamente de la
simetría del sistema cristalino y de sus parámetros de red.
Para los sistemas ortogonales (cúbico, tetragonal y ortorrómbico),
la relación general se expresa de forma más elegante en términos
de su inverso al cuadrado:
$$ \frac{1}{d_{hkl}^2} = \frac{h^2}{a^2} + \frac{k^2}{b^2} +
   \frac{l^2}{c^2} $$
De esta ecuación se derivan los casos de mayor simetría. Para un
sistema tetragonal ($a = b \neq c$), la fórmula se simplifica;
para un sistema cúbico ($a = b = c$), la relación colapsa a:
$$ d_{hkl} = \frac{a}{\sqrt{h^2 + k^2 + l^2}} $$

La importancia física primordial del espaciado $d_{hkl}$ radica
en su rol central en la difracción, particularmente en la
difracción de rayos X (XRD). El cristal en su totalidad actúa
como una red de difracción tridimensional para la radiación incidente.

La interferencia constructiva de la radiación, dispersada por
los planos $(hkl)$, solo ocurre cuando se satisface una condición
geométrica estricta. Esta condición es la \emph{Ley de Bragg}:
$$ n\lambda = 2d_{hkl} \sin(\theta) $$
Donde $n$ es un entero que representa el orden de la difracción
(a menudo se considera $n=1$), $\lambda$ es la longitud de onda
de los rayos X monocromáticos, y $\theta$ es el ángulo de Bragg,
el ángulo de incidencia entre el haz y el plano cristalino.

Esta ecuación es la piedra angular de la cristalografía
experimental. Establece una relación unívoca entre el espaciado
microscópico $d_{hkl}$, una propiedad interna del material, y el
ángulo $\theta$, una cantidad macroscópica medible.

En la práctica, un difractómetro mide la intensidad de la
radiación en función del ángulo de barrido $2\theta$. Los picos
de alta intensidad corresponden a los ángulos $\theta$ que
satisfacen la Ley de Bragg. Al medir las posiciones de estos
picos, se determina el conjunto de espaciados $\{d_{hkl}\}$
presentes en la muestra, lo cual permite la identificación
inequívoca de la estructura de la red de Bravais y el cálculo
preciso de sus parámetros de red.


\section{Convenciones de Notación en Índices de Miller}

La notación de los índices de Miller, $(hkl)$, está sujeta a
convenciones estrictas que son cruciales para su correcta
interpretación.

\subsection{Planos Equivalentes por Inversión}
Un plano cristalográfico se define por sus interceptos;
los índices $(hkl)$ representan una familia de planos paralelos
infinitos. Por convención, un conjunto de índices y su negativo
directo, $(-h, -k, -l)$, se consideran físicamente equivalentes.

Consideremos los índices $(012)$ y $(0\bar{1}\bar{2})$, donde la
barra superior denota un valor negativo (e.g., $-1$).
El plano $(012)$ intercepta los ejes en $(\infty, b, c/2)$.
El plano $(0\bar{1}\bar{2})$ intercepta los ejes en
$(\infty, -b, -c/2)$.

Estos dos planos son paralelos y están separados por la misma
distancia $d_{hkl}$. Geométricamente, describen el mismo conjunto
de planos en el cristal, aunque sus vectores normales apunten en
direcciones opuestas. En el contexto de la difracción, dado que
la Ley de Bragg depende de $d_{hkl}$, ambos planos son
indistinguibles y se consideran idénticos. Pertenecen a la misma
familia de planos $\{012\}$.

\subsection{Múltiplos de Índices y Órdenes de Difracción}
Una situación conceptualmente diferente surge con los índices que
son múltiplos enteros, como $(123)$ y $(246)$.

Por definición geométrica, los índices de Miller $(hkl)$ deben
ser el conjunto de enteros más pequeños que mantienen la misma
proporción (es decir, deben ser coprimos). El plano $(123)$
intercepta los ejes en $(a, b/2, c/3)$.
La notación $(246)$ es, por esta definición, inválida, ya que
puede reducirse a $(123)$ al dividir por un factor común de 2.
Representaría un plano con interceptos $(a/2, b/4, c/6)$, que
es un plano paralelo a $(123)$ pero con un espaciado diferente.

Sin embargo, la notación $(nh, nk, nl)$, como $(246)$,
adquiere un significado físico fundamental en el contexto de la
difracción de rayos X. Se utiliza universalmente como una
taquigrafía para referirse a los órdenes de difracción
superiores ($n > 1$) de la Ley de Bragg ($n\lambda = 2d_{hkl} \sin\theta$).

El pico $(123)$ corresponde al primer orden ($n=1$) de
difracción de los planos $(123)$. El pico $(246)$ se refiere
específicamente al segundo orden ($n=2$) de difracción de
esos mismos planos $(123)$.

Esta notación es matemáticamente consistente. El espaciado
para un plano ficticio $(246)$ sería $d_{246} = d_{123} / 2$.
Al sustituir esto en la Ley de Bragg para $n=1$, se obtiene:
$$ 1 \lambda = 2 d_{246} \sin(\theta_{(246)})
   = 2 \left(\frac{d_{123}}{2}\right) \sin(\theta_{(246)}) $$
$$ \lambda = d_{123} \sin(\theta_{(246)}) $$
Si comparamos esto con la condición de Bragg para el segundo
orden de $(123)$:
$$ 2 \lambda = 2 d_{123} \sin(\theta_{(123), n=2}) $$
$$ \lambda = d_{123} \sin(\theta_{(123), n=2}) $$
Vemos que $\theta_{(246)} = \theta_{(123), n=2}$. Por lo tanto,
$(123)$ y $(246)$ no son iguales: representan el primer y
segundo orden de difracción de la misma familia de planos,
y aparecen como picos distintos en un difractograma.


\section{Relevancia de la Anisotropía Cristalográfica}

El formalismo de las direcciones y planos cristalográficos es
esencial, ya que la estructura periódica de un cristal implica
que sus propiedades, en general, no son isotrópicas. Los índices
de Miller y las direcciones proveen el lenguaje matemático
preciso para cuantificar y predecir esta anisotropía.

\subsection{Influencia en las Propiedades del Material}

La mayoría de los monocristales exhiben una marcada
anisotropía, donde la magnitud de una propiedad física
depende de la dirección $[uvw]$ en la que se mide. Los planos
$(hkl)$ y las direcciones $[uvw]$ son el lenguaje para describir
este fenómeno.

Un ejemplo canónico es la deformación plástica en metales.
El deslizamiento (slip) no ocurre en planos arbitrarios, sino
que se activa preferencialmente en los planos de mayor densidad
planar (DP) y en las direcciones de mayor densidad lineal (DL),
ya que esto requiere la menor energía (menor esfuerzo de cizalla).
Este conjunto, un plano y una dirección específicos, se
denomina \textit{sistema de deslizamiento}. Por ejemplo, en
la estructura FCC, el deslizamiento ocurre en los planos
$\{111\}$ y en las direcciones $\langle 1\bar{1}0 \rangle$.

Similarmente, propiedades como la conductividad eléctrica,
el módulo de Young, y la susceptibilidad magnética (eje de
fácil magnetización) están intrínsecamente ligadas a la
orientación cristalográfica.

\subsection{Rol en la Difracción de Rayos X}

Como se estableció en la sección anterior, los índices de Miller
son el pilar de la interpretación de la difracción. La Ley de
Bragg ($n\lambda = 2d_{hkl}\sin\theta$) relaciona el ángulo de
difracción $\theta$ con el espaciado $d_{hkl}$. Los índices
$(hkl)$ son, precisamente, la etiqueta que identifica de forma
unívoca cada espaciado $d$.

Este formalismo permite ``indexar'' el difractograma: cada pico
de intensidad medido en un ángulo $2\theta$ experimental se
asigna a un plano $(hkl)$ específico. Sin los índices de Miller,
un patrón de difracción sería solo una serie de picos anónimos;
con ellos, se convierte en un mapa directo que revela la red de
Bravais y los parámetros de red del material.

\subsection{Determinación del Crecimiento del Cristal}

La morfología externa de un cristal cultivado (sus facetas) no es
aleatoria, sino que está dictada por la cristalografía. Las
caras externas de un cristal ideal tienden a ser los planos
$(hkl)$ que poseen la energía superficial más baja.

Esto se debe a que los planos de baja energía, que usualmente
corresponden a planos de alto empaquetamiento (alta DP), son
termodinámicamente más estables. Durante el proceso de
cristalización (desde un fundido, vapor o solución), los planos
de alta energía crecen rápidamente y ``desaparecen'', mientras que
los planos estables de bajo índice (como $\{100\}$, $\{110\}$ o
$\{111\}$) crecen más lentamente y definen la forma facetada
final del material.


\printbibliography
\nocite{*}

\end{document}
