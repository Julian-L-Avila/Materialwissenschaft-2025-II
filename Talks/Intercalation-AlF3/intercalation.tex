\documentclass[12pt, a4paper]{article}

\usepackage[utf8]{inputenc}
\usepackage[T1]{fontenc}
\usepackage[spanish]{babel}
\usepackage{amsmath}
\usepackage{amssymb}
\usepackage{graphicx}
\usepackage[margin=1in]{geometry}
\usepackage{hyperref}
\usepackage[version=4]{mhchem}

\hypersetup{
  colorlinks=true,
  linkcolor=black,
  citecolor=black,
  filecolor=black,
  urlcolor=blue,
}

\title{
  Un Reporte Resumen: \\ Intercalación de \ce{AlF3} en Grafito:
  Mecanismos de Sorción Estudiados por AES y DFT
}
\author{Julian L. Avila-Martinez}
\date{\today}

\begin{document}

\maketitle
\thispagestyle{empty}
\vspace{0.5cm}

\begin{abstract}
  \noindent Este reporte resume una investigación combinada, experimental y
  computacional, sobre la intercalación de fluoruro de aluminio (\ce{AlF3}) en
  un sustrato de grafito. Las técnicas principales empleadas fueron la
  \textbf{Espectroscopía de Electrones Auger (AES)} para el análisis de
  superficie experimental y la \textbf{Teoría del Funcional de la Densidad
  (DFT)} para el modelado computacional. Un hallazgo principal fue la excelente
  concordancia entre los datos experimentales y las predicciones teóricas, las
  cuales indican que la intercalación de \ce{AlF3} en el grafito es un proceso
  energéticamente favorable.
\end{abstract}


\section*{Justificación y Motivación}

La investigación está motivada por la necesidad apremiante de encontrar
alternativas a las baterías de ion-litio. Aunque el litio es el estándar de la
tecnología actual de baterías recargables, su limitada abundancia
terrestre presenta un desafío a largo plazo. El aluminio es una alternativa muy
atractiva debido a su gran abundancia. Además, su catión trivalente,
\ce{Al^{3+}}, ofrece una alta capacidad de carga, y el fluoruro de aluminio,
\ce{AlF3}, es un compuesto químicamente estable adecuado para aplicaciones
electroquímicas. Este estudio, por lo tanto, explora la viabilidad fundamental
de un sistema aluminio-grafito, un primer paso crucial hacia el desarrollo de
nuevas baterías recargables de ion-aluminio.

\section*{Investigación Experimental}

El enfoque experimental se centró en el uso de la Espectroscopía de Electrones
Auger (AES), una técnica altamente sensible a la composición elemental de la
región cercana a la superficie de un material. Esta sensibilidad es primordial,
ya que las propiedades de la superficie (i.e., número de coordinación, energía
superficial, reactividad) difieren fundamentalmente de las del material en
volumen (bulk). Se hizo una distinción clave entre la \textbf{adsorción} (la
acumulación de átomos o moléculas en la superficie de un material) y la
\textbf{absorción} (la difusión de partículas hacia el interior del volumen).
La intercalación es un tipo específico de absorción donde moléculas o iones
huéspedes se insertan en la estructura laminar de un material anfitrión.

Se utilizó AES para verificar que estaba ocurriendo una intercalación, en lugar
de una mera adsorción superficial. El efecto Auger es un proceso mecano-cuántico
de tres pasos:
\begin{enumerate}
  \item \textbf{Ionización}: Una partícula de alta energía (e.g. un electrón
    del cañón de electrones) colisiona con un electrón de un nivel interno
    (e.g., en la capa K) de un átomo del sustrato, expulsándolo y creando un
    hueco en dicho nivel.
  \item \textbf{Relajación}: Un electrón de un nivel de energía superior (e.g.
    la capa L\(_1\)) decae para llenar el hueco del nivel internob
  \item \textbf{Emisión Auger}: La energía liberada durante el proceso de
    relajación se transfiere de forma no radiativa a otro electrón (e.g. en
    la capa L\(_2\)), que es entonces emitido del átomo. Esta partícula emitida
    es el \textbf{electrón Auger}.
\end{enumerate}
La energía cinética del electrón Auger emitido es característica de la
estructura electrónica del átomo de origen, lo que permite una identificación
elemental precisa.

Los experimentos se llevaron a cabo utilizando un espectrómetro PHI SAM 590A. Se
sublimaron cristalitos de \ce{AlF3} y se depositaron molécula a molécula sobre un
sustrato de Grafito Piroloítico Altamente Orientado (HOPG). Se utilizó una placa
de cobre como referencia para la calibración del flujo molecular, ya que su
estructura cúbica centrada en las caras (FCC) no permite la intercalación. Para
confirmar la intercalación, se aplicó la \textbf{ley de Beer-Lambert} a los
datos de AES. A medida que el \ce{AlF3} se intercala con éxito, forma capas debajo
de la superficie del grafito, causando una atenuación exponencial de la señal
Auger del carbono que proviene del sustrato. Las observaciones experimentales
revelaron una dependencia crítica de la tasa de deposición: un \textbf{flujo
molecular bajo} resultó en una intercalación exitosa, lo cual se atribuye a que
se concede tiempo suficiente para que las moléculas de \ce{AlF3} se difundan por
la superficie y encuentren sitios energéticamente favorables para penetrar en la
red del grafito. Por el contrario, un \textbf{flujo molecular alto} condujo
predominantemente a la adsorción, donde las moléculas se agregaron en la
superficie del grafito en lugar de entrar en la estructura interna.

\section*{Modelado Computacional}

Los resultados experimentales fueron corroborados por simulaciones de primeros
principios basadas en DFT. Para modelar la dinámica del proceso de
intercalación, se implementó el método de la \textbf{Banda Elástica Empujada
(NEB, por sus siglas en inglés)}. El método NEB es un potente algoritmo
utilizado para determinar el \textbf{camino de mínima energía (MEP)} para una
transición entre un estado inicial y uno final; en este caso, una molécula de
\ce{AlF3} en la superficie del grafito y la misma molécula intercalada entre los
planos de grafeno. Las simulaciones identificaron con éxito una ruta viable y de
baja energía para la intercalación. De manera crucial, los cálculos de la
energética de la reacción revelaron que el proceso global es
\textbf{exotérmico}. Este resultado teórico proporciona una fuerte evidencia de
que la intercalación de \ce{AlF3} en el grafito es un proceso termodinámicamente
espontáneo y favorable, respaldando plenamente las conclusiones experimentales.

\end{document}
