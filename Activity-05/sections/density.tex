\section{Densidades Cristalográficas y Empaquetamiento}

En el análisis de estructuras cristalinas, la eficiencia del
empaquetamiento atómico y las propiedades direccionales se
cuantifican mediante tres métricas fundamentales de densidad,
derivadas directamente de la geometría de la celda unitaria.

\subsection{Densidad Teórica (Volumétrica)}

La densidad teórica ($\rho$) se define como la masa total de los
átomos contenidos en una celda unitaria, dividida por el volumen
de dicha celda ($V_C$). Representa la densidad ideal del material
en un estado cristalino perfecto, conectando la microestructura
atómica con la densidad macroscópica observable.

La fórmula general para la densidad teórica es:
$$ \rho = \frac{n A}{V_C N_A} $$
Donde $n$ es el número de átomos (o unidades fórmula) por celda
unitaria; $A$ es la masa atómica [\si{\gram\per\mol}]; $V_C$ es
el volumen de la celda unitaria [\si{\cubic\centi\metre}]; y $N_A$
es el Número de Avogadro ($\approx \num{6.022e23}$ \si{átomos\per\mol}).

\subsubsection{Ejemplo: Hierro \texorpdfstring{$\alpha$-\ce{Fe}}{α-FE} (Estructura BCC)}
Para la estructura cúbica centrada en el cuerpo (BCC) del
Hierro-$\alpha$, el número de átomos por celda $n$ es 2 (un átomo
central más la contribución de $1/8$ por cada uno de los 8 vértices).
La masa atómica $A$ es $\approx \num{55.845}$ \si{\gram\per\mol}.
En la estructura BCC, el contacto atómico ocurre a lo largo de la
diagonal del cubo, $a\sqrt{3}$, la cual equivale a $4R$, donde $R$
es el radio atómico. De esta relación, se despeja el parámetro
de red $a = 4R/\sqrt{3}$. El volumen $V_C = a^3$ resulta entonces
en $\left(4R/\sqrt{3}\right)^3 = 64R^3/(3\sqrt{3})$.
Sustituyendo $n$ y $V_C$ en la ecuación de densidad:
$$ \rho_{\text{BCC}} = \frac{2 A}{\left(\frac{64R^3}{3\sqrt{3}}\right) N_A}
= \frac{6\sqrt{3} A}{64 R^3 N_A} = \frac{3\sqrt{3} A}{32 R^3 N_A} $$

\subsection{Densidad Lineal (DL)}

La densidad lineal (DL) se define como el número de átomos cuyos
centros son interceptados por un vector de dirección
cristalográfica, dividido por la longitud de dicho vector.
Esta métrica es fundamental para el análisis de la deformación
plástica, ya que las dislocaciones tienden a moverse
preferencialmente a lo largo de direcciones de alto empaquetamiento.

$$ \text{DL} = \frac{\text{Número de átomos centrados en el vector}}
{\text{Longitud del vector de dirección}} $$

\subsubsection{Ejemplo: Estructura FCC - Dirección [110]}
Consideremos la dirección [110] en una estructura FCC. Este vector
conecta el origen $(0,0,0)$ con el vértice $(a,a,0)$, y su
longitud es $L = \sqrt{a^2 + a^2 + 0^2} = a\sqrt{2}$.
El número de átomos centrados en este vector es 2. Este conteo
proviene de la suma de contribuciones: $1/2$ átomo en el origen
$(0,0,0)$, un átomo completo en el centro de la cara
$(a/2, a/2, 0)$, y $1/2$ átomo en el vértice $(a,a,0)$.
La densidad lineal para la dirección [110] es, por tanto:
$$ \text{DL}_{[110]} = \frac{2}{a\sqrt{2}} $$
En la estructura FCC, el contacto atómico se da en la diagonal
de la cara, $a\sqrt{2} = 4R$. Al sustituir esta relación,
la densidad lineal se simplifica notablemente a:
$$ \text{DL}_{[110]} = \frac{2}{4R} = \frac{1}{2R} $$

\subsection{Densidad Planar (DP)}

La densidad planar (DP) se define como el número de átomos
centrados en un plano cristalográfico, por unidad de área de
dicho plano, restringido a los límites de la celda unitaria.
Los planos de alta densidad planar (planos compactos) constituyen
los sistemas de deslizamiento preferenciales para la deformación.

$$ \text{DP} = \frac{\text{Número de átomos centrados en el área planar}}
{\text{Área del plano dentro de la celda}} $$

\subsubsection{Ejemplo: Estructura FCC - Plano (111)}
Este es el plano de empaquetamiento compacto en la estructura FCC.
Intercepta los ejes en $(a,0,0)$, $(0,a,0)$ y $(0,0,a)$,
formando un triángulo equilátero dentro de la celda unitaria.
La longitud de cada lado del triángulo, $L$, corresponde a la
diagonal de una cara: $L = a\sqrt{2}$.
El área de esta sección planar ($A_p$) es la de un triángulo
equilátero de lado $L$:
$$ A_p = \frac{1}{2} L \left(L \sin\left(\frac{\tau}{6}\right) \right) $$
$$ A_p = \frac{1}{2} \left(a\sqrt{2}\right) \left(a\sqrt{2}
	\frac{\sqrt{3}}{2}\right)
= \frac{a^2 \sqrt{3}}{2} $$
El número de átomos $N_p$ centrados en esta área se determina
sumando las contribuciones de los átomos que intercepta.
El plano pasa por 3 átomos en los vértices del triángulo (cada
uno contribuyendo $1/6$ al área interna) y 3 átomos en los
centros de las caras adyacentes (cada uno contribuyendo $1/2$).
El número total de átomos es, por tanto:
$N_p = 3 (1/6) + 3 (1/2) = 2$ átomos.
La densidad planar para el plano (111) resulta ser:
$$ \text{DP}_{(111)} = \frac{N_p}{A_p} = \frac{2}{a^2 \sqrt{3} / 2}
= \frac{4}{a^2 \sqrt{3}} $$
Usando la relación de FCC, $a\sqrt{2} = 4R \implies a = 2\sqrt{2}R$:
$$ \text{DP}_{(111)} = \frac{4}{(2\sqrt{2}R)^2 \sqrt{3}}
= \frac{4}{8R^2 \sqrt{3}} = \frac{1}{2\sqrt{3}R^2} $$
Esta configuración representa el empaquetamiento 2D más denso.
