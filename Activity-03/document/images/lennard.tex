\documentclass[border=5mm]{standalone}
\usepackage{luamplib}
\begin{document}
\mplibtextextlabel{enable}
\begin{mplibcode}
beginfig(1);

  numeric u; u = 34;

  path xx, yy;
  yy = (2.7 down -- 4.2 up) scaled u;
  xx = origin -- 10 right scaled u;
  xx := xx shifted point 0 of yy;

  path p;
  p = ((.7, 4){dir -89} .. (1, 0) .. (2.5, -2) {right} .. (5, -1) ... (10, 0) {right}) scaled u;

  path a, b, c;
  a = (xpart point 1 of p, ypart point 0 of p) -- point 1 of p -- point 4 of p;
  b = (0, ypart point 1 of p) -- point 1 of p -- (xpart point 1 of p, ypart point 0 of xx);
  c = (0, ypart point 2 of p) -- point 2 of p -- (xpart point 2 of p, ypart point 0 of xx);

  draw a;
  draw b dashed evenly scaled 1/2;
  draw c dashed evenly scaled 1/2;

  draw xx;
  draw yy;

  draw p withpen pencircle scaled 3/4 withcolor (0.5,0.0,0.5);

  % Added this line for the equation
  label(btex $V(r) = 4\varepsilon \left[ \left(\frac{\sigma}{r}\right)^{12} - \left(\frac{\sigma}{r}\right)^{6} \right]$ etex, (6u, 2.5u));

  label.lft("$0$", point 0 of b);
  label.lft("$-\varepsilon$", point 0 of c);

  label.bot("$\strut\sigma$", point 2 of b);
  label.bot("$\strut r_m$", point 2 of c);

  label(TEX("\textbf{Energy}") rotated 90, point 1/2 of yy shifted 24 left);
  label("$\mathbf{r}$", point 1/2 of xx shifted 24 down);

endfig;
\end{mplibcode}
\end{document}
