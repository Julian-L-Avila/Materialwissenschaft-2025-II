\section{Espaciado Interplanar y la Ley de Bragg}

El espaciado interplanar, denotado $d_{hkl}$, es un parámetro
geométrico fundamental de la red cristalina. Se define como la
distancia perpendicular que separa dos planos paralelos
adyacentes en una familia de planos, identificados por los
índices de Miller $(hkl)$.

El cálculo de este espaciado depende intrínsecamente de la
simetría del sistema cristalino y de sus parámetros de red.
Para los sistemas ortogonales (cúbico, tetragonal y ortorrómbico),
la relación general se expresa de forma más elegante en términos
de su inverso al cuadrado:
$$ \frac{1}{d_{hkl}^2} = \frac{h^2}{a^2} + \frac{k^2}{b^2} +
   \frac{l^2}{c^2} $$
De esta ecuación se derivan los casos de mayor simetría. Para un
sistema tetragonal ($a = b \neq c$), la fórmula se simplifica;
para un sistema cúbico ($a = b = c$), la relación colapsa a:
$$ d_{hkl} = \frac{a}{\sqrt{h^2 + k^2 + l^2}} $$

La importancia física primordial del espaciado $d_{hkl}$ radica
en su rol central en la difracción, particularmente en la
difracción de rayos X (XRD). El cristal en su totalidad actúa
como una red de difracción tridimensional para la radiación incidente.

La interferencia constructiva de la radiación, dispersada por
los planos $(hkl)$, solo ocurre cuando se satisface una condición
geométrica estricta. Esta condición es la \emph{Ley de Bragg}:
$$ n\lambda = 2d_{hkl} \sin(\theta) $$
Donde $n$ es un entero que representa el orden de la difracción
(a menudo se considera $n=1$), $\lambda$ es la longitud de onda
de los rayos X monocromáticos, y $\theta$ es el ángulo de Bragg,
el ángulo de incidencia entre el haz y el plano cristalino.

Esta ecuación es la piedra angular de la cristalografía
experimental. Establece una relación unívoca entre el espaciado
microscópico $d_{hkl}$, una propiedad interna del material, y el
ángulo $\theta$, una cantidad macroscópica medible.

En la práctica, un difractómetro mide la intensidad de la
radiación en función del ángulo de barrido $2\theta$. Los picos
de alta intensidad corresponden a los ángulos $\theta$ que
satisfacen la Ley de Bragg. Al medir las posiciones de estos
picos, se determina el conjunto de espaciados $\{d_{hkl}\}$
presentes en la muestra, lo cual permite la identificación
inequívoca de la estructura de la red de Bravais y el cálculo
preciso de sus parámetros de red.
