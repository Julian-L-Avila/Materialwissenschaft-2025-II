\section{Polimorfismo y Alotropía}

El polimorfismo es la capacidad de un material sólido de existir
en más de una estructura cristalina, o red de Bravais,
manteniendo una composición química idéntica. Cada una de estas
estructuras, conocidas como polimorfos, corresponde a un mínimo
de energía libre de Gibbs diferente y es, por lo tanto,
termodinámicamente estable bajo un conjunto específico de
condiciones de presión ($P$) y temperatura ($T$).

La transición entre fases polimórficas es una transición de
fase de primer orden, generalmente implicando un cambio
discontinuo en el volumen, la simetría y la entalpía.

Cuando este fenómeno se observa en un elemento puro, como el
carbono o el hierro, se utiliza el término más específico de
alotropía. Las diferentes formas se denominan alótropos.
La alotropía es, por tanto, un subconjunto del polimorfismo.

\subsection{Ejemplo: Alotropía del Hierro (\texorpdfstring{\ce{Fe}}{Fe})}
Un ejemplo de importancia fundamental en la ciencia de materiales
es la alotropía del hierro puro. A presión atmosférica y por
debajo de \qty{912}{\celsius}, el hierro es estable en la fase
$\alpha$-\ce{Fe} (ferrita), la cual posee la estructura cúbica centrada
en el cuerpo (BCC) analizada en la sección anterior.

Al superar los \qty{912}{\celsius}, el hierro experimenta una
transición de fase alotrópica a la fase $\gamma$-\ce{Fe}, conocida
como austenita. La austenita posee una estructura cúbica centrada
en las caras (FCC). Esta transformación (BCC $\to$ FCC) es
esencial, ya que la estructura FCC, con un factor de
empaquetamiento atómico superior, permite una solubilidad de
carbono significativamente mayor. Este hecho es el principio
fundamental para la formación y el tratamiento térmico de
los aceros.

Si se continúa calentando, el hierro vuelve a una fase BCC
($\delta$-\ce{Fe}) a \qty{1394}{\celsius}, antes de alcanzar su punto
de fusión a \qty{1538}{\celsius}. Las propiedades mecánicas y
magnéticas de estas fases son drásticamente diferentes.
