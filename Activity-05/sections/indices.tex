\section{Determinación de Índices Cristalográficos Direccionales}

La definición formal de una dirección cristalográfica $[uvw]$ se construye a
partir del vector de desplazamiento relativo $\Delta r$ entre dos puntos
nodales de la red, $P_{\text{inicial}}$ y $P_{\text{final}}$. El procedimiento
algorítmico requiere la reducción de las componentes de este vector a su
conjunto de enteros coprimos más pequeño, preservando la colinealidad.

Sean las coordenadas generalizadas en la base de la red $\{e_i\}$. El vector
se define como:
\begin{equation}
  r = (x_f - x_i)e_1 + (y_f - y_i)e_2 + (z_f - z_i)e_3.
\end{equation}

Procedemos al análisis de los tres casos propuestos:

\subsection{Caso A: Eje Principal}
\textbf{Datos:} $P_i = (0,0,0)$, $P_f = (1,0,0)$.

El vector de desplazamiento es inmediato:
\begin{equation}
  r_A = (1-0)e_1 + (0-0)e_2 + (0-0)e_3 = e_1.
\end{equation}
Las componentes ya son enteras y mínimas. La dirección coincide con el eje
fundamental de la celda.
\begin{equation}
  \text{Índices:} \quad [100]
\end{equation}

\subsection{Caso B: Diagonal Principal (Cúbica)}
\textbf{Datos:} $P_i = (0,0,0)$, $P_f = (1,1,1)$.

El vector resultante atraviesa el cuerpo de la celda desde el origen:
\begin{equation}
  r_B = (1-0)e_1 + (1-0)e_2 + (1-0)e_3 = e_1 + e_2 + e_3.
\end{equation}
Al ser las tres componentes la unidad, no requieren normalización.
\begin{equation}
  \text{Índices:} \quad [111]
\end{equation}

\subsection{Caso C: Vector Generalizado}
\textbf{Datos:} $P_i = (1/2, 1, 0)$, $P_f = (0,0,1)$.

Calculamos las componentes fraccionarias del desplazamiento:
\begin{equation}
  r_C = (0 - 1/2)e_1 + (0 - 1)e_2 + (1 - 0)e_3 = -\frac{1}{2}e_1 - e_2 + e_3.
\end{equation}
Para satisfacer la definición de índices de Miller (números enteros), aplicamos
un factor de escala escalar $\lambda = 2$ para eliminar el denominador, una
operación que es invariante respecto a la dirección del vector:
\begin{equation}
  2 \cdot r_C = -e_1 - 2e_2 + 2e_3.
\end{equation}
Adoptando la convención cristalográfica donde los componentes negativos se
denotan con una barra superior ($\bar{u}$):
\begin{equation}
  \text{Índices:} \quad [\bar{1}\bar{2}2]
\end{equation}

\section{Determinación de Índices de Miller para Planos Cristalográficos}

La notación de Miller $(hkl)$ especifica una familia de planos mediante un
vector normal en el espacio recíproco. El procedimiento estándar invierte los
interceptos axiales y normaliza a enteros coprimos. Formalmente, si los
interceptos son $x_1 e_1, x_2 e_2, x_3 e_3$, los índices se derivan de la
relación de proporcionalidad $h:k:l = x_1^{-1} : x_2^{-1} : x_3^{-1}$.

\subsection{Plano A: El Plano Octaédrico}
\textbf{Interceptos:} $(1, 0, 0), (0, 1, 0), (0, 0, 1)$.

Los interceptos normalizados respecto a los parámetros de red son $\mathbf{x} =
(1, 1, 1)$. Calculamos los recíprocos:
\begin{equation}
  \left( 1^{-1}, 1^{-1}, 1^{-1} \right) = (1, 1, 1).
\end{equation}
Al ser enteros unitarios, no requieren reducción.
\begin{equation}
  (hkl) = (111).
\end{equation}
Este plano representa la máxima densidad planar en sistemas FCC.

\subsection{Plano B: Plano Prismático General}
\textbf{Interceptos:} $(1, 0, 0), (0, 2, 0)$.

La ausencia de un tercer intercepto explícito implica paralelismo con el eje
$e_3$. Matemáticamente, el intercepto tiende al infinito asintótico:
$\mathbf{x} = (1, 2, \infty)$.
Tomando los inversos:
\begin{equation}
  \left( \frac{1}{1}, \frac{1}{2}, \lim_{\xi \to \infty}\frac{1}{\xi} \right)
  = \left( 1, 0.5, 0 \right).
\end{equation}
Para recuperar la naturaleza diofántica de los índices, aplicamos el factor de
escala mínimo $\lambda = 2$:
\begin{equation}
  (hkl) = 2 (1, 0.5, 0) = (210).
\end{equation}

\subsection{Plano C: Formulación Algebraica (Bivector)}
\textbf{Definición:} El plano subyacente al bivector $B = e_3 \wedge e_1$.

En lugar de buscar interceptos geométricos, utilizamos el álgebra de Clifford
$\mathcal{C}\ell_{3}$. El bivector $B$ codifica intrínsecamente la orientación del
subespacio plano. El vector normal $\mathbf{n}$ (asociado a los índices de
Miller) es el dual de Hodge (o complemento ortogonal) del bivector en el
espacio tridimensional euclidiano.

Utilizando el pseudoescalar $I = e_1 \wedge e_2 \wedge e_3$, la relación de
dualidad establece:
\begin{equation}
  \mathbf{n} \propto -I B = -(e_1 \wedge e_2 \wedge e_3)(e_3 \wedge e_1).
\end{equation}
Dada la anticonmutatividad del producto exterior y la normalización $e_i^2=1$:
\begin{equation}
  \mathbf{n} \propto e_2.
\end{equation}
El vector normal $e_2$ corresponde a los índices direccionales $[010]$. Por
consiguiente, la familia de planos ortogonales a este vector se denota:
\begin{equation}
  (hkl) = (010).
\end{equation}
