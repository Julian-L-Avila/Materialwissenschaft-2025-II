\documentclass[
  11pt,
  a4paper,
  numbers=noenddot
]{scrartcl}

\usepackage[utf8]{inputenc}
\usepackage[T1]{fontenc}
\usepackage{lmodern}
\usepackage[spanish]{babel}
\usepackage{csquotes}
\usepackage[sorting=nyt]{biblatex}

\addbibresource{./references.bib}

\usepackage[
  a4paper,
  left=2.5cm,
  right=2.5cm,
  top=3cm,
  bottom=3cm
]{geometry}
\usepackage{microtype}
\usepackage{xcolor}

\usepackage{amsmath, amssymb, amsfonts}
\usepackage{amsthm}
\usepackage{siunitx}
\usepackage{derivative}
\usepackage{braket}
\usepackage[version=4]{mhchem}

\usepackage{graphicx}
\usepackage{booktabs}
\usepackage{multirow}
\usepackage{listings}

\usepackage{enumitem}
\usepackage[title]{appendix}

\usepackage{hyperref}
\usepackage[spanish]{cleveref}

\hypersetup{
  colorlinks=true,
  linkcolor=black,
  citecolor=blue,
  urlcolor=blue,
  pdftitle={Actividad 03: Estructura Atómica y Enlaces},
  pdfauthor={Julian Avila},
  pdfsubject={Material Science},
  pdfkeywords={Ashby},
  bookmarksopen=true,
  bookmarksnumbered=true
}
\urlstyle{same}

\sisetup{
  group-digits=true,
  group-separator={\,},
  separate-uncertainty
}

\theoremstyle{plain}
\newtheorem{theorem}{Teorema}[section]
\newtheorem{proposition}[theorem]{Proposición}

\theoremstyle{definition}
\newtheorem{definition}{Definición}[section]
\newtheorem{problem}{Problema}[section]
\newtheorem{example}{Ejemplo}[section]

\theoremstyle{remark}
\newtheorem{remark}{Observación}[section]

\renewcommand{\arraystretch}{1.2}
\raggedbottom

\begin{document}

\title{Actividad 03: Estructura Atómica y Enlaces}
\author{Julian Avila \\ \small\texttt{jlavilam@udistrital.edu.co}}
\date{Universidad Distrital Francisco José de Caldas \\ \today}
\maketitle

\section{Actividad 02 -- Diagramas de Ashby}

\begin{problem}[Física de los Hornos]
	Explicar el principio físico de un horno de fundición (ferrería),
	reverbero, horno Bessemer, horno de arco, inducción, inducción al
	vacío y horno de tratamiento térmico para la metalurgia.
\end{problem}

\subsection{Alto Horno}

El alto horno es un reactor químico diseñado para la reducción
carbotérmica de minerales de hierro. Su principio operativo se
fundamenta en una gestión rigurosa del flujo de materia y energía
en contracorriente. Una carga sólida (mineral, coque, fundente)
desciende por gravedad, mientras que gases calientes ascienden desde la
base.

La generación de energía térmica es iniciada por la combustión de coque
con aire precalentado ($T > \qty{1200}{\kelvin}$), inyectado por toberas:
\ce{C + O2 -> CO2}. Esta reacción eleva la
temperatura por encima de \qty{2000}{\kelvin}. El \ce{CO2} reacciona
inmediatamente con carbono incandescente en la reacción endotérmica de
Boudouard, \ce{C + CO2 <=> 2CO},
generando el principal agente reductor, el monóxido de carbono.

Desde una perspectiva termodinámica, el proceso es viable porque la
energía libre de Gibbs ($ \Delta G $) para la oxidación del carbono
se vuelve más negativa que la de los óxidos de hierro a altas
temperaturas. La reducción del hierro ocurre secuencialmente a medida que la
carga desciende a zonas de mayor temperatura:
$ \ce{3Fe2O3 -> 2Fe3O4 -> FeO -> Fe} $.

El transporte de calor es un proceso complejo que involucra convección
forzada, donde el número de Péclet es elevado, y radiación térmica
desde las zonas incandescentes hacia la carga porosa.

\subsection{Horno de Reverbero}

A diferencia del alto horno, este diseño aísla la carga del
combustible para evitar su contaminación. El principio físico
fundamental es la maximización de la transferencia de calor por
radiación. Una llama y gases calientes fluyen a lo largo de un techo
refractario abovedado, que se calienta hasta la incandescencia.

Este techo actúa como un cuerpo radiante secundario, emitiendo energía
hacia la carga metálica situada en la solera. El flujo de calor
radiante ($q$) se rige por la ley de Stefan-Boltzmann,
$ q = \sigma \epsilon F_{1-2} (T_1^4 - T_2^4) $, donde $T_1$ es la
temperatura del techo, $T_2$ la de la carga, $\epsilon$ la emisividad y
$F_{1-2}$ es el factor de forma (o de visión), un término geométrico
que cuantifica la fracción de energía que intercepta la carga. La
geometría del horno es, por tanto, un parámetro de diseño crítico.

\subsection{Convertidor Bessemer}

La física del convertidor Bessemer es la de un reactor químico
autógeno, cuya energía proviene de reacciones de oxidación exotérmicas
\emph{in situ}. No requiere fuente de calor externa. Se inyecta aire u
oxígeno a presión a través del arrabio fundido, induciendo la oxidación
selectiva de impurezas.

La termodinámica dicta la secuencia de oxidación: elementos con mayor
afinidad por el oxígeno, como el silicio
(\ce{Si + O2 -> SiO2}) y el manganeso
(\ce{2Mn + O2 -> 2MnO}), reaccionan primero,
formando una escoria. Finalmente, se oxida el carbono,
\ce{2C + O2 -> 2CO}, cuya violenta formación
gaseosa provoca una ebullición característica.

Desde la fluidodinámica, la inyección del gas genera una turbulencia
extrema y un área interfacial gas-líquido inmensa, lo que resulta en
una cinética de transferencia de masa extraordinariamente rápida,
completando el afino del acero en minutos.

\subsection{Horno de Arco Eléctrico}

Su principio físico es la generación de un arco voltaico de alta
potencia entre electrodos de grafito y la carga metálica. El arco es un
plasma, un gas ionizado estable cuyas temperaturas pueden exceder los
\qty{6000}{\kelvin}.

La transferencia de energía al metal es un proceso multifísico:
\begin{enumerate}
	\item \textbf{Radiación térmica:} Es el mecanismo dominante. La
		columna de plasma irradia energía intensamente según la ley de
		Stefan-Boltzmann.
	\item \textbf{Transferencia en los electrodos:} Bombardeo directo de
		electrones e iones en los puntos de anclaje del arco con el baño
		metálico.
	\item \textbf{Convección:} Transferencia de calor desde el plasma
		caliente al metal.
\end{enumerate}
La potencia eléctrica disipada, $ P = V_{\text{arc}} I_{\text{arc}} $, se
controla con precisión para optimizar la fusión de la chatarra.

\subsection{Calentamiento por Inducción}

Este horno opera bajo los principios del electromagnetismo. Una
corriente alterna ($ I(t) $) fluye por una bobina que envuelve un
crisol con la carga metálica. Por la ley de Ampère, esta corriente
genera un campo magnético variable en el tiempo, $ B(t) $.

La ley de Faraday, $ \nabla \wedge E = - \pdv{B}!{t} $,\footnote{Se emplea uso del
formalismo del álgebra de Clifford $\mathcal{C}\ell_{3} (\mathbb{R})$, donde $E$
es un vector y $B$ un bivector.} dicta que $B(t)$ induce
un campo eléctrico $E$. Este campo impulsa la circulación de corrientes
parásitas (Eddy currents), $J = \sigma_e E$.

El calentamiento se produce por efecto Joule, con una densidad de
potencia volumétrica dada por $ p = J \cdot E = J^2 \sigma^{-1}_e $. Un
fenómeno clave es el efecto piel (\emph{skin effect}), que confina a
$J$ a una capa superficial de profundidad
$ \delta = \sqrt{2 \rho_e (\omega \mu)^{-1}} $, donde $\rho_e$ es la
resistividad, $\omega$ la frecuencia angular y $\mu$ la permeabilidad
magnética.

Adicionalmente, la interacción del campo $B$ con las corrientes $J$
genera una fuerza de Lorentz ($ F = J \cdot B $), que induce un
flujo de agitación (stirring) en el metal líquido, garantizando una
homogeneización térmica y química superior.

\subsection{Inducción al Vacío (VIM)}

El horno VIM acopla la física de la inducción con los principios de la
termodinámica en baja presión. Al operar en vacío, la presión parcial
de gases ($p_i$) sobre el metal se reduce drásticamente.

Según la ley de Sieverts, la solubilidad de gases diatómicos como
\ce{N2}, \ce{O2} y \ce{H2} en el metal es proporcional a $\sqrt{p_i}$.
Una disminución de $p_i$ reduce el potencial químico del gas en la
atmósfera, impulsando su desorción del metal fundido para alcanzar un
nuevo equilibrio.

El vacío también facilita la eliminación de elementos volátiles con
alta presión de vapor y desplaza equilibrios de desoxidación como
\ce{[C] + [O] <=> CO_{(g)}}.
La extracción continua del producto gaseoso CO, conforme al principio
de Le Châtelier, lleva la reacción a completarse.

\subsection{Tratamiento Térmico: Física del Estado Sólido}

Estos hornos manipulan la microestructura del material en estado
sólido. El principio físico subyacente es la cinética de las
transformaciones de fase, controlada por ciclos térmicos precisos.

El calentamiento de una pieza se rige por la ecuación de difusión del
calor, $ \pdv{T}!{t} = \alpha \nabla^2 T $, donde $\alpha$ es la
difusividad térmica. El objetivo es llevar el material a una región de
su diagrama de fases donde una nueva estructura cristalina sea estable
(e.g., austenización del acero).

La etapa crítica es el enfriamiento. La velocidad, $ \mdif{t}{T} $,
determina la transformación de fase resultante.
\begin{itemize}
	\item \textbf{Enfriamiento lento:} Permite transformaciones por
		difusión (nucleación y crecimiento), formando microestructuras
		como ferrita, perlita o bainita.
	\item \textbf{Enfriamiento rápido (temple):} Suprime la difusión. La
		transformación ocurre de forma displaciva (sin difusión), mediante
		un mecanismo de cizallamiento de la red cristalina, dando lugar a
		fases metaestables como la martensita.
\end{itemize}

\begin{problem}[Diagrama de Ashby]
	Interprete el diagrama de Ashby presentado en la \cref{fig:ashby-3}.
	\begin{figure}[htbp!]
		\centering
		\includegraphics[width=0.5\linewidth]{./images/ashby.png}
		\caption{Diagrama de Ashby (Módulo de Young y Densidad).}
		\label{fig:ashby-3}
	\end{figure}
\end{problem}

\subsubsection{Eje Y: Módulo de Young, \texorpdfstring{$E$ (\si{\giga\pascal})}{E (GPa)}}

Esta propiedad cuantifica la rigidez de un material en el régimen
elástico. Se define como la constante de proporcionalidad entre el
esfuerzo uniaxial ($\sigma$) y la deformación unitaria ($\epsilon$)
según la Ley de Hooke, $\sigma = E \epsilon$.

El módulo de Young está determinado por la naturaleza de los enlaces interatómicos.
Es proporcional a la curvatura del potencial de energía interatómica $U(r)$ en
la distancia de equilibrio $r_0$:
$$ E \propto \left( \mdif{r}^2 U \right)_{r=r_0} $$
Un valor de $E$ elevado implica enlaces fuertes y una gran resistencia a
la deformación elástica. El rango logarítmico del eje abarca siete
órdenes de magnitud, evidenciando la vasta diversidad de materiales.

\subsubsection{Eje X: Densidad, \texorpdfstring{$\rho$
(\si{\mega\gram\per\cubic\meter})}{rho (Mg/m3)}}

La densidad es la masa por unidad de volumen. Microscópicamente, está
determinada por tres factores:
\begin{enumerate}
	\item La masa atómica ($A$) de los elementos constituyentes.
	\item El factor de empaquetamiento atómico (APF), una propiedad
		geométrica de la estructura cristalina o molecular.
	\item La porosidad presente en el material.
\end{enumerate}

\subsection{Interpretación de las Familias de Materiales}

Las regiones en el diagrama agrupan materiales con características
físicas y estructurales similares.

\subsubsection{Metales (Gris)}

Se ubican en la región superior derecha: alta rigidez y alta densidad.
Su enlace metálico, caracterizado por un gas de electrones
deslocalizados, proporciona una fuerte cohesión no direccional. Esto,
sumado a sus estructuras cristalinas compactas (e.g., FCC, BCC con APF
elevados), resulta en módulos de Young altos y densidades notables.

\subsubsection{Cerámicas Técnicas (Rojo)}

Ocupan la cúspide del diagrama, exhibiendo la máxima rigidez. Sus
enlaces, predominantemente iónicos (fuertes, no direccionales) y/o
covalentes (muy fuertes y direccionales), exigen una energía muy alta
para deformarse. Su densidad es moderada, menor que la de muchos
metales, pues suelen componerse de átomos más ligeros (\ce{O}, \ce{Si}, \ce{Al},
\ce{C}).

\subsubsection{Polímeros y Elastómeros (Azul)}

Se localizan en la región central e inferior, con baja rigidez y
densidad. Su comportamiento mecánico no está dominado por el
estiramiento de enlaces primarios, sino por interacciones secundarias
(fuerzas de van der Waals) y por la rotación de enlaces en las largas
cadenas moleculares. En los elastómeros, la rigidez es particularmente
baja y su origen es fundamentalmente entrópico: la fuerza restauradora
proviene de la tendencia del sistema a maximizar su entropía
conformacional (retorno al estado de ovillo estadístico).

\subsubsection{Compuestos (Amarillo)}

Estos materiales de ingeniería se posicionan estratégicamente para
optimizar la relación rigidez-peso. Combinan una matriz (usualmente
polimérica, de baja densidad) con un refuerzo de alta rigidez (e.g.,
fibras de carbono, vidrio). Su objetivo es maximizar el módulo
específico, $E \rho^{-1}$, un índice de rendimiento clave. Por ello, se
ubican en la deseable región superior izquierda del diagrama.

\subsubsection{Materiales Naturales (Marrón)}

Exhiben una gran diversidad, a menudo como compuestos jerárquicos
optimizados por la evolución. La madera, un compuesto celular de fibras
de celulosa en una matriz de lignina, logra una excelente rigidez
específica.

\subsubsection{Espumas (Verde)}

Ocupan la esquina inferior izquierda: densidad y rigidez extremadamente
bajas. Son estructuras celulares con una alta fracción de volumen de
gas. Su respuesta mecánica no está dictada por las propiedades del
material base, sino por la arquitectura de la espuma. La deformación se
produce por la flexión y el pandeo de las paredes celulares.

\begin{problem}[Material para una taza]
	Para fabricar una taza, ¿qué se debe tener en cuenta en la selección
	del material desde una perspectiva de la física de materiales?
\end{problem}

La selección de un material para una taza es un problema de diseño que
requiere satisfacer un conjunto de restricciones termofísicas,
mecánicas y químicas.

\subsubsection{Restricciones Termofísicas}

\paragraph{Resistencia al Choque Térmico.} El material debe soportar
gradientes de temperatura abruptos sin fracturarse. Al
verter un líquido caliente, la superficie interna se expande
repentinamente, generando un esfuerzo de tracción en la superficie
externa (más fría).

\paragraph{Aislamiento Térmico.} Para mantener la bebida caliente y el
exterior a una temperatura manejable, se requiere una baja
conductividad térmica, $k$. A nivel microscópico, esto implica una baja
eficiencia en la propagación de fonones (cuantos de vibración de la
red). Materiales con estructuras cristalinas complejas, desorden
atómico (vidrios) o grandes diferencias de masa atómica favorecen la
dispersión de fonones y, por tanto, poseen un bajo valor de $k$.

\subsubsection{Restricciones Mecánicas}

\paragraph{Rigidez y Resistencia a la Compresión.} El material debe ser
suficientemente rígido ($E$ elevado) y resistente a la compresión para
soportar su propio peso y el del líquido sin deformarse. Las cerámicas
y vidrios, con sus fuertes enlaces iónicos/covalentes, cumplen
excelentemente este requisito.

\paragraph{Fragilidad y Tenacidad a la Fractura.} La contrapartida de
los enlaces cerámicos es su fragilidad. La estructura atómica inhibe el
deslizamiento de dislocaciones, impidiendo la deformación plástica como
mecanismo de disipación de energía. Por tanto, la energía de un
impacto se invierte en la propagación de grietas. La propiedad que
cuantifica esta resistencia es la tenacidad a la fractura, $K_{\text{Ic}}$.
Las cerámicas tienen valores de $K_{\text{Ic}}$ muy bajos en comparación con
los metales, lo que las hace susceptibles a la fractura catastrófica
ante un defecto o impacto.

\paragraph{Dureza Superficial.} Se requiere una alta dureza para resistir
la abrasión y el rayado por el uso de utensilios (e.g., cucharas) y
durante la limpieza. La dureza está directamente relacionada con la
fuerza de los enlaces interatómicos, por lo que las cerámicas son
ideales en este aspecto.

\subsubsection{Restricciones Químicas y de Biocompatibilidad}

\paragraph{Inercia Química.} El material debe ser químicamente inerte. No
debe reaccionar, corroerse ni degradarse en contacto con bebidas que
pueden ser ácidas (café, pH $\approx 5$) o alcalinas. La elevada
energía de los enlaces en óxidos cerámicos (e.g., $\ce{SiO2}$,
$\ce{Al2O3}$) les confiere una estabilidad termodinámica excepcional en
estos entornos.

\paragraph{Biocompatibilidad.} El material debe ser apto para el contacto
con alimentos. Esto significa que no debe haber lixiviación (leaching)
de especies iónicas tóxicas hacia la bebida. En el caso de cerámicas
esmaltadas, es crucial que el vidriado se haya procesado a una
temperatura adecuada para garantizar la completa vitrificación y evitar
la liberación de metales pesados (e.g., \ce{Pb}, \ce{Cd}) que pudieran estar
presentes en los pigmentos.


\section{Actividad 02 -- Enlaces Atómicos}

\begin{problem}[Números Cuánticos y Propiedades Físicas]
  ¿Cómo los números cuánticos,
  gobiernan las propiedades físicas de un material?
\end{problem}

Las partículas elementales son excitaciones de campos cuánticos
fundamentales, clasificadas por su espín. El Teorema Espín-Estadística
establece una correspondencia fundamental: partículas de espín
semientero (e.g., electrones, espín $\frac{\hbar}{2}$) son fermiones, descritos
por una función de onda total $\Psi$ que debe ser antisimétrica bajo el
intercambio de las coordenadas (espaciales y de espín) de dos
partículas idénticas:
$ \Psi(x_1, x_2, \dots) = - \Psi(x_2, x_1, \dots) $.
Esta exigencia de antisimetría orquesta, desde el nivel más elemental,
la vasta diversidad de propiedades de los materiales.

La antisimetría de $\Psi$ implica directamente el Principio de Exclusión
de Pauli. El espacio de estados de un sistema de dos fermiones no es el
producto tensorial completo $\mathcal{H} \otimes \mathcal{H}$ del
espacio de Hilbert de una partícula, sino su subespacio
antisimétrico, $\Lambda^2(\mathcal{H})$. Un estado de dos partículas
$\ket{\alpha}$ y $\ket{\beta}$ se representa por su producto exterior:
$$ \ket{\Psi} = \ket{\alpha} \wedge \ket{\beta} \equiv \frac{1}{\sqrt{2}}
(\ket{\alpha} \otimes \ket{\beta} - \ket{\beta} \otimes \ket{\alpha}) $$
Por la propiedad antisimétrica del producto exterior ($a \wedge b = -b
\wedge a$), si intentamos colocar ambos fermiones en el mismo estado
cuántico $\ket{\alpha}$, el estado del sistema es necesariamente el
vector nulo:
$$ \ket{\alpha} \wedge \ket{\alpha} = 0 $$
Un estado nulo tiene probabilidad cero de existir. Esta restricción
prohíbe la ``condensación'' de fermiones en un mismo estado, forzándolos
a apilarse en una jerarquía de estados energéticos.

Al formar un sólido, los orbitales atómicos discretos se solapan, y
según modelos como el de ``tight-binding'', sus niveles de energía se
hibridan en cuasi-continuos energéticos: las bandas de
energía. Los electrones, obedeciendo la estadística de Fermi-Dirac,
llenan estas bandas hasta una energía máxima a $T= \qty{0}{\kelvin}$, la
energía de Fermi ($E_F$). La topología de estas bandas cerca
de $E_F$ define las propiedades electrónicas del material.
\begin{itemize}
  \item \textbf{Aislantes/Semiconductores:} $E_F$ se sitúa dentro de
    una brecha energética (\emph{band gap}, $E_g$) que separa una banda
    de valencia llena de una banda de conducción vacía. Se requiere una
    energía $\geq E_g$ para excitar un electrón y permitir la
    conducción.
  \item \textbf{Metales:} $E_F$ cruza una o más bandas, resultando en
    una banda parcialmente llena. Esto garantiza una densidad de
    estados finita en la energía de Fermi, $g(E_F) > 0$, permitiendo la
    excitación de electrones a estados vacíos con una energía
    infinitesimal y, por ende, una alta conductividad eléctrica.
\end{itemize}

El magnetismo surge de la interacción de intercambio, un efecto
sin análogo clásico, fruto de la interacción entre la repulsión de
Coulomb y el principio de antisimetría. La antisimetría impone una
correlación entre las coordenadas espaciales y de espín de los
electrones. Con la minimización de su energía de repulsión electrostática, los
electrones pueden adoptar una configuración espacial que los separe, lo
cual, satisfaciendo el teorema, puede forzar a sus espines a
alinearse. Este fenómeno se modela con un término en el Hamiltoniano
efectivo, $ H_{\text{ex}} = -2J \mathbf{S}_i \cdot \mathbf{S}_j $, donde
$J$ es la integral de intercambio. Si $J>0$, se favorece el
ferromagnetismo (espines paralelos).

Así, propiedades macroscópicas como la conductividad, la transparencia
o el magnetismo son manifestaciones directas de la naturaleza
fermiónica del electrón y la simetría fundamental de su función de onda.

\begin{problem}[Familias de la tabla periódica]
  Ubique las familias de elementos en la tabla periódica y describa
  su utilidad.
  \begin{figure}[htbp!]
    \centering
    \includegraphics[width=0.8\linewidth]{./images/Periodic_table_large.pdf}
    \caption{Tabla Periódica. \cite{commons_fileperiodic_2025}}
    \label{fig:periodic-table}
  \end{figure}
\end{problem}

\subsection{Metales Alcalinos y Alcalinotérreos}

\paragraph{Ubicación.} Grupos 1 y 2, el bloque $s$ de la tabla periódica.

Configuración de valencia $ns^1$ y $ns^2$. El apantallamiento eficaz del núcleo
por las capas internas y su gran radio atómico resulta en las energías de
ionización más bajas de todos los elementos. Esto los convierte en agentes
reductores extremadamente potentes y químicamente muy reactivos. Su enlace
metálico es relativamente débil, al contribuir con solo uno o dos electrones al
``mar'' de Fermi, lo que explica su baja densidad y blandura.

\subsection{Metales de Transición}

\paragraph{Ubicación.} Grupos 3 al 12, el bloque $d$.

Caracterizados por el llenado de los orbitales $(n-1)d$. La pequeña diferencia
de energía entre los niveles $ns$ y $(n-1)d$ permite múltiples estados de
oxidación, propiedad que los convierte en excelentes catalizadores. En presencia
de un campo ligando, la degeneración de los orbitales $d$ se rompe. Las
transiciones electrónicas $d$-$d$ entre estos niveles absorben fotones en el
espectro visible, generando compuestos coloreados. El llenado de orbitales $d$,
a menudo deja electrones desapareados. La interacción de intercambio entre estos
espines puede alinear sus momentos magnéticos, dando lugar al ferromagnetismo
(\ce{Fe}, \ce{Co}, \ce{Ni}).

\subsection{Lantánidos y Actínidos}

\paragraph{Ubicación.} Bloque $f$, las dos filas inferiores.

Se caracterizan por el llenado de los orbitales $4f$ (lantánidos) y $5f$
(actínidos). Los orbitales $4f$ están espacialmente contraídos y apantallados
por las capas externas ($5s^2$, $5p^6$), por lo que participan poco en el enlace
químico. Esto provoca que todos los lantánidos tengan propiedades químicas muy
similares. Su configuración $f$ les confiere propiedades magnéticas y ópticas
únicas.  Los actínidos son todos radiactivos debido a la inestabilidad nuclear
de sus isótopos.

\subsection{Metaloides y No Metales}

\paragraph{Ubicación.} Región derecha de la tabla, separados por una
línea diagonal.

Los metaloides (\ce{Si}, \ce{Ge}) son semiconductores. Su estructura de bandas
presenta una brecha energética ($E_g$) pequeña, lo que permite controlar su
conductividad mediante dopaje (introducción de impurezas) para crear portadores
de carga (electrones o huecos). Los no metales (\ce{C}, \ce{N}, \ce{O}, \ce{S})
tienen altas energías de ionización y afinidades electrónicas; forman enlaces
covalentes fuertes y direccionales. El Carbono (\ce{C}) es excepcional por su
capacidad de hibridación ($sp, sp^2, sp^3$), base de la química orgánica y de
alótropos como el diamante y el grafeno.

\subsection{Halógenos y Gases Nobles}

\paragraph{Ubicación.} Grupos 17 y 18, extremo derecho de la tabla.

\textbf{Halógenos:} Configuración de valencia $ns^2np^5$. Les falta un electrón
para alcanzar una capa llena, lo que se traduce en una afinidad electrónica muy
alta. Son los elementos más electronegativos y reactivos.

\textbf{Gases Nobles:} Configuración $ns^2np^6$. Esta capa de valencia cerrada
es una configuración de mínima energía, lo que les confiere una excepcional
estabilidad e inercia química.


\section{Actividad 03 -- Enlaces Atómicos II}

\begin{problem}[Propiedades según tipo de Enlace]
  ¿Qué propiedades fisicoquímicas se originan en cada tipo de enlace
  atómico y cuáles son direccionales?
\end{problem}

\subsection{Enlace Covalente (Direccional)}

Este enlace se origina en la interacción de intercambio, un efecto
puramente cuántico que disminuye la energía de un sistema de electrones
cuando sus funciones de onda se solapan constructivamente. Es
altamente direccional porque la maximización de este
solapamiento, y por ende la minimización de la energía, depende
críticamente de la geometría y orientación de los orbitales atómicos
de valencia ($s, p, d, f$). La hibridación de orbitales, como la $sp^3$
en el carbono, genera orbitales orientados en ángulos específicos (e.g.,
\ang{109.5} en una geometría tetraédrica), fijando la estructura
molecular y cristalina.

\paragraph{Propiedades Fisicoquímicas.}
\begin{itemize}
  \item \textbf{Mecánicas:} Si los enlaces forman una red
    tridimensional (e.g., diamante, \ce{SiO2}), el material es
    extremadamente duro, con un módulo de compresibilidad elevado. La
    energía para deformar el cristal es la energía necesaria para
    distorsionar estos fuertes enlaces localizados.
  \item \textbf{Electrónicas:} Generalmente son aislantes o
    semiconductores. En el lenguaje de la teoría de bandas, los
    orbitales moleculares enlazantes forman una banda de valencia
    llena, mientras que los antienlazantes forman una banda de
    conducción vacía. La fuerte localización de los electrones crea
    una gran brecha de energía ($E_g$) entre ambas, impidiendo la
    conductividad.
  \item \textbf{Solubilidad:} Suelen ser insolubles en disolventes
    polares. Es energéticamente desfavorable romper los fuertes enlaces
    de hidrógeno de un solvente como el agua para solvatar una molécula
    que solo puede interactuar mediante débiles fuerzas de van der
    Waals.
\end{itemize}

\subsection{Enlace Iónico (No Direccional)}

Se produce por la transferencia de electrones, creando cationes y
aniones que interactúan mediante la fuerza de Coulomb. Esta fuerza es
central e isotrópica ($F \propto r^{-2}$), por lo que es no
direccional. Un ion atrae a todos los iones de carga opuesta en su
vecindad con igual intensidad. La estructura cristalina resultante es
aquella que maximiza el empaquetamiento y la energía de cohesión.

\paragraph{Propiedades Fisicoquímicas.}
\begin{itemize}
  \item \textbf{Mecánicas:} Son duros pero frágiles. La dureza
    proviene de la fuerte atracción electrostática. La fragilidad se
    debe a que un deslizamiento de planos atómicos (movimiento de una
    dislocación) enfrentaría iones de la misma carga, generando una
    repulsión que fractura el cristal.
  \item \textbf{Eléctricas:} Son aislantes en estado sólido,
    pues los iones están fijos en la red. Sin embargo, en estado
    fundido o en disolución, los iones ganan movilidad y se convierten
    en excelentes conductores iónicos.
  \item \textbf{Térmicas:} Poseen puntos de fusión muy elevados, ya que
    se requiere una gran energía térmica para superar la energía
    de red del cristal.
\end{itemize}

\subsection{Enlace Metálico (No Direccional)}

Se describe como una red de cationes inmersa en un ``gas'' o ``mar'' de
electrones de valencia deslocalizados. Cuánticamente, los electrones de
valencia ocupan estados deslocalizados (ondas de Bloch) que se extienden
por todo el cristal. La atracción entre el gas de electrones (negativo)
y los núcleos iónicos (positivos) es de naturaleza electrostática y, por
tanto, no direccional.

\paragraph{Propiedades Fisicoquímicas.}
\begin{itemize}
  \item \textbf{Conductividad:} Excelente conductividad eléctrica y
    térmica. Los electrones en estados cercanos a la energía de Fermi
    son libres de acelerarse en un campo eléctrico. Estos mismos
    electrones móviles son portadores muy eficientes de energía
    térmica, una correlación descrita por la Ley de Wiedemann-Franz.
  \item \textbf{Mecánicas:} Son dúctiles y maleables. La
    naturaleza deslocalizada y no direccional del enlace permite que
    los planos atómicos se deslicen unos sobre otros sin que se rompa
    la cohesión del material. El ``mar'' de electrones se adapta a la
    nueva configuración.
  \item \textbf{Ópticas:} Son opacos y lustrosos. El gas de electrones
    libres se comporta como un plasma. Por debajo de una frecuencia
    característica (la frecuencia de plasma, $\omega_p$),
    la luz es eficientemente reflejada. Por encima de $\omega_p$ (en
    el UV para muchos metales), se vuelven transparentes.
\end{itemize}

\subsection{Enlaces Secundarios (Intermoleculares)}

\subsubsection{Enlace por Puente de Hidrógeno (Direccional)}

Es una interacción fuerte que ocurre cuando un \ce{H} unido a un átomo muy
electronegativo (\ce{N}, \ce{O}, \ce{F}; el ``donador'') es atraído por otro
átomo electronegativo (el ``aceptor''). Es altamente direccional porque no es
puramente electrostático; posee un carácter covalente parcial, modelado como una
interacción entre el par solitario del aceptor y el orbital antienlazante
$\sigma^*$ del enlace donador-\ce{H}. La fuerza es máxima con una alineación
donador-\ce{H}-aceptor de $\approx$\ang{180}.

\paragraph{Propiedades Fisicoquímicas.}
\begin{itemize}
  \item \textbf{Térmicas:} Causa un aumento anómalo en los puntos de
    ebullición. La energía extra para romper estos enlaces explica por
    qué el \ce{H2O} (P.E. \qty{100}{\celsius}) es líquido, mientras que
    el \ce{H2S} (P.E. \qty{-60}{\celsius}) es un gas.
  \item \textbf{Estructurales:} Define la doble hélice del ADN y las
    estructuras secundarias de proteínas. En el hielo (fase Ih), fuerza
    una estructura hexagonal abierta que es menos densa que el agua
    líquida.
\end{itemize}

\subsubsection{Interacciones de Van der Waals (No Direccionales)}

Son fuerzas universales pero débiles, originadas por correlaciones en
las fluctuaciones de las nubes electrónicas. Se componen de:
\begin{enumerate}
  \item Interacciones dipolo-dipolo (Keesom).
  \item Interacciones dipolo-dipolo inducido (Debye).
  \item Interacciones dipolo inducido-dipolo inducido (dispersión de
    London), que son dominantes y de origen cuántico ($U \propto -r^{-6}$).
\end{enumerate}
Aunque algunas componentes tienen dependencia angular, en fases
condensadas se promedian, resultando en una cohesión no
direccional que favorece el empaquetamiento denso.

\paragraph{Propiedades Fisicoquímicas.}
\begin{itemize}
  \item \textbf{Térmicas:} Los sólidos moleculares (yodo, hielo seco)
    tienen puntos de fusión y ebullición muy bajos, ya que la energía
    térmica ($k_B T$) supera fácilmente estas débiles interacciones.
  \item \textbf{Mecánicas:} Son materiales blandos y compresibles, con
    baja resistencia mecánica.
\end{itemize}

\begin{problem}[Grafito y Diamante]
  El grafito es buen conductor eléctrico (solo en el plano de sus
  láminas), mientras que el diamante es un aislante. Explique esta
  diferencia en función de sus enlaces covalentes y la estructura
  electrónica.
\end{problem}

La abismal diferencia en la conductividad eléctrica entre estos dos
alótropos del carbono es una manifestación directa de sus distintas
estructuras de bandas electrónicas, que a su vez son consecuencia de
la hibridación de los orbitales de valencia.

\subsection{Diamante: Aislante de Banda Prohibida Ancha
(\texorpdfstring{$sp^3$}{sp3})}

En la estructura del diamante, cada átomo de carbono presenta una
hibridación $sp^3$, formando cuatro orbitales híbridos idénticos
orientados tetraédricamente (\ang{109.5}). Cada uno de estos orbitales
forma un enlace covalente simple (enlace $\sigma$) con un átomo vecino,
creando una red tridimensional rígida y continua.

Desde la perspectiva de la física del estado sólido, la totalidad de
los cuatro electrones de valencia de cada átomo está fuertemente
localizada en estos enlaces $\sigma$. Estos estados electrónicos
enlazantes constituyen la banda de valencia, la cual está
completamente llena. Los estados antienlazantes ($\sigma^*$) forman la
banda de conducción, que está completamente vacía.

La clave es que entre ambas bandas existe una brecha de energía
prohibida (band gap, $E_g$) muy grande, del orden de
\qty{5.5}{\electronvolt}. A temperatura ambiente, la energía térmica
disponible ($k_B T \approx \qty{0.025}{\electronvolt}$) es
insignificante en comparación con $E_g$, por lo que es prácticamente
imposible excitar un electrón de la banda de valencia a la de
conducción. Sin portadores de carga móviles, el diamante es uno de los
mejores aislantes eléctricos conocidos.

\subsection{Grafito: Semimetal Anisotrópico (\texorpdfstring{$sp^2$}{sp2})}

En el grafito, cada átomo de carbono utiliza una hibridación
$sp^2$, formando tres enlaces $\sigma$ con sus vecinos en un plano, lo que
da lugar a una estructura de láminas hexagonales. El cuarto electrón de
valencia reside en un orbital $p$ no hibridado ($p_z$),
perpendicular a dicho plano.

Los orbitales $p_z$ de todos los átomos en una lámina se solapan
lateralmente, formando bandas $\pi$ (enlazante) y $\pi^*$
(antienlazante) deslocalizadas a lo largo de todo el plano
bidimensional. Aquí reside la diferencia fundamental con el diamante:
en el grafito, la banda $\pi$ (de valencia) y la banda $\pi^*$ (de
conducción) no están separadas por una brecha de energía. De hecho,
se tocan en puntos específicos (los puntos de Dirac) de la
zona de Brillouin hexagonal.

Esto significa que la energía de Fermi ($E_F$) se encuentra
precisamente en la energía donde estas bandas se degeneran. El material
no tiene brecha energética ($E_g=0$), lo que lo clasifica como un
semimetal. Con una densidad de estados finita en la energía de
Fermi, una cantidad infinitesimal de energía es suficiente para excitar
electrones a estados conductores. Estos electrones deslocalizados en el
sistema $\pi$ se mueven libremente a lo largo de las láminas,
otorgando al grafito una alta conductividad eléctrica en el plano.

Entre las láminas, la unión se debe a débiles fuerzas de Van der Waals,
lo que resulta en un muy mal solapamiento de las funciones de onda en
la dirección perpendicular. Esto impide el movimiento de electrones
entre capas, haciendo que el grafito sea un aislante en esa dirección y,
por tanto, un material con una alta anisotropía en su conductividad eléctrica.

\begin{problem}[Potencial Lennard-Jones]
    Interprete el gráfico del Potencial Lennard-Jones presentado en la
    \cref{fig:lennar-jones}.
    \begin{figure}[htbp!]
        \centering
        \includegraphics[width=0.8\linewidth]{./images/lennard.pdf}
        \caption{Potencial Lennard-Jones.}
        \label{fig:lennar-jones}
    \end{figure}
\end{problem}

El potencial de Lennard-Jones es un modelo fenomenológico fundamental en
la física de la materia condensada. Describe la energía potencial de
interacción $V(r)$ entre un par de átomos o moléculas neutras y
esféricamente simétricas en función de su separación $r$. Su forma
funcional captura la física esencial de las interacciones no enlazantes
y sirve como pilar en la simulación de dinámica molecular.

La expresión matemática del potencial es:
$$ V(r) = 4\epsilon \left[ \left(\frac{\sigma}{r}\right)^{12} -
\left(\frac{\sigma}{r}\right)^6 \right] $$
donde $\epsilon$ es la profundidad del pozo de potencial y $\sigma$ es
la distancia a la cual el potencial es nulo.

\subsection{Interpretación de los Términos Físicos}

El potencial se compone de dos términos que representan un balance entre
atracción y repulsión.

\paragraph{Término Atractivo ($ -\sigma^6 r^{-6} $).}
Dominante a distancias intermedias y largas. Este término modela la
fuerza de dispersión de London, una interacción de van der
Waals de origen puramente cuántico. Surge de las correlaciones entre
fluctuaciones cuánticas en la distribución de carga de los átomos. Un
dipolo instantáneo en un átomo induce un dipolo en el vecino, y la
interacción entre estos dipolos fluctuantes resulta en una atracción.
La dependencia $r^{-6}$ es el término de orden más bajo derivado de la
teoría de perturbaciones de segundo orden para la interacción
dipolo-dipolo en el régimen no retardado.

\paragraph{Término Repulsivo ($ + \sigma^{12}r^{-12} $).}
Dominante a distancias muy cortas ($r \ll \sigma$). Este término
representa la repulsión de Pauli. Cuando las nubes
electrónicas de dos átomos comienzan a solaparse, el Principio de
Exclusión de Pauli obliga a los electrones a ocupar orbitales de mayor
energía (antienlazantes), resultando en un drástico aumento de la
energía total. La elección del exponente 12, aunque físicamente menos
rigurosa que un potencial exponencial (tipo Buckingham), es una
aproximación matemáticamente conveniente que modela una ``pared'' de
potencial extremadamente abrupta.

\subsection{Análisis de la Gráfica y sus Parámetros}

\paragraph{Comportamiento Asintótico.}
Para $r \to \infty$, $V(r) \to 0$, lo que refleja correctamente la
ausencia de interacción a grandes distancias.

\paragraph{Punto de Cruce ($ r = \sigma $).}
A esta distancia, $V(\sigma) = 4\epsilon [1 - 1] = 0$. El parámetro
$\sigma$ se interpreta como el diámetro de colisión efectivo
de las partículas. Para $r < \sigma$, la interacción es repulsiva
($V > 0$), y para $r > \sigma$, es atractiva ($V < 0$).

\paragraph{Mínimo del Potencial y Distancia de Equilibrio.}
La posición de equilibrio, $r_m$, corresponde al mínimo de $V(r)$,
donde la fuerza neta es nula, $F(r_m) = - \mdif{r} V |_{r_m} = 0$.
Derivando el potencial e igualando a cero:
$$ \mdif{r}{V} = 4\epsilon \left[ -12 \sigma^{12}{r^{-13}} +
6\sigma^6{r^{-7}} \right] = 0 \implies 2\sigma^6 = r_m^6 $$
se obtiene $r_m = 2^{1/6}\sigma \approx 1.122\sigma$. Esta es la
distancia de separación más estable.

\paragraph{Profundidad del Pozo ($-\epsilon$).}
El valor del potencial en el mínimo es $V(r_m) = -\epsilon$. Este valor
representa la energía de cohesión o la energía de enlace del
par. Es la energía requerida para separar las dos partículas desde su
posición de equilibrio hasta el infinito.

\paragraph{Vibraciones y Estabilidad.}
Cerca del mínimo, el potencial puede aproximarse por una parábola
(desarrollo de Taylor de segundo orden), análogo a un oscilador
armónico. La ``constante de resorte'' efectiva del enlace, $k$, está dada
por la curvatura del potencial en el mínimo: $k = \mdif{r}^2{V}|_{r_m}$.
Esta curvatura determina la frecuencia de vibración de las partículas
alrededor de su posición de equilibrio.


\printbibliography
\nocite{*}

\end{document}
