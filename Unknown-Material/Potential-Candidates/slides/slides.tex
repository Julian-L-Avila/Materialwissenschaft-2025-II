\documentclass[aspectratio=169]{beamer}

\usepackage{fontspec}
\usepackage{unicode-math}
\usepackage{graphicx}
\usepackage[dvipsnames]{xcolor}
\usepackage[sorting=nyt]{biblatex}
\usepackage{ragged2e}
\usepackage{setspace}
\usepackage{bookmark}
\usepackage{booktabs}
\usepackage{siunitx}
\usepackage{derivative}
\usepackage[version=4]{mhchem}
\usepackage{pgfplots}
\usepackage{tikz}
\usepackage{tikz-3dplot}
\usetikzlibrary{calc}
\usetikzlibrary{decorations.pathmorphing}
\usetikzlibrary{overlay-beamer-styles}
\usetikzlibrary{positioning, arrows.meta, backgrounds, patterns}

\pgfplotsset{compat=1.18}

\usetheme[numbering=counter, progressbar=frametitle, background=light]{metropolis}

\usecolortheme{dove}
\definecolor{primary}{RGB}{88, 24, 124}
\definecolor{secondary}{RGB}{0, 168, 150}

\setbeamercovered{transparent}
\setbeamercolor{frametitle}{fg=white, bg=primary}
\setbeamercolor{progress bar}{fg=secondary}
\setbeamerfont{frametitle}{size=\large, series=\bfseries}
\setbeamerfont{title}{size=\LARGE, series=\bfseries}
\setbeamercolor{alerted text}{fg=secondary}
\setbeamercolor{block title}{fg=primary}
\setbeamercolor{block body}{bg=white}

\setstretch{1.1}

\setmainfont{Fira Sans}
\setmathfont{TeX Gyre Pagella Math}

\title{Unknown Material Characterization}
\subtitle{Preliminary Results and Potential Candidates}
\date{\today}
\author{Julian L. Avila-Martinez}
\institute{Physics, Universidad Distrital Francisco José de Caldas \\
jlavilam@udistrital.edu.co}

\begin{document}

\begin{frame}
	\titlepage
\end{frame}

\begin{frame}{Outline}
	\setbeamertemplate{section in toc}[sections numbered]
	\tableofcontents
\end{frame}

% --- Introduction ---
\section{Introduction}
\begin{frame}{Analysis Objective}
	\begin{columns}
		\begin{column}{0.6\textwidth}
			The purpose of this preliminary phase is not immediate identification,
			but the \textbf{logical restriction} of the search space.

			\vspace{1em}
			We aim to classify the material based on:
			\begin{itemize}[<+->]
				\item Physical Properties
				\item Chemical Behavior
				\item Spectroscopic Data (XRF)
			\end{itemize}
		\end{column}
		\begin{column}{0.4\textwidth}
			\centering
			\begin{tikzpicture}
				\node[circle, draw=mLightBrown, fill=mLightBrown!10, text width=2cm,
					align=center, thick] {Unknown\\Sample};
			\end{tikzpicture}
		\end{column}
	\end{columns}
\end{frame}

% --- Physical Observations ---
\section{Experimental Results}
\begin{frame}{Physical and Morphological Evidence}
	\begin{alertblock}{Key Observation: Dermal Reaction}
		Upon accidental contact with skin, the grains leave persistent \textbf{dark/black spots}.
		\begin{itemize}[<+->]
			\item Indicates photoreduction of metal ions to elemental metal.
		\end{itemize}
	\end{alertblock}

	\vspace{0.5em}

	\begin{columns}[t]
		\begin{column}{0.48\textwidth}
			\textbf{Morphology}
			\begin{itemize}[<+->]
				\item White microcrystalline powder.
				\item \emph{Markedly hygroscopic} (forms clumps).
			\end{itemize}
		\end{column}
		\begin{column}{0.48\textwidth}
			\textbf{Thermal Properties}
			\begin{itemize}[<+->]
				\item Low melting point.
				\item \alert{No carbonization}.
				\item \textbf{Conclusion:} Inorganic Compound.
			\end{itemize}
		\end{column}
	\end{columns}
\end{frame}

% --- NEW SLIDE: Photos of Dark Spots ---
\begin{frame}{Visual Evidence: Dermal Reaction}
	\centering
	\textbf{Photoreduction of the sample on organic tissue}
	\vspace{1em}

	\begin{columns}
		\begin{column}{0.48\textwidth}
			\centering
			\includegraphics[width=\linewidth, height=0.6\textheight,
			keepaspectratio]{./images/sebastian-hand.jpg}
			\vspace{0.5em}
			{\footnotesize \emph{\\ Figure 1: Sebastian's Hand}}
		\end{column}
		\begin{column}{0.48\textwidth}
			\centering
			\includegraphics[width=\linewidth, height=0.6\textheight,
			keepaspectratio]{./images/adriano-finger.jpg}
			\vspace{0.5em}
			{\footnotesize \emph{ \\ Figure 2: Adriano's Finger}}
		\end{column}
	\end{columns}
\end{frame}

% --- The XRF Result ---
\begin{frame}{The Elemental ``Fingerprint''}
	X-Ray Fluorescence (XRF) spectroscopy provided the most conclusive data:

	\vspace{1em}
	\centering
	\begin{tikzpicture}
		\node[draw=mDarkTeal, fill=mDarkTeal, text=white, rounded corners, inner
			sep=15pt, font=\Huge\bfseries] (ag) {Ag};
		\node[right=of ag, align=left, font=\Large] {Primary
			Constituent:\\\textbf{SILVER}};
	\end{tikzpicture}

	\pause
	\vspace{1em}
	\raggedright
	\small This explains the skin stains: $\ce{Ag+ -> Ag^0}$ (Black metallic silver).
\end{frame}

\begin{frame}{XRF Spectrum Analysis}
	\centering
	\includegraphics[width=0.9\textwidth, height=0.75\textheight,
	keepaspectratio]{./images/xrf-spectrum.pdf}

	\vspace{0.5em}
	{\small Spectrum confirming characteristic Silver (Ag) emission lines.}
\end{frame}

% --- The Solubility Paradox ---
\begin{frame}{The Solubility Paradox}
	Solubility results presented an apparent contradiction:

	\vspace{1em}
	\begin{columns}
		\begin{column}{0.5\textwidth}
			\pause
			\begin{block}{Deionized Water}
				\centering
				\textbf{High Solubility}\\
				Clear, colorless solution.
			\end{block}
		\end{column}
		\begin{column}{0.5\textwidth}
			\pause
			\begin{block}{Sulfuric Acid (\ce{H2SO4})}
				\centering
				\alert{\textbf{``Insolubility''}}\\
				Formation of a white solid.
			\end{block}
		\end{column}
	\end{columns}

	\vspace{1.5em}
	\centering
	\emph{How can it be soluble in water but insoluble in dilute acid?}
\end{frame}

% --- Solving the Puzzle ---
\section{Analysis}
\begin{frame}{Solving the Paradox}
	It is not physical insolubility; it is a \textbf{Precipitation Reaction}.

	\vspace{1em}
	Upon dissolving, the sample releases $\ce{Ag+}$ ions, which encounter sulfate $\ce{SO4^2-}$ ions in the acid.

	\vspace{1em}
	\begin{center}
		\large
		$\ce{2Ag+ (aq) + \alert{SO4^2- (aq)} -> \alert{Ag2SO4 (s)} v}$
	\end{center}

	\vspace{1em}
	\textbf{Conclusion:} The unknown is a water-soluble silver salt.
\end{frame}

% --- The 3 Candidates ---
\section{Candidates}
\begin{frame}[standout]
	The 3 Main Candidates
\end{frame}

\begin{frame}{Candidate 1: Silver Nitrate (\ce{AgNO3})}
	\textbf{Probability: High}

	\begin{itemize}
		\item[$\checkmark$] \textbf{Solubility:} Extremely high in water
			(\qty{2560}{\gram\per\litre}).
		\item[$\checkmark$] \textbf{Skin:} Causes characteristic black stains (chemical burn via reduction).
		\item[$\checkmark$] \textbf{Thermal:} Low melting point (\qty{212}{\celsius}).
		\item[$\checkmark$] \textbf{Chemical:} Reacts with \ce{H2SO4} to precipitate sulfate.
	\end{itemize}

	\footnotesize \emph{This is the most common soluble silver salt in laboratories.}
\end{frame}

\begin{frame}{Candidate 2: Silver Fluoride (\ce{AgF})}
	\textbf{Probability: Medium}

	\begin{itemize}
		\item[$\checkmark$] \textbf{Hygroscopicity:} Matches perfectly (\ce{AgF} is deliquescent, absorbs water voraciously).
		\item[$\checkmark$] \textbf{Solubility:} The only water-soluble silver halide.
		\item[\texttimes] \textbf{Rarity:} Less common than nitrate.
		\item[\texttimes] \textbf{Appearance:} Often has a yellowish tint (though can be pure white).
	\end{itemize}
\end{frame}

\begin{frame}{Candidate 3: Silver Perchlorate (\ce{AgClO4})}
	\textbf{Probability: Low/Medium}

	\begin{itemize}
		\item[$\checkmark$] \textbf{Solubility:} Extreme solubility
			(\qty{5570}{\gram\per\litre}), even higher than nitrate.
		\item[$\checkmark$] \textbf{Inorganic:} Consistent with lack of carbonization.
		\item[\texttimes] \textbf{Risk:} Potentially explosive under certain conditions (organic shock), requiring special handling not reported.
	\end{itemize}
\end{frame}

% --- Next Steps ---
\section{Future Work}
\begin{frame}{Next Steps}
	To distinguish between \ce{NO3-}, \ce{F-}, and \ce{ClO4-}:

	\begin{enumerate}
		\item \textbf{Brown Ring Test:} Specific confirmation for Nitrates.
		\item \textbf{X-Ray Diffraction (XRD):} Definitive identification of the crystalline phase.
		\item \textbf{Raman Spectroscopy:} Differentiation of anion vibrational modes.
	\end{enumerate}
\end{frame}

\begin{frame}[plain]
	\centering
	\vspace{1cm}
	{\Huge \textbf{Thank you!}}\\[1cm]
	{\large Questions or Comments?}
\end{frame}

\begin{frame}
	\titlepage
\end{frame}

\end{document}
