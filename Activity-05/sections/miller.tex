\section{Convenciones de Notación en Índices de Miller}

La notación de los índices de Miller, $(hkl)$, está sujeta a
convenciones estrictas que son cruciales para su correcta
interpretación.

\subsection{Planos Equivalentes por Inversión}
Un plano cristalográfico se define por sus interceptos;
los índices $(hkl)$ representan una familia de planos paralelos
infinitos. Por convención, un conjunto de índices y su negativo
directo, $(-h, -k, -l)$, se consideran físicamente equivalentes.

Consideremos los índices $(012)$ y $(0\bar{1}\bar{2})$, donde la
barra superior denota un valor negativo (e.g., $-1$).
El plano $(012)$ intercepta los ejes en $(\infty, b, c/2)$.
El plano $(0\bar{1}\bar{2})$ intercepta los ejes en
$(\infty, -b, -c/2)$.

Estos dos planos son paralelos y están separados por la misma
distancia $d_{hkl}$. Geométricamente, describen el mismo conjunto
de planos en el cristal, aunque sus vectores normales apunten en
direcciones opuestas. En el contexto de la difracción, dado que
la Ley de Bragg depende de $d_{hkl}$, ambos planos son
indistinguibles y se consideran idénticos. Pertenecen a la misma
familia de planos $\{012\}$.

\subsection{Múltiplos de Índices y Órdenes de Difracción}
Una situación conceptualmente diferente surge con los índices que
son múltiplos enteros, como $(123)$ y $(246)$.

Por definición geométrica, los índices de Miller $(hkl)$ deben
ser el conjunto de enteros más pequeños que mantienen la misma
proporción (es decir, deben ser coprimos). El plano $(123)$
intercepta los ejes en $(a, b/2, c/3)$.
La notación $(246)$ es, por esta definición, inválida, ya que
puede reducirse a $(123)$ al dividir por un factor común de 2.
Representaría un plano con interceptos $(a/2, b/4, c/6)$, que
es un plano paralelo a $(123)$ pero con un espaciado diferente.

Sin embargo, la notación $(nh, nk, nl)$, como $(246)$,
adquiere un significado físico fundamental en el contexto de la
difracción de rayos X. Se utiliza universalmente como una
taquigrafía para referirse a los órdenes de difracción
superiores ($n > 1$) de la Ley de Bragg ($n\lambda = 2d_{hkl} \sin\theta$).

El pico $(123)$ corresponde al primer orden ($n=1$) de
difracción de los planos $(123)$. El pico $(246)$ se refiere
específicamente al segundo orden ($n=2$) de difracción de
esos mismos planos $(123)$.

Esta notación es matemáticamente consistente. El espaciado
para un plano ficticio $(246)$ sería $d_{246} = d_{123} / 2$.
Al sustituir esto en la Ley de Bragg para $n=1$, se obtiene:
$$ 1 \lambda = 2 d_{246} \sin(\theta_{(246)})
   = 2 \left(\frac{d_{123}}{2}\right) \sin(\theta_{(246)}) $$
$$ \lambda = d_{123} \sin(\theta_{(246)}) $$
Si comparamos esto con la condición de Bragg para el segundo
orden de $(123)$:
$$ 2 \lambda = 2 d_{123} \sin(\theta_{(123), n=2}) $$
$$ \lambda = d_{123} \sin(\theta_{(123), n=2}) $$
Vemos que $\theta_{(246)} = \theta_{(123), n=2}$. Por lo tanto,
$(123)$ y $(246)$ no son iguales: representan el primer y
segundo orden de difracción de la misma familia de planos,
y aparecen como picos distintos en un difractograma.
