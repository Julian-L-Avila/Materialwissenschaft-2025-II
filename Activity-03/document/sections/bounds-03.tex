\section{Actividad 03 -- Enlaces Atómicos II}

\begin{problem}[Propiedades según tipo de Enlace]
  ¿Qué propiedades fisicoquímicas se originan en cada tipo de enlace
  atómico y cuáles son direccionales?
\end{problem}

\subsection{Enlace Covalente (Direccional)}

Este enlace se origina en la interacción de intercambio, un efecto
puramente cuántico que disminuye la energía de un sistema de electrones
cuando sus funciones de onda se solapan constructivamente. Es
altamente direccional porque la maximización de este
solapamiento, y por ende la minimización de la energía, depende
críticamente de la geometría y orientación de los orbitales atómicos
de valencia ($s, p, d, f$). La hibridación de orbitales, como la $sp^3$
en el carbono, genera orbitales orientados en ángulos específicos (e.g.,
\ang{109.5} en una geometría tetraédrica), fijando la estructura
molecular y cristalina.

\paragraph{Propiedades Fisicoquímicas.}
\begin{itemize}
  \item \textbf{Mecánicas:} Si los enlaces forman una red
    tridimensional (e.g., diamante, \ce{SiO2}), el material es
    extremadamente duro, con un módulo de compresibilidad elevado. La
    energía para deformar el cristal es la energía necesaria para
    distorsionar estos fuertes enlaces localizados.
  \item \textbf{Electrónicas:} Generalmente son aislantes o
    semiconductores. En el lenguaje de la teoría de bandas, los
    orbitales moleculares enlazantes forman una banda de valencia
    llena, mientras que los antienlazantes forman una banda de
    conducción vacía. La fuerte localización de los electrones crea
    una gran brecha de energía ($E_g$) entre ambas, impidiendo la
    conductividad.
  \item \textbf{Solubilidad:} Suelen ser insolubles en disolventes
    polares. Es energéticamente desfavorable romper los fuertes enlaces
    de hidrógeno de un solvente como el agua para solvatar una molécula
    que solo puede interactuar mediante débiles fuerzas de van der
    Waals.
\end{itemize}

\subsection{Enlace Iónico (No Direccional)}

Se produce por la transferencia de electrones, creando cationes y
aniones que interactúan mediante la fuerza de Coulomb. Esta fuerza es
central e isotrópica ($F \propto r^{-2}$), por lo que es no
direccional. Un ion atrae a todos los iones de carga opuesta en su
vecindad con igual intensidad. La estructura cristalina resultante es
aquella que maximiza el empaquetamiento y la energía de cohesión.

\paragraph{Propiedades Fisicoquímicas.}
\begin{itemize}
  \item \textbf{Mecánicas:} Son duros pero frágiles. La dureza
    proviene de la fuerte atracción electrostática. La fragilidad se
    debe a que un deslizamiento de planos atómicos (movimiento de una
    dislocación) enfrentaría iones de la misma carga, generando una
    repulsión que fractura el cristal.
  \item \textbf{Eléctricas:} Son aislantes en estado sólido,
    pues los iones están fijos en la red. Sin embargo, en estado
    fundido o en disolución, los iones ganan movilidad y se convierten
    en excelentes conductores iónicos.
  \item \textbf{Térmicas:} Poseen puntos de fusión muy elevados, ya que
    se requiere una gran energía térmica para superar la energía
    de red del cristal.
\end{itemize}

\subsection{Enlace Metálico (No Direccional)}

Se describe como una red de cationes inmersa en un ``gas'' o ``mar'' de
electrones de valencia deslocalizados. Cuánticamente, los electrones de
valencia ocupan estados deslocalizados (ondas de Bloch) que se extienden
por todo el cristal. La atracción entre el gas de electrones (negativo)
y los núcleos iónicos (positivos) es de naturaleza electrostática y, por
tanto, no direccional.

\paragraph{Propiedades Fisicoquímicas.}
\begin{itemize}
  \item \textbf{Conductividad:} Excelente conductividad eléctrica y
    térmica. Los electrones en estados cercanos a la energía de Fermi
    son libres de acelerarse en un campo eléctrico. Estos mismos
    electrones móviles son portadores muy eficientes de energía
    térmica, una correlación descrita por la Ley de Wiedemann-Franz.
  \item \textbf{Mecánicas:} Son dúctiles y maleables. La
    naturaleza deslocalizada y no direccional del enlace permite que
    los planos atómicos se deslicen unos sobre otros sin que se rompa
    la cohesión del material. El ``mar'' de electrones se adapta a la
    nueva configuración.
  \item \textbf{Ópticas:} Son opacos y lustrosos. El gas de electrones
    libres se comporta como un plasma. Por debajo de una frecuencia
    característica (la frecuencia de plasma, $\omega_p$),
    la luz es eficientemente reflejada. Por encima de $\omega_p$ (en
    el UV para muchos metales), se vuelven transparentes.
\end{itemize}

\subsection{Enlaces Secundarios (Intermoleculares)}

\subsubsection{Enlace por Puente de Hidrógeno (Direccional)}

Es una interacción fuerte que ocurre cuando un \ce{H} unido a un átomo muy
electronegativo (\ce{N}, \ce{O}, \ce{F}; el ``donador'') es atraído por otro
átomo electronegativo (el ``aceptor''). Es altamente direccional porque no es
puramente electrostático; posee un carácter covalente parcial, modelado como una
interacción entre el par solitario del aceptor y el orbital antienlazante
$\sigma^*$ del enlace donador-\ce{H}. La fuerza es máxima con una alineación
donador-\ce{H}-aceptor de $\approx$\ang{180}.

\paragraph{Propiedades Fisicoquímicas.}
\begin{itemize}
  \item \textbf{Térmicas:} Causa un aumento anómalo en los puntos de
    ebullición. La energía extra para romper estos enlaces explica por
    qué el \ce{H2O} (P.E. \qty{100}{\celsius}) es líquido, mientras que
    el \ce{H2S} (P.E. \qty{-60}{\celsius}) es un gas.
  \item \textbf{Estructurales:} Define la doble hélice del ADN y las
    estructuras secundarias de proteínas. En el hielo (fase Ih), fuerza
    una estructura hexagonal abierta que es menos densa que el agua
    líquida.
\end{itemize}

\subsubsection{Interacciones de Van der Waals (No Direccionales)}

Son fuerzas universales pero débiles, originadas por correlaciones en
las fluctuaciones de las nubes electrónicas. Se componen de:
\begin{enumerate}
  \item Interacciones dipolo-dipolo (Keesom).
  \item Interacciones dipolo-dipolo inducido (Debye).
  \item Interacciones dipolo inducido-dipolo inducido (dispersión de
    London), que son dominantes y de origen cuántico ($U \propto -r^{-6}$).
\end{enumerate}
Aunque algunas componentes tienen dependencia angular, en fases
condensadas se promedian, resultando en una cohesión no
direccional que favorece el empaquetamiento denso.

\paragraph{Propiedades Fisicoquímicas.}
\begin{itemize}
  \item \textbf{Térmicas:} Los sólidos moleculares (yodo, hielo seco)
    tienen puntos de fusión y ebullición muy bajos, ya que la energía
    térmica ($k_B T$) supera fácilmente estas débiles interacciones.
  \item \textbf{Mecánicas:} Son materiales blandos y compresibles, con
    baja resistencia mecánica.
\end{itemize}

\begin{problem}[Grafito y Diamante]
  El grafito es buen conductor eléctrico (solo en el plano de sus
  láminas), mientras que el diamante es un aislante. Explique esta
  diferencia en función de sus enlaces covalentes y la estructura
  electrónica.
\end{problem}

La abismal diferencia en la conductividad eléctrica entre estos dos
alótropos del carbono es una manifestación directa de sus distintas
estructuras de bandas electrónicas, que a su vez son consecuencia de
la hibridación de los orbitales de valencia.

\subsection{Diamante: Aislante de Banda Prohibida Ancha
(\texorpdfstring{$sp^3$}{sp3})}

En la estructura del diamante, cada átomo de carbono presenta una
hibridación $sp^3$, formando cuatro orbitales híbridos idénticos
orientados tetraédricamente (\ang{109.5}). Cada uno de estos orbitales
forma un enlace covalente simple (enlace $\sigma$) con un átomo vecino,
creando una red tridimensional rígida y continua.

Desde la perspectiva de la física del estado sólido, la totalidad de
los cuatro electrones de valencia de cada átomo está fuertemente
localizada en estos enlaces $\sigma$. Estos estados electrónicos
enlazantes constituyen la banda de valencia, la cual está
completamente llena. Los estados antienlazantes ($\sigma^*$) forman la
banda de conducción, que está completamente vacía.

La clave es que entre ambas bandas existe una brecha de energía
prohibida (band gap, $E_g$) muy grande, del orden de
\qty{5.5}{\electronvolt}. A temperatura ambiente, la energía térmica
disponible ($k_B T \approx \qty{0.025}{\electronvolt}$) es
insignificante en comparación con $E_g$, por lo que es prácticamente
imposible excitar un electrón de la banda de valencia a la de
conducción. Sin portadores de carga móviles, el diamante es uno de los
mejores aislantes eléctricos conocidos.

\subsection{Grafito: Semimetal Anisotrópico (\texorpdfstring{$sp^2$}{sp2})}

En el grafito, cada átomo de carbono utiliza una hibridación
$sp^2$, formando tres enlaces $\sigma$ con sus vecinos en un plano, lo que
da lugar a una estructura de láminas hexagonales. El cuarto electrón de
valencia reside en un orbital $p$ no hibridado ($p_z$),
perpendicular a dicho plano.

Los orbitales $p_z$ de todos los átomos en una lámina se solapan
lateralmente, formando bandas $\pi$ (enlazante) y $\pi^*$
(antienlazante) deslocalizadas a lo largo de todo el plano
bidimensional. Aquí reside la diferencia fundamental con el diamante:
en el grafito, la banda $\pi$ (de valencia) y la banda $\pi^*$ (de
conducción) no están separadas por una brecha de energía. De hecho,
se tocan en puntos específicos (los puntos de Dirac) de la
zona de Brillouin hexagonal.

Esto significa que la energía de Fermi ($E_F$) se encuentra
precisamente en la energía donde estas bandas se degeneran. El material
no tiene brecha energética ($E_g=0$), lo que lo clasifica como un
semimetal. Con una densidad de estados finita en la energía de
Fermi, una cantidad infinitesimal de energía es suficiente para excitar
electrones a estados conductores. Estos electrones deslocalizados en el
sistema $\pi$ se mueven libremente a lo largo de las láminas,
otorgando al grafito una alta conductividad eléctrica en el plano.

Entre las láminas, la unión se debe a débiles fuerzas de Van der Waals,
lo que resulta en un muy mal solapamiento de las funciones de onda en
la dirección perpendicular. Esto impide el movimiento de electrones
entre capas, haciendo que el grafito sea un aislante en esa dirección y,
por tanto, un material con una alta anisotropía en su conductividad eléctrica.

\begin{problem}[Potencial Lennard-Jones]
    Interprete el gráfico del Potencial Lennard-Jones presentado en la
    \cref{fig:lennar-jones}.
    \begin{figure}[htbp!]
        \centering
        \includegraphics[width=0.8\linewidth]{./images/lennard.pdf}
        \caption{Potencial Lennard-Jones.}
        \label{fig:lennar-jones}
    \end{figure}
\end{problem}

El potencial de Lennard-Jones es un modelo fenomenológico fundamental en
la física de la materia condensada. Describe la energía potencial de
interacción $V(r)$ entre un par de átomos o moléculas neutras y
esféricamente simétricas en función de su separación $r$. Su forma
funcional captura la física esencial de las interacciones no enlazantes
y sirve como pilar en la simulación de dinámica molecular.

La expresión matemática del potencial es:
$$ V(r) = 4\epsilon \left[ \left(\frac{\sigma}{r}\right)^{12} -
\left(\frac{\sigma}{r}\right)^6 \right] $$
donde $\epsilon$ es la profundidad del pozo de potencial y $\sigma$ es
la distancia a la cual el potencial es nulo.

\subsection{Interpretación de los Términos Físicos}

El potencial se compone de dos términos que representan un balance entre
atracción y repulsión.

\paragraph{Término Atractivo ($ -\sigma^6 r^{-6} $).}
Dominante a distancias intermedias y largas. Este término modela la
fuerza de dispersión de London, una interacción de van der
Waals de origen puramente cuántico. Surge de las correlaciones entre
fluctuaciones cuánticas en la distribución de carga de los átomos. Un
dipolo instantáneo en un átomo induce un dipolo en el vecino, y la
interacción entre estos dipolos fluctuantes resulta en una atracción.
La dependencia $r^{-6}$ es el término de orden más bajo derivado de la
teoría de perturbaciones de segundo orden para la interacción
dipolo-dipolo en el régimen no retardado.

\paragraph{Término Repulsivo ($ + \sigma^{12}r^{-12} $).}
Dominante a distancias muy cortas ($r \ll \sigma$). Este término
representa la repulsión de Pauli. Cuando las nubes
electrónicas de dos átomos comienzan a solaparse, el Principio de
Exclusión de Pauli obliga a los electrones a ocupar orbitales de mayor
energía (antienlazantes), resultando en un drástico aumento de la
energía total. La elección del exponente 12, aunque físicamente menos
rigurosa que un potencial exponencial (tipo Buckingham), es una
aproximación matemáticamente conveniente que modela una ``pared'' de
potencial extremadamente abrupta.

\subsection{Análisis de la Gráfica y sus Parámetros}

\paragraph{Comportamiento Asintótico.}
Para $r \to \infty$, $V(r) \to 0$, lo que refleja correctamente la
ausencia de interacción a grandes distancias.

\paragraph{Punto de Cruce ($ r = \sigma $).}
A esta distancia, $V(\sigma) = 4\epsilon [1 - 1] = 0$. El parámetro
$\sigma$ se interpreta como el diámetro de colisión efectivo
de las partículas. Para $r < \sigma$, la interacción es repulsiva
($V > 0$), y para $r > \sigma$, es atractiva ($V < 0$).

\paragraph{Mínimo del Potencial y Distancia de Equilibrio.}
La posición de equilibrio, $r_m$, corresponde al mínimo de $V(r)$,
donde la fuerza neta es nula, $F(r_m) = - \mdif{r} V |_{r_m} = 0$.
Derivando el potencial e igualando a cero:
$$ \mdif{r}{V} = 4\epsilon \left[ -12 \sigma^{12}{r^{-13}} +
6\sigma^6{r^{-7}} \right] = 0 \implies 2\sigma^6 = r_m^6 $$
se obtiene $r_m = 2^{1/6}\sigma \approx 1.122\sigma$. Esta es la
distancia de separación más estable.

\paragraph{Profundidad del Pozo ($-\epsilon$).}
El valor del potencial en el mínimo es $V(r_m) = -\epsilon$. Este valor
representa la energía de cohesión o la energía de enlace del
par. Es la energía requerida para separar las dos partículas desde su
posición de equilibrio hasta el infinito.

\paragraph{Vibraciones y Estabilidad.}
Cerca del mínimo, el potencial puede aproximarse por una parábola
(desarrollo de Taylor de segundo orden), análogo a un oscilador
armónico. La ``constante de resorte'' efectiva del enlace, $k$, está dada
por la curvatura del potencial en el mínimo: $k = \mdif{r}^2{V}|_{r_m}$.
Esta curvatura determina la frecuencia de vibración de las partículas
alrededor de su posición de equilibrio.
