\documentclass[11pt, a4paper]{article}

\usepackage[utf8]{inputenc}
\usepackage[T1]{fontenc}
\usepackage[spanish]{babel}
\usepackage{amsmath}
\usepackage{amssymb}
\usepackage{graphicx}
\usepackage[margin=1in]{geometry}
\usepackage{hyperref}
\usepackage[version=4]{mhchem}
\usepackage{siunitx}

\hypersetup{
  colorlinks=true,
  linkcolor=black,
  citecolor=black,
  filecolor=black,
  urlcolor=blue,
}

\title{
  \textbf{Resumen: Materiales Cuánticos para una Inteligencia Artificial Sostenible}
}
\author{Julian L. Avila-Martinez}
\date{19 de Septiembre 2025}

\begin{document}

\maketitle
\thispagestyle{empty}

\begin{center}
  \textbf{Ponente:} Paula Giraldo-Gallo \\
  \textbf{Afiliación:} Grupo de Materiales Cuánticos, Universidad de los Andes
\end{center}

\vspace{0.5cm}

\section*{El Desafío: Almacenamiento de Datos Insostenible}

El crecimiento exponencial de los datos, actualmente estimado en más de
$10^{23}$~bytes, presenta un desafío de sostenibilidad significativo. Estos
datos se almacenan físicamente en centros de datos que colectivamente ocupan un
área comparable a la de un país pequeño y consumen aproximadamente el 1\% de la
energía total del mundo. La tecnología central, la memoria ferromagnética,
depende de la inducción de corrientes eléctricas para escribir y leer datos.
Este proceso es fundamentalmente ineficiente debido a dos factores principales:
\begin{itemize}
  \item \textbf{Ineficiencia Energética:} Se disipa una cantidad significativa
    de energía en forma de calentamiento por efecto Joule, lo que requiere
    sistemas de refrigeración masivos que a su vez consumen enormes recursos (e.
    g. un estimado de 2 litros de agua por cada imagen generada por IA).
  \item \textbf{Límites Físicos de Escalabilidad:} Los bits ferromagnéticos no
    pueden reducirse por debajo de aproximadamente 16~\si{\nano\metre} sin volverse
    térmicamente inestables debido al efecto superparamagnético, lo que
    provoca la pérdida de datos. Esto impone un límite estricto a la densidad de
    almacenamiento.
\end{itemize}

\section*{La Solución: Materiales Cuánticos Multiferroicos}

La charla propone un cambio de paradigma desde los materiales clásicos hacia los
materiales cuánticos, aquellos cuyas propiedades están dominadas por
fenómenos de la mecánica cuántica y no pueden ser descritos por aproximaciones
clásicas. Dentro de esta clase, los materiales multiferroicos ofrecen
una solución directa a la crisis energética del almacenamiento de datos.

A diferencia de los ferromagnetos convencionales que requieren una corriente
eléctrica para cambiar los estados magnéticos, los multiferroicos exhiben un
acoplamiento entre sus parámetros de orden magnético y eléctrico. Esto permite
que la polarización magnética (el bit de datos) sea controlada por un
campo eléctrico aplicado (una diferencia de potencial) en lugar de una
corriente. Este cambio en el mecanismo puede reducir la energía necesaria para
escribir datos en aproximadamente cuatro órdenes de magnitud.

\section*{Un Novedoso Material Multiferroico 2D a Temperatura Ambiente}

Para abordar tanto el consumo de energía como la densidad espacial, el grupo ha
desarrollado un novedoso material multiferroico 2D basado en un
Dicalcogenuro de Metal de Transición (TMD). Los TMDs tienen una fórmula
química general de \ce{MX2}, donde M es un metal de transición y X es un
calcógeno. Se caracterizan por fuertes enlaces covalentes dentro de un plano 2D
y débiles fuerzas de van der Waals entre planos, lo que permite su exfoliación
en capas individuales.

El material específico desarrollado es una aleación compleja con la fórmula
\ce{W(Se_{1-x}Te_x)_{2-\delta}}:
\begin{itemize}
  \item \textbf{M (Metal de Transición):} Wolframio (W)
  \item \textbf{X (Calcógeno):} Selenio (Se)
  \item \textbf{Dopante:} Telurio (Te) se sustituye por Selenio para ajustar las
    propiedades del material.
  \item \textbf{Defecto:} Se introduce intencionadamente una vacancia ($\delta$)
    en los sitios del calcógeno.
\end{itemize}
El material exhibe varias propiedades clave diseñadas:
\begin{itemize}
  \item \textbf{Ferromagnetismo por Vacancias:} La introducción de vacancias de
    selenio ($\delta$) es el principal impulsor para inducir ferromagnetismo en
    el material. Una mayor concentración de vacancias resulta en una respuesta
    ferromagnética más fuerte.
  \item \textbf{Piezoelectricidad Sintonizable:} El material también es
    piezoeléctrico. La magnitud de la respuesta piezoeléctrica se correlaciona
    directamente con la concentración de Telurio (x), lo que permite ajustar
    esta propiedad.
\end{itemize}

\section*{Conclusión y Relevancia}

La investigación culmina en la creación de lo que se reporta como el
primer material multiferroico 2D que es funcional a temperatura
ambiente. Esto representa un avance significativo, ya que los multiferroicos 2D
anteriores solo eran estables a temperaturas criogénicas, lo que los hacía
imprácticos para aplicaciones de consumo o en centros de datos. El desarrollo de
una memoria magnética controlada por voltaje a temperatura ambiente en un
formato 2D aborda directamente los desafíos críticos del consumo de energía y la
densidad de almacenamiento. Con una patente en trámite, este material podría
allanar el camino para una nueva generación de dispositivos lógicos y de memoria
de ultra alta densidad y bajo consumo, esenciales para el escalado sostenible de
la inteligencia artificial.

\end{document}
