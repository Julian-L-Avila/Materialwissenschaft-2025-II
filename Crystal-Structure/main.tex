% ==============================================================================
% PREAMBLE
% ==============================================================================
\documentclass[11pt]{scrartcl}

% --- Font and Encoding ---
\usepackage{fontspec}

% --- Typographic Enhancements ---
\usepackage{microtype} % Mejora la justificación y el espaciado
\usepackage[spanish, bidi=basic, provide=*]{babel} % Configuración de idioma
\usepackage{csquotes}

% --- Mathematics and Science ---
\usepackage{amsmath}    % Herramientas matemáticas principales de AMS
\usepackage[version=4]{mhchem}     % Para fórmulas químicas como \ce{AgF}
\usepackage{siunitx}    % Composición tipográfica profesional de unidades y números

% --- Tables and Figures ---
\usepackage{booktabs}   % Para tablas de calidad profesional (\toprule, etc.)
\usepackage{graphicx}
\usepackage{caption}    % Personalización de leyendas

% --- Page Layout and Metadata ---
\usepackage[a4paper, margin=1in]{geometry} % Márgenes adecuados
\usepackage{authblk}    % Para afiliaciones de autor

% --- Bibliography ---
\usepackage[backend=biber, style=numeric-comp, sorting=none]{biblatex}
\addbibresource{./crystal.bib}

% --- Hyperlinks ---
\usepackage[colorlinks=true,
linkcolor=blue,
urlcolor=blue,
citecolor=blue]{hyperref}

% ==============================================================================
% DOCUMENT METADATA
% ==============================================================================
\title{Análisis Estructural del \ce{AgF}}
\author{Julian L. Avila-Martinez}
\affil{Programa de F\'isica \\ Universidad Distrital Francisco Jos\'e de Caldas}
\date{\today}


% ==============================================================================
% BEGIN DOCUMENT
% ==============================================================================
\begin{document}

\maketitle
\thispagestyle{empty}

% ==============================================================================
% ABSTRACT
% ==============================================================================
\begin{abstract}
  \noindent % Sin sangría para el resumen
  El fluoruro de plata(I) (\ce{AgF}) es un compuesto inorgánico binario que sirve como
  ejemplo canónico para el estudio de conceptos fundamentales en la física
  del estado sólido y la ciencia de los materiales.
A diferencia de otros
  monohaluros de plata, el \ce{AgF} cristaliza en la estructura de sal de
  roca (rock salt), altamente simétrica, lo que lo convierte en un sistema ideal
  para investigar principios cristalográficos, operaciones de simetría y
  la naturaleza del enlace iónico.
Este informe presenta un
  análisis del \ce{AgF}, comenzando con una descripción
  de su estructura cristalina y progresando hacia un examen de
  sus simetrías subyacentes.
La naturaleza y la fuerza del enlace
  interatómico se cuantificarán mediante un análisis de la
  electronegatividad y la energía de red.
\end{abstract}

% ==============================================================================
% SECTION 1: CRYSTAL STRUCTURE
% ==============================================================================
\section{La Estructura Cristalográfica del \ce{AgF}}

\subsection{Sistema Cristalino y el Prototipo de Sal de Roca}
\begin{figure}[htbp!]
  \centering
  \includegraphics[width=0.25\linewidth]{./images/SilverI-fluoride-3D-ionic.png}
  \caption{Estructura Cristalina del \ce{AgF} usando Ovito.}
\label{fig:crystal}
\end{figure}

A temperatura y presión ambiente, el fluoruro de plata(I) adopta un
sistema cristalino cúbico, cristalizando en el tipo de estructura de sal
de roca (NaCl)~\cite{Seidell2015}.
Esta estructura es una de las
  disposiciones más comunes para compuestos binarios con una relación
  estequiométrica 1:1.
Puede ser deconstruida conceptualmente en dos
  componentes primarios: una red (lattice) y una base.
La red, que
  define la disposición periódica de puntos en el espacio, es cúbica
  centrada en las caras (FCC).
La base, que especifica los átomos
  asociados con cada punto de la red, consiste en dos iones: un catión
  de plata ($\ce{Ag+}$) y un anión de fluoruro ($\ce{F-}$).
La estructura de sal de roca se describe con mayor precisión como dos
  subredes FCC interpenetrantes.
Una subred está compuesta enteramente por
  cationes $\ce{Ag+}$, y la otra está compuesta por aniones $\ce{F-}$.
Estas
  dos subredes están desplazadas una respecto a la otra por un vector
  correspondiente a la mitad de la diagonal del cuerpo de la celda unitaria, tal como
  $(a/2, a/2, a/2)$, donde $a$ es el parámetro de red.
\subsection{Parámetros de Celda Unitaria y Posiciones Atómicas}
Las dimensiones precisas de la celda unitaria del \ce{AgF} han sido determinadas
con alta precisión utilizando difracción de rayos X de monocristal a
baja temperatura.
A \SI{100}{K}, el parámetro de red es
$a = \SI{4.92171(14)}{\angstrom}$~\cite{Lozinsek2023}. Este valor puede
compararse con una medición más antigua a temperatura ambiente de
$a = \SI{4.936(1)}{\angstrom}$, lo que ilustra el fenómeno de
contracción térmica~\cite{Lozinsek2023}.
La disminución observada en el parámetro de red a temperaturas más
bajas es una manifestación macroscópica directa de la anarmonicidad
del potencial interatómico.
En un pozo de potencial perfectamente
  armónico (parabólico), los átomos oscilarían simétricamente alrededor
  de su posición de equilibrio, sin resultar en ningún cambio en la
  distancia interatómica promedio con la temperatura.
Sin embargo, los
  potenciales interatómicos reales son asimétricos: más pronunciados en
  el lado compresivo y más suaves en el lado expansivo.
A medida que
  aumenta la energía térmica, los átomos oscilan con mayor amplitud, y
  esta asimetría provoca que la distancia de separación promedio
  aumente, lo que conduce a la expansión térmica.
Los datos para
el \ce{AgF} proporcionan un claro ejemplo experimental de este principio
fundamental del estado sólido.
La celda unitaria cúbica convencional del \ce{AgF} contiene cuatro
  unidades fórmula, denotadas como $Z=4$~\cite{Lozinsek2023}.
Esto puede
  verificarse sumando las contribuciones iónicas dentro de la celda:
  cada una de las dos subredes FCC interpenetrantes contiene cuatro
  iones.
Para una única red FCC, esto se calcula a partir de los 8
  sitios de las esquinas (cada uno compartido por 8 celdas,
  contribuyendo $8 \times 1/8 = 1$ ion) y los 6 sitios centrados en las
  caras (cada uno compartido por 2 celdas, contribuyendo
  $6 \times 1/2 = 3$ iones), para un total de 4 iones por celda.
Dentro del marco cristalográfico estándar para el grupo espacial
$Fm\overline{3}m$, las posiciones atómicas se definen mediante
coordenadas de Wyckoff.
Los iones $\ce{Ag+}$ ocupan los sitios 4a en
  coordenadas fraccionarias (0, 0, 0) y posiciones equivalentes
  generadas por las traslaciones de centrado FCC, mientras que los iones
  $\ce{F-}$ ocupan los sitios 4b en (1/2, 1/2, 1/2) y sus
  posiciones equivalentes FCC.
\subsection{Distancias Interatómicas y Coordinación}
La disposición geométrica de la estructura de sal de roca dicta una
relación simple entre el parámetro de red y la distancia al vecino más
cercano, que corresponde a la longitud del enlace \ce{Ag}-F ($r_0$).
Esta
distancia es exactamente la mitad del parámetro de red: $r_0 = a/2$.
Usando
la constante de red precisa a baja temperatura, la longitud de enlace
calculada es:
\[
  r_0 = \frac{\SI{4.92171}{\angstrom}}{2} = \SI{2.460855}{\angstrom}
\]
Este valor calculado muestra una concordancia excepcional con la
longitud de enlace \ce{Ag}-F medida experimentalmente de
$\SI{2.46085(7)}{\angstrom}$ del mismo estudio a baja
temperatura~\cite{Lozinsek2023}.
Esta consistencia interna proporciona una
validación poderosa para el modelo estructural de sal de roca,
demostrando que la dimensión macroscópica de la celda unitaria está
directa y predeciblemente vinculada a la disposición atómica
microscópica.
El entorno de coordinación es una característica definitoria de la
estructura. Cada ion $\ce{Ag+}$ está rodeado por seis iones $\ce{F-}$
vecinos más cercanos ubicados en los vértices de un octaedro regular y,
recíprocamente, cada ion $\ce{F-}$ está coordinado octaédricamente por
seis iones $\ce{Ag+}$~\cite{Sanderson1960}.
Esto da como resultado una
coordinación 6:6 para el compuesto. La distancia al segundo vecino más
cercano, que es la separación entre iones del mismo tipo (p. ej.,
\ce{Ag}-\ce{Ag} o \ce{F}-\ce{F}), viene dada por $a/\sqrt{2}$, que se
calcula en aproximadamente \SI{3.480}{\angstrom} a \SI{100}{K}.
Los parámetros cristalográficos clave para el \ce{AgF} se resumen en
la Tabla~\ref{tab:cryst-data}.

\begin{table}[htbp!]
  \centering
  \caption{Resumen de Datos Cristalográficos del \ce{AgF} (a \SI{100}{K}).
Datos obtenidos de~\cite{Lozinsek2023}.}
  \label{tab:cryst-data}
  \begin{tabular}{@{}ll@{}}
    \toprule
    Propiedad                           & Valor                                 \\
    \midrule
    Sistema Cristalino               
& Cúbico                                \\
    Estructura Prototípica              & Sal de Roca (NaCl)                    \\
    Grupo Espacial (Hermann-Mauguin)    & $Fm\overline{3}m$ 
(No. 225)          \\
    Parámetro de Red, $a$               & \SI{4.92171(14)}{\angstrom}           \\
    Volumen de Celda Unitaria, $V$      & \SI{119.22(1)}{\angstrom\cubed}       \\
    Unidades Fórmula por Celda, $Z$     & 4        
                            \\
    Longitud de Enlace \ce{Ag}-F, $r_0$ & \SI{2.46085(7)}{\angstrom}            \\
    Número de Coordinación (Ag y F)     & 6 (Octaédrica)                        \\
    
\bottomrule
  \end{tabular}
\end{table}

% ==============================================================================
% SECTION 2: SYMMETRY
% ==============================================================================
\section{Análisis de Simetría del Cristal de \ce{AgF}}

\subsection{La Red de Bravais Cúbica Centrada en las Caras}
La estructura cristalina del \ce{AgF} se construye sobre una Red de
Bravais Cúbica Centrada en las Caras (FCC), una de las 14 posibles
redes de Bravais tridimensionales~\cite{ITC2016}.
Una red de Bravais es un
  arreglo infinito de puntos discretos donde la disposición y
  orientación parecen idénticas desde cualquier punto del
  arreglo~\cite{Ladd2012}.
La elección de una red FCC se indica
  explícitamente por la letra inicial 'F' en el símbolo del grupo
  espacial de Hermann-Mauguin, $Fm\overline{3}m$~\cite{Ladd2012}.
Mientras que la celda unitaria convencional de una red FCC es un cubo
  con puntos de red en cada esquina y en el centro de cada cara, una
  celda unitaria primitiva más fundamental, que contiene solo un punto
  de red, también puede definirse como un romboedro.
\subsection{Grupo Puntual y Grupo Espacial $Fm\overline{3}m$}
La simetría completa de un cristal se describe por su grupo espacial,
que abarca tanto la simetría traslacional de la red de Bravais como
las operaciones del grupo puntual (rotaciones, reflexiones,
inversiones) que dejan al menos un punto fijo~\cite{ITC2016}.
Para
el \ce{AgF}, el grupo espacial es $Fm\overline{3}m$ (No. 225 en las
Tablas Internacionales de Cristalografía), que es el grupo espacial
de mayor simetría para un sistema cúbico~\cite{Seidell2015}.
Una deconstrucción sistemática de la notación de Hermann-Mauguin
proporciona una descripción completa de la simetría:
\begin{description}
  \item[F:] Denota una red de Bravais Centrada en las Caras (Face-centered).
\item[$m\overline{3}m$:] Este es el símbolo del grupo puntual del
    cristal, también conocido como clase cristalina.
Pertenece al sistema
    cúbico y se lee como "m-barra-3-m".
\begin{itemize}
      \item El primer símbolo, \textbf{m}, se refiere a planos
        especulares perpendiculares a los ejes cristalográficos
        primarios, las direcciones $\langle 100 \rangle$ (es decir,
        planos paralelos a las caras del cubo).
\item El segundo símbolo, \textbf{$\overline{3}$}, indica la
        presencia de cuatro ejes de rotoinversión de orden 3 (three-fold)
        a lo largo de las diagonales espaciales del cubo, las
        direcciones $\langle 111 \rangle$.
\item El tercer símbolo, \textbf{m}, se refiere a planos
        especulares perpendiculares a las direcciones de las diagonales
        de las caras, las direcciones $\langle 110 \rangle$.
\end{itemize}
\end{description}
El grupo puntual $m\overline{3}m$ (o $O_h$ en notación de Schönflies)
es la clase holoédrica (de más alta simetría) del sistema cúbico y
contiene un total de 48 operaciones de simetría distintas.
\subsection{Elementos Principales de Simetría}
La alta simetría del cristal de \ce{AgF} se manifiesta en una rica
colección de elementos de simetría dentro de su celda unitaria.
La
estructura es centrosimétrica, lo que significa que posee un
centro de inversión. Otros elementos de simetría clave incluyen:
\begin{itemize}
  \item \textbf{Ejes de Rotación:} Tres ejes de rotación de orden 4
    (4-fold) que pasan por los centros de las caras opuestas, cuatro
    ejes de rotación de orden 3 (3-fold) a lo largo de las diagonales
    del cuerpo, y seis ejes de rotación de orden 2 (2-fold) que pasan
    por los centros de las aristas opuestas.
\item \textbf{Planos Especulares:} Nueve planos especulares en
    total, consistentes en tres planos paralelos a las caras del cubo y
    seis planos diagonales que bisecan los ángulos entre las caras.
\end{itemize}
Este grado excepcionalmente alto de simetría tiene profundas
consecuencias para las propiedades físicas del material.
Según el
Principio de Neumann, los elementos de simetría de cualquier propiedad
física de un cristal deben incluir los elementos de simetría del grupo
puntual del cristal.
Para un cristal cúbico perteneciente al grupo
  puntual $m\overline{3}m$, esto requiere que todas las propiedades
  tensoriales de segundo rango (como la conductividad eléctrica, la
  expansión térmica y la constante dieléctrica) deban ser
  isotrópicas.
Esto significa que la respuesta del material es
  independiente de la dirección. Por lo tanto, la clasificación
  cristalográfica abstracta $Fm\overline{3}m$ conduce directamente a la
  predicción física poderosa y comprobable de que el \ce{AgF} debe
  exhibir un comportamiento isotrópico para estas propiedades.
% ==============================================================================
% SECTION 3: BONDING
% ==============================================================================
\section{Naturaleza y Fuerza del Enlace \ce{Ag-F}}

\subsection{Electronegatividad y Carácter Iónico}
La naturaleza del enlace químico entre la plata y el flúor puede
evaluarse utilizando el concepto de electronegatividad, que mide la
tendencia de un átomo a atraer electrones compartidos.
Según la escala
de Pauling, el flúor es el elemento más electronegativo con un valor
de $\chi_F = 3.98$, mientras que la plata tiene un valor de
$\chi_{Ag} = 1.93$~\cite{Sanderson1960}.
La diferencia de electronegatividad,
$\Delta\chi$, proporciona un indicador confiable del tipo de enlace:
\[
  \Delta\chi = \chi_F - \chi_{Ag} = 3.98 - 1.93 = 2.05
\]
Una diferencia de esta magnitud ($\Delta\chi > 2.0$) indica
fuertemente que el enlace es predominantemente
iónico~\cite{Sanderson1960}.
Esto implica una transferencia significativa
  de un electrón desde el átomo de plata menos electronegativo al átomo
  de flúor altamente electronegativo, resultando en la formación de
  iones estables $\ce{Ag+}$ y $\ce{F-}$ unidos por atracción
  electrostática.
Sin embargo, es necesaria una visión más matizada. Se ha
observado que la mayoría de los compuestos de plata, debido a la
polarizabilidad del catión $\ce{Ag+}$, retienen un grado de carácter
covalente~\cite{ITC2016}.
Por lo tanto, aunque el modelo iónico puro
  ($\ce{Ag+}\ce{F-}$) es una aproximación altamente efectiva para
  describir las propiedades de estado sólido del \ce{AgF}, el enlace se
  describe con mayor precisión como un enlace covalente altamente polar.
\subsection{Energía de Red como Medida de Cohesión}
Para un sólido iónico, la energía cohesiva (la energía que mantiene
unido al cristal) se cuantifica mejor mediante la energía de red
($U_L$).
La energía de red se define formalmente como el cambio de
  entalpía requerido para separar un mol del compuesto iónico sólido en
  sus iones gaseosos constituyentes a una separación infinita (p. ej.,
$\ce{AgF(s) -> Ag+(g) + F-(g)}$).
Una energía de red grande y positiva
significa fuerzas electrostáticas fuertes y un cristal altamente
estable.

% ==============================================================================
% BIBLIOGRAPHY
% ==============================================================================
\printbibliography

\end{document}
