\documentclass[a4paper]{article}

\usepackage[utf8]{inputenc}
\usepackage[T1]{fontenc}
\usepackage[spanish,es-tabla]{babel}
\usepackage{amsmath}
\usepackage{amssymb}
\usepackage{authblk}
\usepackage[version=4]{mhchem}
\usepackage[margin=2.5cm]{geometry}

\usepackage{graphicx}
\usepackage{booktabs}
\usepackage{microtype}
\usepackage{csquotes}
\usepackage[backend=biber,style=numeric]{biblatex}
\usepackage{hyperref}
\usepackage[spanish]{cleveref}

\title{Reporte Resultados Preliminares de Clasificación de Material}
\author{Julian L. Avila-Martinez}
\affil{Programa Académico de Física, Facultad de Ciencias Matemáticas y Naturales \\
Universidad Distrital Francisco José de Caldas}
\date{}

\begin{document}

\maketitle

\begin{abstract}
	Este reporte detalla los hallazgos de la caracterización inicial de un
	compuesto desconocido. El material es un polvo cristalino, blanco e
	higroscópico. No es ferromagnético y exhibe un bajo punto de fusión sin
	carbonización, indicando que es una sal inorgánica. Las pruebas de
	solubilidad revelaron una alta solubilidad en agua desionizada pero
	insolubilidad en \ce{H2SO4} 0.1 M. La espectroscopía de Fluorescencia de
	Rayos X (XRF) confirmó la presencia de Plata (\ce{Ag}) como constituyente
	primario. Adicionalmente, se observó una reacción fotoquímica o de
	reducción al contacto con materia orgánica. Estos hechos establecen que el
	desconocido es una sal de plata inorgánica soluble en agua. El contra-anión
	específico aún está por determinarse.
\end{abstract}

\section{Introducción}
La identificación de materiales desconocidos es una tarea fundamental en la
ciencia de materiales. Este reporte describe la primera fase de análisis de una
muestra desconocida, enfocándose en datos objetivos, cualitativos y
semicuantitativos. El objetivo de esta fase inicial no es la identificación
definitiva, sino restringir lógicamente la clase del material y definir una
ruta clara para el subsecuente análisis cuantitativo.

\section{Metodología y Resultados Experimentales}

\subsection{Análisis Físico y Morfológico}
La muestra es un polvo microcristalino, fino y de color blanco. El material es
marcadamente higroscópico, absorbiendo fácilmente la humedad atmosférica
ambiental. Esta propiedad se observó al formarse aglomeraciones o ``grumos''
significativos en el contenedor de almacenamiento con el tiempo.

Se observó una interacción reactiva notable al contacto incidental con la
epidermis. Los granos de material adheridos a la piel experimentaron un
oscurecimiento progresivo, dejando manchas oscuras persistentes. Este fenómeno
es consistente con la reducción in situ de iones de plata a plata metálica
elemental ($\ce{Ag^0}$), catalizada por la luz o agentes reductores orgánicos
presentes en la piel.

\subsection{Evaluación de Propiedades Físicas}

\subsubsection{Susceptibilidad Magnética}
Se realizó una prueba cualitativa usando un imán permanente de alta intensidad
de neodimio (\ce{NdFeB}). El polvo no exhibió ninguna fuerza atractiva o
repulsiva observable. Por lo tanto, el material se clasifica como no
ferromagnético (es decir, diamagnético o débilmente paramagnético).

\subsubsection{Comportamiento Térmico Cualitativo}
Se calentó una pequeña cantidad de la muestra en una cuchara de laboratorio
sobre la llama de un mechero Bunsen. El material se fundió rápidamente a una
temperatura cualitativamente baja. El material fundido resultante fue
translúcido y blanco en los bordes. De manera crucial, no se observó
carbonización significativa, humo o residuo de carbón negro característico de
la combustión de compuestos orgánicos. Este hallazgo indica fuertemente que el
desconocido es una sal inorgánica.

\subsection{Perfil de Solubilidad}
Se evaluó la solubilidad del material en dos solventes clave:
\begin{itemize}
	\item \textbf{Agua Desionizada:} La sustancia es altamente soluble en agua,
		disolviéndose rápidamente a temperatura ambiente para formar una
		solución clara e incolora.
	\item \textbf{Ácido Sulfúrico 0.1 M (\ce{H2SO4}):} Cuando se añadió a la
		solución de \ce{H2SO4} 0.1 M, el polvo no pareció disolverse y se
		asentó en el fondo como un sólido blanco.
\end{itemize}

\subsection{Análisis Composicional (XRF)}
Se realizó un análisis elemental inicial no destructivo usando espectroscopía
de Fluorescencia de Rayos X (XRF). El espectro resultante, presentado en la
\cref{fig:xrf_spectrum}, identifica claramente las líneas de emisión
características de la Plata (\ce{Ag}). Esto proporciona una confirmación
definitiva de que el desconocido es un compuesto a base de plata.

\begin{figure}[htbp]
	\centering
	\includegraphics[width=0.8\textwidth]{./images/xrf-spectrum.pdf}
	\caption{Espectro de Fluorescencia de Rayos X (XRF) de la muestra
		desconocida. Se observan picos prominentes correspondientes a las
		transiciones electrónicas de la Plata (\ce{Ag}), confirmando su presencia
	como el catión metálico principal.}
	\label{fig:xrf_spectrum}
\end{figure}

\section{Análisis de Resultados}
Los datos experimentales permiten una restricción lógica y directa de la
identidad del desconocido.

\begin{enumerate}
	\item Los datos de XRF confirman un compuesto de plata.
	\item La prueba térmica (sin carbonización) lo establece como una sal
		inorgánica.
	\item La alta solubilidad en agua confirma que es una sal de plata
		inorgánica soluble en agua.
	\item El oscurecimiento en contacto con la piel valida la presencia de
		iones de plata reducibles.
\end{enumerate}

La observación de insolubilidad en \ce{H2SO4} 0.1 M es de importancia crítica.
Este comportamiento está en aparente contradicción con su alta solubilidad en
agua. No se trata de una simple insolubilidad física, sino del resultado
esperado de una reacción de precipitación inmediata. A medida que la sal
soluble en agua se disuelve, libera iones \ce{Ag+}, que reaccionan
inmediatamente con los iones \ce{SO4^2-} del ácido para formar Sulfato de Plata
(\ce{Ag2SO4}), una sal blanca e insoluble:
$$ \ce{2Ag+ (aq) + SO4^2- (aq) -> Ag2SO4 (s)} $$
Este nuevo precipitado es visualmente indistinguible del polvo original, dando
la apariencia de insolubilidad. Por lo tanto, la prueba ácida confirma
exitosamente la presencia de iones \ce{Ag+} solubles. La tarea analítica
central se reduce ahora a identificar el contra-anión desconocido del material.

\section{Trabajo Futuro Propuesto}
Para identificar definitivamente el compuesto, se debe caracterizar el anión
desconocido. Se proponen los siguientes pasos analíticos:

\begin{itemize}
	\item \textbf{Difracción de Rayos X (XRD):} Este es el objetivo principal.
		El análisis por XRD del polvo original producirá un patrón de
		difracción. La comparación de este patrón con una referencia en la base
		de datos ICDD proporcionará la identificación definitiva de la fase
		cristalina de la sal completa.
	\item \textbf{Espectroscopía Vibracional (FTIR/Raman):} Se emplearán tanto
		la Espectroscopía Infrarroja por Transformada de Fourier (FTIR) como la
		espectroscopía Raman. Estas técnicas son ideales para identificar
		aniones poliatómicos (p.ej., \ce{NO3-}, \ce{ClO4-}, \ce{SO4^2-}, etc.)
		por sus modos vibracionales únicos y característicos.
	\item \textbf{Pruebas Químicas Puntuales:} Como método de cribado rápido y
		de bajo costo, se pueden emplear pruebas químicas dirigidas (p.ej., la
		prueba del anillo pardo para nitrato) para detectar aniones comunes
		sospechosos.
\end{itemize}

\end{document}
