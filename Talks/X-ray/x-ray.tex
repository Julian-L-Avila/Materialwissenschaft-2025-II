\documentclass[11pt]{article}

\usepackage[utf8]{inputenc}
\usepackage[T1]{fontenc}
\usepackage[spanish]{babel}
\usepackage{amsmath}
\usepackage{amssymb}
\usepackage{graphicx}
\usepackage[margin=1in]{geometry}
\usepackage{hyperref}
\usepackage[version=4]{mhchem}
\usepackage{siunitx}

\hypersetup{
  colorlinks=true,
  linkcolor=black,
  citecolor=black,
  filecolor=black,
  urlcolor=blue,
}

\title{
  \textbf{Resumen: Difracción de Rayos X}
}
\author{Julian L. Avila-Martinez}
\date{26 de Septiembre 2025}

\begin{document}

\maketitle
\thispagestyle{empty}

\begin{center}
  \textbf{Ponente:} Cristian \\
  \textbf{Afiliación:} Universidad Nacional de Colombia
\end{center}

\section{Introducción}

Este informe resume los conceptos esenciales presentados en la charla de
Cristian sobre los fundamentos y aplicaciones prácticas de la difracción de
rayos X (DRX). El enfoque se centra en el papel crítico del polimorfismo en la
industria farmacéutica y en los fundamentos físico-técnicos de la técnica de
DRX. La DRX es una herramienta indispensable para el estudio de la simetría y el
orden en la materia condensada, cuya correcta aplicación exige una profunda
comprensión de sus principios.

\section{El Fenómeno del Polimorfismo}

\textbf{Definición:} El polimorfismo es la capacidad de un material sólido de
existir en múltiples estructuras cristalinas distintas, conocidas como
polimorfos. Aunque son químicamente idénticos, los polimorfos presentan
diferentes arreglos en la red cristalina y/o conformaciones moleculares. Esto
conduce a diferentes energías de red y, en consecuencia, a diferentes energías
libres de Gibbs ($G = U + PV - TS$). El polimorfo termodinámicamente más estable
a una temperatura y presión dadas es aquel con la menor energía libre de Gibbs.
Las formas metaestables pueden persistir debido a altas barreras cinéticas para
la transformación.

\subsection*{Impacto en la Farmacología}
Las propiedades fisicoquímicas, en particular la velocidad de disolución, pueden
diferir drásticamente entre polimorfos, lo que tiene consecuencias
significativas en la biodisponibilidad y seguridad de los medicamentos.

\begin{itemize}
  \item \textbf{Paracetamol (Acetaminofén):} Este fármaco existe en dos formas
    principales. La Forma I (estable) tiene un tiempo de disolución de
    aproximadamente 8 horas, mientras que la Forma II (metaestable) se disuelve
    en solo 2 horas. La administración involuntaria de la forma incorrecta
    podría conducir a una sobredosis rápida y a una potencial toxicidad.
  \item \textbf{Rifaximina:} Este antibiótico existe en 12 formas de hidratos,
    de las cuales solo una, la forma delta, es terapéuticamente activa y está
    aprobada por la FDA. Las otras formas son insolubles e ineficaces.
\end{itemize}
Técnicas como la cromatografía son incapaces de detectar estas diferencias, ya
que requieren la destrucción de la estructura cristalina. Esto subraya la
importancia de la DRX como herramienta esencial de control de calidad.

\section{Principios Físicos de la Difracción de Rayos X}

La DRX es una técnica no destructiva que aprovecha la interferencia constructiva
de rayos X monocromáticos dispersados por los electrones en una red cristalina.

\subsection*{Ley de Bragg y Espacio Recíproco}
La condición para la interferencia constructiva se describe mediante la Ley de
Bragg:
$$ n\lambda = 2d_{hkl} \sin\theta $$
Donde $n$ es un número entero, $\lambda$ es la longitud de onda de los rayos X,
$d_{hkl}$ es el espaciado interplanar para una familia de planos de red con
índices de Miller $(hkl)$, y $\theta$ es el ángulo de dispersión.

Desde una perspectiva más fundamental del espacio recíproco, los picos de
difracción ocurren cuando el vector de dispersión $\Delta\vec{k} =
\vec{k}_{\text{final}} - \vec{k}_{\text{inicial}}$ coincide con un vector de la
red recíproca $\vec{G}_{hkl}$. Para la dispersión elástica,
$|\vec{k}_{\text{final}}| = |\vec{k}_{\text{inicial}}| = 2\pi/\lambda$. Esta
condición geométrica es la base de la Ley de Bragg y se visualiza elegantemente
mediante la construcción de la \textbf{Esfera de Ewald}. Se observa un pico de
difracción precisamente cuando un punto de la red recíproca $\vec{G}_{hkl}$ yace
sobre la superficie de la Esfera de Ewald.

\section{Configuración Experimental (Geometría Bragg-Brentano)}

El difractómetro consta de tres componentes principales: una fuente de rayos X,
un goniómetro con portamuestras y un detector.
\begin{enumerate}
  \item \textbf{Generación de Rayos X:} Los electrones, emitidos por un
    filamento caliente (emisión termoiónica), se aceleran hacia un blanco
    metálico (p. ej., Cu). Esto produce un espectro continuo
    (\textit{Bremsstrahlung}) y rayos X \textbf{característicos} discretos,
    específicos del elemento. La radiación $K_\alpha$ (transición $L \to K$) se
    utiliza para los experimentos de difracción.
  \item \textbf{Filtrado:} Para obtener una fuente monocromática, se filtra la
    radiación $K_\beta$ (transición $M \to K$). Para una fuente de cobre, se
    utiliza un \textbf{filtro de Níquel}, ya que el borde de absorción K del
    níquel se encuentra energéticamente entre las líneas de emisión $K_\alpha$ y
    $K_\beta$ del cobre.
  \item \textbf{Geometría y Detección:} En la geometría Bragg-Brentano
    ($\theta-2\theta$), la fuente de rayos X y el detector giran de forma
    sincronizada. Esto equivale a la rotación de la red recíproca en relación
    con la Esfera de Ewald. El detector mide la intensidad en función del ángulo
    $2\theta$.
\end{enumerate}

\subsection*{Preparación de la Muestra}
Para la difracción de polvo, la muestra debe estar finamente molida para
asegurar una orientación aleatoria de los cristalitos, y prensada de forma plana
en el portamuestras para minimizar errores geométricos. Una altura de muestra
incorrecta conduce a desplazamientos y ensanchamiento de los picos.

\section{Interpretación de Datos de DRX}

El difractograma resultante es una ``huella dactilar'' inequívoca de la fase
cristalina.
\begin{itemize}
  \item \textbf{Materiales Cristalinos:} Muestran \textbf{picos de Bragg} agudos
    y bien definidos a ángulos $2\theta$ específicos, que corresponden a los
    espaciados de la red ($d_{hkl}$).
  \item \textbf{Materiales Amorfos:} Carecen de orden de largo alcance. Generan
    una señal ancha y difusa, conocida como \textbf{halo amorfo}, en lugar de
    picos agudos. La posición del halo puede proporcionar información sobre la
    distancia promedio al vecino más cercano.
\end{itemize}

\section{Dispersión Anómala de Rayos X}

El factor de dispersión atómica $f$ describe la capacidad de dispersión de un
átomo y es una cantidad compleja:
$$ f(\vec{q}, E) = f_0(\vec{q}) + f'(E) + i f''(E) $$
Donde $f_0$ es el término de dispersión de Thomson, y $f', f''$ son las
correcciones de dispersión anómala. Estos términos se vuelven significativos
cuando la energía de los rayos X incidentes $E$ se sintoniza cerca de un borde
de absorción de un elemento. La dispersión anómala de rayos X a bajo ángulo
(ASAXS) aprovecha esta dependencia de la energía para aislar matemáticamente la
contribución a la dispersión de un elemento específico mediante mediciones a
varias energías justo por debajo de su borde de absorción, obteniendo así
información estructural específica del elemento.

\section*{Conclusión}

La presentación trato la relevancia fundamental de la DRX en la
ciencia de materiales moderna y, en particular, para el control de calidad en la
industria farmacéutica. La capacidad de la técnica para distinguir entre
polimorfos la convierte en una herramienta indispensable para garantizar la
seguridad y eficacia de los medicamentos.

\end{document}
